%%%%%%%%%%%%%%%%%%%%%%%%%%%%%%%%%%%%%%%%%
% Tecnológico de Costa Rica/Instructivo de Laboratorio de Instrumentación I
% LaTeX Template
% Version 3.1 (25/3/14)
%
% This template has been downloaded from:
% http://www.LaTeXTemplates.com
%
% Original author:
% Linux and Unix Users Group at Virginia Tech Wiki 
% (https://vtluug.org/wiki/Example_LaTeX_chem_lab_report)
%
% License:
% CC BY-NC-SA 3.0 (http://creativecommons.org/licenses/by-nc-sa/3.0/)
%
%%%%%%%%%%%%%%%%%%%%%%%%%%%%%%%%%%%%%%%%%

%----------------------------------------------------------------------------------------
%	PACKAGES AND DOCUMENT CONFIGURATIONS
%----------------------------------------------------------------------------------------

\documentclass[12pt,letterpaper]{report}
\makeatletter
\def\input@path{{../common/}{../guide/}{../data/}{../code/}}
\makeatother
\usepackage{amsmath}
\usepackage{amssymb}
\usepackage{siunitx}
\usepackage{float}
\usepackage{tikz}
\usepackage{url}
\usepackage[siunitx,american,RPvoltages]{circuitikz}
\ctikzset{capacitors/scale=0.7}
\ctikzset{diodes/scale=0.7}
\usepackage{tabularx}
\newcolumntype{C}{>{\centering\arraybackslash}X}
\renewcommand\tabularxcolumn[1]{m{#1}}% for vertical centering text in X column
\usepackage{tabu}
\usepackage[spanish,es-tabla,activeacute]{babel}
\usepackage{babelbib}
\usepackage{booktabs}
\usepackage{pgfplots}
\usepackage{hyperref}
\hypersetup{colorlinks = true,
            linkcolor = black,
            urlcolor  = blue,
            citecolor = blue,
            anchorcolor = blue}
\usepgfplotslibrary{units, fillbetween} 
\pgfplotsset{compat=1.16}
\usepackage{bm}
\usetikzlibrary{arrows, arrows.meta, shapes, 3d, perspective, positioning}
\renewcommand{\sin}{\sen} %change from sin to sen
\usepackage{bohr}
\setbohr{distribution-method = quantum,insert-missing = true}
\usepackage{elements}
\usepackage{verbatim}
\input{arduinoLanguage.tex}
\usepackage{csvsimple}
\usepackage{geometry} 
\geometry{left=18mm,right=18mm,top=21mm,bottom=21mm,headheight=15pt}

\setlength\parindent{0pt} % Removes all indentation from paragraphs

\renewcommand{\labelenumi}{\alph{enumi}.} % Make numbering in the enumerate environment by letter rather than number (e.g. section 6)
\usepackage{fancyhdr}
\pagestyle{fancy}

%----------------------------------------------------------------------------------------
\lhead{Instructivo de Laboratorio de Instrumentación I}
\rhead{\begin{picture}(0,0) \put(-60,0){\includegraphics[width=20mm]{logo.png}} \end{picture}}
\newcommand{\obj}{Objetivos}
\newcommand{\mat}{Materiales y equipo}
\newcommand{\pro}{Procedimiento}
\newcommand{\capacidad}{Al finalizar este laboratorio el estudiante estará en capacidad de:}
\newcommand{\antesde}{Antes de empezar el laboratorio presente el siguiente cuestionario lleno.}
%----------------------------------------------------------------------------------------
%	DOCUMENT INFORMATION
%----------------------------------------------------------------------------------------


\addto\captionsspanish{\renewcommand{\chaptername}{Laboratorio}}
\addto\captionsspanish{\renewcommand{\tablename}{Tabla}}
\begin{document}
% ------------------------------------------------------------
\begin{titlepage}
    \begin{center}
\vspace*{1in}
\begin{figure}[htb]
\begin{center}
\includegraphics[width=11cm]{logo.png}
\end{center}
\end{figure}
\vspace*{0.4in}
\begin{Large}
Escuela de Física\\
\vspace*{0.15in}
Ingeniería Física\\
\vspace*{0.8in}
\end{Large}
\vspace*{0.2in}
\begin{Large}
\textbf{Instructivo de Laboratorio} \\
\end{Large}
\vspace*{0.3in}
\begin{large}
Instrumentación I\\
\end{large}
\vspace*{2.5in}
\begin{Large}
\textbf{\today}\\
Versión: 0.1\\
\end{Large}
\rule{80mm}{0.1mm}\\
\vspace*{0.1in}
\begin{large}
Realizado por: Juan J. Rojas\\
\end{large}
\end{center}
\end{titlepage}
% ------------------------------------------------------------
\tableofcontents
% ------------------------------------------------------------
\chapter{Medición de estímulos eléctricos}
\section{\obj}
\capacidad
\begin{itemize}
    \item Configurar un sensor digital para lograr una medición eléctrica con la mayor resolución posible.
    \item Calcular el error asociado a la medición utilizando una multimetro digital de alta precisión.
\end{itemize}

\section{\mat}
\textbf{A suministrar por la Escuela:}
\begin{itemize}
    \item 1 ARDUINO UNO R4 MINIMA
    \item 1 sensor digital de voltaje, corriente y potencia (INA219)
    \item 1 mini breadboard
    \item Cables de interconexión (jumpers) macho-macho
\end{itemize}

\section{\pro}
\begin{enumerate}
        
    \item Realice las conexiones del circuito tal y como se indica en la Figura \ref{fig:L1F1}.
    \item Abra el programa Arduino IDE

    \begin{figure}[H]
        \tikzset{dig/.style={muxdemux, muxdemux def={Lh=5, Rh=5, NL=2, NB=0, NR=0, w=2}}}
        \centering
        \begin{circuitikz} 
            \draw 
            (0,3.5) 
            node[dig] (p){\rotatebox{90}{\small POWER}}
            (0,0) 
            node[dig] (m){\rotatebox{90}{\small ANALOG IN}};
            \draw (m.blpin 1) node[above left]{\small A0};
            \draw (m.blpin 2) node[above left]{\small A1};
            \draw (p.blpin 1) node[above left]{\small 5V};
            \draw (p.blpin 2) node[above left]{\small GND};
            \draw
            (-4,6)
                to[american potentiometer,n=mypot, l_=0$\sim$\SI{10}{\kilo\ohm},i=$i$]
            (-4,4) 
            (mypot.wiper) |- (p.blpin 1)
            (-4,4)
                to[R, l_=\SI{4.7}{\kilo\ohm}, v^=$v_1$]
            (-4,1) 
                to[R, l_=\SI{2.2}{\kilo\ohm}, v^=$v_2$]
            (-4,-2)
            -- ++(1.8,0)
            |-
            (p.blpin 2)
            ;
            \draw[orange]
            (-4,3.8) -- ++ (1.5,0)
            |-
            (m.blpin 1)
            ;
            \draw[brown]
            (-4,0.8) -- ++ (1.2,0)
            |-
            (m.blpin 2)
            ;
            \draw[dashed,blue]
            (-2,6) -- (1.5,6)node[midway, below, align=center]{ARDUINO UNO\\ R4 MINIMA} -- (1.5,-1.5) -- (-2,-1.5) -- cycle;
        \end{circuitikz}
        \caption{Conexión de circuito para el Laboratorio 1}
        \label{fig:L1F1}
    \end{figure}

    \item Incluya el siguiente código en el Arduino IDE
    {\scriptsize 
        \lstinputlisting[language=Arduino, numbers=none, showstringspaces=false]{L01/L01.ino}
    }

    \item Tomando en cuenta el valor de la resistencia que se le asignó, calcule el valor de la corriete esperada en el circuito utilizando la Ley de Ohm.
    \item Investigue en el datasheet que significa el parametro PGA y como se relaciona con el rango del sensor. Ver 8.5.1
    \item Configure el sensor para que el rango de medición sea el más adecuado para la resistencia que se le asignó, suba el código modificado al Arduino.
    \item Realice tres mediciones de 20s de duración con una frecuencia de 1Hz al mismo tiempo que mide la corriente con un multímetro digital de alta precisión.
    \item Calibre el sensor utilizando un multímetro digital de alta precisión y cambie nuevamente el código para que el valor medido por el sensor sea lo más cercano posible al valor medido por el multímetro. Ver 8.5.2.1. Suba el código modificado al Arduino.
    \item Realice tres mediciones de 20s de duración con una frecuencia de 1Hz al mismo tiempo que mide la corriente con un multímetro digital de alta precisión.
    \item Para cada ser de medicines calcule lo siguiente e incluya en una tabla:
        \begin{itemize}
            \item El valor promedio, $\overline{R}$ 
            \item La desviación estándar, $\sigma$
            \item El valor de la incertidumbre estándar, $\sigma_x = \sigma / \sqrt{n}$
            \item La exactitud
            \item La precisión
            \item La repetibilidad
        \end{itemize}
        
    \item Tome en cuenta lo siguiente:
        \begin{itemize}
            \item La exactitud se calculará como el mayor error relativo obtenido en cualquier punto de los datos en todas las corridas de medición.
            \item La precisión se calculará como la mayor desviación estándar entre todas las corridas de medición.
            \item La repetibilidad se calculará de la siguiente manera:
            \begin{equation*}
                \mathrm{repetibilidad} = \sqrt{\dfrac{\sum_{i=1}^n(\mathrm{error\,absoluto})^2}{n}}
            \end{equation*}
        \end{itemize}


    % \section{Resultados}

    % \begin{table}[H]
    %     \centering
    %     \caption{Mediciones tomadas a una frecuencia de \SI{1}{\hertz}}
    %     \vspace{0.5cm}
    %     \begin{tabular}{ccccccc}%
    %     \toprule
    %     \bfseries &  \multicolumn{2}{c}{\textbf{corrida 1}} & \multicolumn{2}{c}{\textbf{corrida 2}} & \multicolumn{2}{c}{\textbf{corrida 3}}\\
    %     \bfseries $t$ & \bfseries $i$ & \bfseries $v_1$ & \bfseries $i$ & \bfseries $v_1$ & \bfseries $i$ & \bfseries $v_1$\\
    %     {[\si{\second}]} & [\si{\ampere}] & [\si{\volt}] & [\si{\ampere}] & [\si{\volt}] & [\si{\ampere}] & [\si{\volt}]\\
    %     \midrule
    %     \csvreader[
    %         late after line=\\,
    %         late after last line=,
    %         before reading={\catcode`\#=12},
    %         after reading={\catcode`\#=6}]%
    %         {data.csv}{1=\t,2=\ci,3=\vi,4=\cii,5=\vii,6=\ciii,7=\viii}{\t &\ci & \vi &\cii & \vii &\ciii & \viii}\\
    %         \bottomrule
    %     \end{tabular}
    %     \label{tab:L1T1}
    % \end{table}
\end{enumerate}
\input{02_temperatura}
\chapter{Sensores de presión}
\section{\obj}
\capacidad
\begin{itemize}
\item Utilizar un sensor para medir presión absoluta.
\end{itemize}

\section{\mat}
\begin{itemize}
\item 1 Arduino UNO R4 MINIMA
\item 1 sensor digital de presión BMP280
\item 1 mini breadboard de 170 pines
\item Jumpers macho-macho
\end{itemize}

\section{Actividad 1}
\subsection{\pro}
\begin{enumerate}
    \item Conecte el sensor BMP280 tal como se muestra en la Figura \ref{fig:pres1}
    \item Copie el siguiente código en el IDE de Arduino (el codigo se puede encontrar en: \href{https://github.com/juanjorojash/instrumentacion_I/blob/master/code/L03/L03.ino}{GitHub})
    \lstinputlisting[language=Arduino,numbers=none]{../code/L03/L03.ino}  
    \item Analice cada linea de código y cuando entienda su funcionamiento corra el \emph{sketch} usando el botón \emph{Upload}
    \item Haga contacto con el encapsulado del sensor de forma que su dedos calienten el sensor, observe como cambia el valor en la terminal. 
    \item Modifique el programa de forma que lo que se imprima en la terminal sea algo como esto:
\begin{verbatim}
24.8,806.1
24.7,806.3
\end{verbatim}
    \item Instale las librerías \emph{pyserial}, \emph{matplotlib}, \emph{drawnow} y \emph{datetime} en Python para ser utilizadas luego.
    \item Copie el siguiente código en su IDE de Python. (el codigo se puede encontrar en: \href{https://github.com/juanjorojash/instrumentacion_I/blob/master/code/L03/L03.py}{GitHub})
    \lstinputlisting[language=Python,numbers=none]{../code/L03/L03.py}
    \item Modifique el puerto COM para que coincida con el puerto en el que esta conectado el Arduino UNO R4 MINIMA
    \item Modifique el código para que cada toma de datos se almacene en un archivo $.csv$ y para que se genere una gráfica en tiempo real.
    \item Realice tres tomas de datos de 60 segundos cada una
\end{enumerate}

\begin{figure}[H]
    \tikzset{comm/.style={muxdemux, muxdemux def={Lh=5, Rh=5, NL=2, NB=0, NR=0, w=2}}}
    \tikzset{BMP280/.style={muxdemux, muxdemux def={Lh=5, Rh=5, NL=0, NB=0, NR=6, w=2}}}
    \centering
    \begin{circuitikz} 
        \draw (0,3.5) node[comm] (m){\rotatebox{90}{\small communication}}
        (0,0) node[comm] (p){\rotatebox{90}{\small power}}
        ;
        \draw (m.blpin 1) node[above left]{\small SDA};
        \draw (m.blpin 2) node[above left]{\small SCL};
        \draw (p.blpin 1) node[above left]{\small GND};
        \draw (p.blpin 2) node[above left]{\small +3V3};
        \draw (-5,-3) node[BMP280,rotate=90] (s){\rotatebox{-90}{\small BMP280}}
        (s.brpin 1) node[above right,rotate=90]{\scriptsize SDO}
        (s.brpin 2) node[above right,rotate=90]{\scriptsize CSB}
        (s.brpin 3) node[above right,rotate=90]{\scriptsize SDA}
        (s.brpin 4) node[above right,rotate=90]{\scriptsize SCL}
        (s.brpin 5) node[above right,rotate=90]{\scriptsize GND}
        (s.brpin 6) node[above right,rotate=90]{\scriptsize VCC}
        ;
        \draw[blue]
        (p.blpin 2)
        -|
        (s.brpin 6)
        (p.blpin 2)
        -|
        (s.brpin 1)
        ;
        \draw[green]
        (p.blpin 1)
        -|
        (s.brpin 5)
        ;
        \draw[red]
        (m.blpin 2)
        -| 
        (s.brpin 4)
        ;
        \draw[brown]
        (m.blpin 1)
        -| 
        (s.brpin 3)
        ;
        \draw[dashed,blue]
        (-2.5,6) -- (1,6)node[midway, below]{UNO R4 MINIMA} -- (1,-1.5) -- (-2.5,-1.5) -- cycle;
    \end{circuitikz}
    \caption{Conexión de sensor de presión absoluta BMP280}
    \label{fig:pres1}
\end{figure}

\subsection{Análisis}
\begin{enumerate}
    \item Determine la presión atmosférica local usando el promedio de las mediciones que las tres tomas de datos, incluya la precisión de la medición 
    \item Determine la altitud local usando el valor de presión calculado ¿Es este valor cercano al esperado para la altitud del campus (1407 msnm)?
\end{enumerate}

\section{Actividad 2}
\subsection{\pro}
\begin{enumerate}
    \item Usando la misma configuración anterior, realice una nueva toma de datos de 60s en la que lleve el sensor lo mas cerca del piso posible y luego lo mas alto posible lentamente y repita durante los 60 segundos de la toma
\end{enumerate}
\subsection{Análisis}
\begin{enumerate}
    \item ¿Se puede observar el cambio de altura en las mediciones?
    \item Según la hoja de datos del sensor ¿Debería detectar estos cambios?
    \item ¿El tiempo entre mediciones es realmente un segundo?
\end{enumerate}

\chapter{Sensores de peso}
\section{\obj}
\capacidad
\begin{itemize}
\item Implementar un sistema de medición de peso utilizando el módulo HX711 con calibración multipunto
\end{itemize}

\section{\mat}
\begin{itemize}
\item 1 Arduino UNO R4 MINIMA
\item Celda de carga de \SI{1}{\kilogram}
\item 1 amplicador HX711
\item Pesas de referencia (500g, 1000g, 1500g)
\item Cable USB y computadora con Arduino IDE
\end{itemize}

\section{Actividad 1}
\subsection{\pro}
\begin{enumerate}
    \item Conecte el amplificador HX711 tal como se muestra en la Figura %\ref{fig:pes1}
    \item Copie el siguiente código en el IDE de Arduino (el codigo se puede encontrar en: \href{https://github.com/juanjorojash/instrumentacion_I/blob/master/code/L04/L04.ino}{GitHub})
    \lstinputlisting[language=Arduino,numbers=none]{../code/L04/L04.ino}  
    \item Analice cada linea de código y cuando entienda su funcionamiento corra el \emph{sketch} usando el botón \emph{Upload}
    \item En la terminal serial ingrese la letra "c", esto iniciará un proceso de calibración, siga las instrucciones del menú, calibre con peso máximo de 200 gramos.
    \item Copie el siguiente código en su IDE de Python. (el codigo se puede encontrar en: \href{https://github.com/juanjorojash/instrumentacion_I/blob/master/code/L04/L04.py}{GitHub})
    \lstinputlisting[language=Python,numbers=none]{../code/L04/L04.py}
    \item Modifique el puerto COM para que coincida con el puerto en el que esta conectado el Arduino UNO R4 MINIMA.
    \item Coloque un peso de 100 gramos y realice una toma de datos de 20 muestras.
    \item Coloque un peso de 300 gramos y realice una toma de datos de 20 muestras.
    \item Repita este procedimiento para calibraciones con peso máximo de 500 gramos y 900 gramos. 
\end{enumerate}

% \begin{figure}[H]
%     \tikzset{comm/.style={muxdemux, muxdemux def={Lh=5, Rh=5, NL=2, NB=0, NR=0, w=2}}}
%     \tikzset{BMP280/.style={muxdemux, muxdemux def={Lh=5, Rh=5, NL=0, NB=0, NR=6, w=2}}}
%     \centering
%     \begin{circuitikz} 
%         \draw (0,3.5) node[comm] (m){\rotatebox{90}{\small communication}}
%         (0,0) node[comm] (p){\rotatebox{90}{\small power}}
%         ;
%         \draw (m.blpin 1) node[above left]{\small SDA};
%         \draw (m.blpin 2) node[above left]{\small SCL};
%         \draw (p.blpin 1) node[above left]{\small GND};
%         \draw (p.blpin 2) node[above left]{\small +3V3};
%         \draw (-5,-3) node[BMP280,rotate=90] (s){\rotatebox{-90}{\small BMP280}}
%         (s.brpin 1) node[above right,rotate=90]{\scriptsize SDO}
%         (s.brpin 2) node[above right,rotate=90]{\scriptsize CSB}
%         (s.brpin 3) node[above right,rotate=90]{\scriptsize SDA}
%         (s.brpin 4) node[above right,rotate=90]{\scriptsize SCL}
%         (s.brpin 5) node[above right,rotate=90]{\scriptsize GND}
%         (s.brpin 6) node[above right,rotate=90]{\scriptsize VCC}
%         ;
%         \draw[blue]
%         (p.blpin 2)
%         -|
%         (s.brpin 6)
%         (p.blpin 2)
%         -|
%         (s.brpin 1)
%         ;
%         \draw[green]
%         (p.blpin 1)
%         -|
%         (s.brpin 5)
%         ;
%         \draw[red]
%         (m.blpin 2)
%         -| 
%         (s.brpin 4)
%         ;
%         \draw[brown]
%         (m.blpin 1)
%         -| 
%         (s.brpin 3)
%         ;
%         \draw[dashed,blue]
%         (-2.5,6) -- (1,6)node[midway, below]{UNO R4 MINIMA} -- (1,-1.5) -- (-2.5,-1.5) -- cycle;
%     \end{circuitikz}
%     \caption{Conexión de sensor de presión absoluta BMP280}
%     \label{fig:pes1}
% \end{figure}

\subsection{Análisis}
\begin{enumerate}
    \item ¿Cuál es la resolución teórica del módulo HX711 si trabaja con un convertidor de 24 bits y una celda de carga de \SI{1}{\kilogram}? ¿Qué implicaciones tiene esto en la medición de masas pequeñas? Como recomendación, utilizar la siguiente ecuación:
    \begin{equation*}
        \text{Resolución} = \frac{\text{Capacidad de la celda}}{2^{\text{Número de bits}}}
    \end{equation*} 
    \item ¿Observó alguna diferencia entre las mediciones realizadas con diferentes puntos de calibración?¿Cuál fue mejor?
\end{enumerate}

\printbibliography

\end{document}