\chapter{Características generales de instrumentos}
\section{\obj}
\capacidad
\begin{itemize}
\item Utilizar la fuente de tensión en voltaje continuo.
\item Utilizar un registrador para tomar datos de sensores.
\item Utilizar sensores para medir corriente y voltaje.
\item Aplicar los conocimientos relacionados a los instrumentos de medición, para el cálculo de valores tales como: exactitud, precisión, error porcentual, desviación estándar, incertidumbre, repetibilidad.
\end{itemize}

\section{\mat}
\begin{itemize}
\item 1 fuente de corriente continua
\item 1 interfaz ScienceWorkshop\,\textsuperscript{\textregistered}\,750
\item 1 sensor de corriente CI-6556
\item 1 sensor de voltaje UI-5100
\item 2 Cables de conexión tipo Banana macho-macho
\end{itemize}


\section{\pro}
\begin{enumerate}
\item Realice las conexiones del circuito tal y como se indica en la Figura \ref{fig:L1F1}.
\item Encienda la interfaz ScienceWorshop\,\textsuperscript{\textregistered}\,750 y realice la configuración en Capstone
\item Configure la toma de datos a una frecuencia de \SI{1}{\hertz}, condición de inicio: ninguna, condición de parada: basada en el tiempo (\SI{21}{\second}) 
\item Encienda la fuente y colóquela en \SI{0.1}{\volt}

\begin{figure}[H]
    \centering
    \begin{circuitikz} 
        \draw
        (0,0) 	
            to[V, l=$V_f$] 
        (0,4)
        	to[short] 
        (3,4)
        	to[rmeter, t=A]
        (3,2) 
            to[R, l=$R_{sh}$]
        (3,0)-- (0,0)
        (3,4) -- (5,4)
            to[rmeter, t=V] 
        (5,0) -- (3,0)
        ;
        \draw[dashed,blue]
        (2,3.8) -- (4,3.8) -- (4,0.2) -- (2,0.2) -- (2,3.8);
    \end{circuitikz}
    \caption{Medición de corriente y voltaje en un circuito.}
    \label{fig:L1F1}
\end{figure}

\item Inicie la toma de datos y proceda a incrementar el valor del voltaje poco a poco de forma que a los 10 segundos no haya sobrepasado los \SI{0.9}{\ampere} luego baje lentamente de hasta llegar a \SI{0.1}{\volt}
\item Repita la medición tres veces siguiendo el mismo procedimiento, tome en cuenta que los resultados no tienen que ser iguales, complete la Tabla \ref{tab:L1T1}
\item Calcule indirectamente (Ley de Ohm) para cada caso el valor de $R_{sh}$.
\item Realice otra tabla con los valores de $R_{sh}$ para cada corrida y grafique los resultados obtenidos.
\item Para el conjunto de valores de $R_{sh}$, calcule lo siguiente e incluya en una tabla:
    \begin{itemize}
        \item El valor promedio, $\overline{R_{sh}}$ 
        \item La desviación estándar, $\sigma$
        \item El valor promedio de error relativo tomando \SI{1}{\ohm} como valor real
        \item El valor de la incertidumbre estándar, $\sigma_x = \sigma / \sqrt{n}$
        \item La exactitud
        \item La precisión
        \item La repetibilidad
    \end{itemize}
    
\item Tome en cuenta lo siguiente:
    \begin{itemize}
        \item La exactitud se calculará como el mayor error relativo obtenido en cualquier punto de los datos en todas las corridas de medición.
        \item La precisión se calculará como la mayor desviación estándar entre todas las corridas de medición.
        \item La repetibilidad se calculará de la siguiente manera:
        \begin{equation*}
            \mathrm{repetibilidad} = \sqrt{\dfrac{\sum_{i=1}^n(\mathrm{error\,absoluto})^2}{n}}
        \end{equation*}
    \end{itemize}


\section{Resultados}

\begin{table}[H]
    \centering
    \caption{Mediciones tomadas a una frecuencia de \SI{1}{\hertz}}
    \vspace{0.5cm}
    \begin{tabular}{ccccccc}%
    \toprule
    \bfseries &  \multicolumn{2}{c}{\textbf{corrida 1}} & \multicolumn{2}{c}{\textbf{corrida 2}} & \multicolumn{2}{c}{\textbf{corrida 3}}\\
    \bfseries tiempo & \bfseries corriente & \bfseries voltaje & \bfseries corriente & \bfseries voltaje & \bfseries corriente & \bfseries voltaje\\
    {[\si{\second}]} & [\si{\ampere}] & [\si{\volt}] & [\si{\ampere}] & [\si{\volt}] & [\si{\ampere}] & [\si{\volt}]\\
    \midrule
    \csvreader[
        late after line=\\,
        late after last line=,
        before reading={\catcode`\#=12},
        after reading={\catcode`\#=6}]%
        {data.csv}{1=\t,2=\ci,3=\vi,4=\cii,5=\vii,6=\ciii,7=\viii}{\t &\ci & \vi &\cii & \vii &\ciii & \viii}\\
        \bottomrule
    \end{tabular}
    \label{tab:L1T1}
\end{table}
\end{enumerate}