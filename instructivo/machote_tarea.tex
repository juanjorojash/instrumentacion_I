\documentclass[12pt]{article}
\usepackage[spanish,activeacute]{babel}
\usepackage[margin=1in]{geometry}
\usepackage{amsmath,amssymb}
\usepackage{multicol}
\usepackage{siunitx}
\usepackage[american,RPvoltages]{circuitikz} %Este se usa para hacer circuitos
\usepackage{float}
\usepackage{hyperref}
\usetikzlibrary{babel}
\usepackage{tabularx}



% *** GRAPHICS RELATED PACKAGES ***
%
\usepackage{graphicx}
\graphicspath{{../pdf/}{../png/}}
\DeclareGraphicsExtensions{.pdf,.jpg,.png}
\usepackage{subfigure}



% Cambio de nombre de cuadro a Tabla
\renewcommand{\listtablename}{Índice de tablas}
\renewcommand{\tablename}{Tabla}

\begin{document}
\noindent
\begin{tabularx}{\linewidth}{Xr}
\textbf{Ingeniería Física}& \textbf{Tarea \#1} \\
\textbf{Escuela de Física}& \\
\textbf{Tecnológico de Costa Rica}& \textbf{Instrumentación I} \\
\end{tabularx}\\

\noindent\rule[2ex]{\textwidth}{2pt}
\begin{tabularx}{\linewidth}{Xr}
\textbf{Integrantes:} & \\
Nombre Apellido & Carné
\end{tabularx}

\noindent\rule[2ex]{\textwidth}{2pt}

\begin{enumerate}
\item ¿Primera pregunta de la tarea?\\
Se debe blablabla

\end{enumerate}

\end{document}
