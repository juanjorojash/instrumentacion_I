\chapter{Mediciones de nivel}
\section{\obj}
\capacidad
\begin{itemize}
\item Utilizar un sensor para medir nivel.
\item Determinar la curva de calibración para medición de volumen.
\end{itemize}

\section{\mat}
\begin{itemize}
\item 1 Arduino UNO R4 MINIMA
\item 1 sensor ultrasónico HC-SR04
\item 1 mini breadboard de 55 pines
\item 1 beaker de 250 ml
\item Jumpers macho-macho y hembra-hembra
\end{itemize}

\section{Actividad 1}
\subsection{\pro}
\begin{enumerate}
    \item Conecte el sensor HC-SR04 tal como se muestra en la Figura \ref{fig:nive1}
    \item Copie el siguiente código en el IDE de Arduino (el codigo se puede encontrar en: \href{https://github.com/juanjorojash/instrumentacion_I/blob/master/algoritmos/BMP280/BMP280_lib/BMP280_lib.ino}{GitHub})
    \lstinputlisting[language=Arduino,numbers=none]{algoritmos/nivel/nivel.ino}  
    \item Analice cada linea de código y cuando entienda su funcionamiento corra el \emph{sketch} usando el botón \emph{Upload}
    \item Haga contacto con el encapsulado del sensor de forma que su dedos calienten el sensor, observe como cambia el valor en la terminal. 
    \item Modifique el programa de forma que lo que se imprima en la terminal sea algo como esto:
\begin{verbatim}
24.8,806.1
24.7,806.3
\end{verbatim}
    \item Instale las librerías \emph{pyserial}, \emph{matplotlib}, \emph{drawnow} y \emph{datetime} en Python para ser utilizadas luego.
    \item Copie el siguiente código en su IDE de Python. (el codigo se puede encontrar en: \href{https://github.com/juanjorojash/instrumentacion_I/blob/master/algoritmos/BMP280/BMP280_lib/serial_read.py}{GitHub})
    \lstinputlisting[language=Python,numbers=none]{algoritmos/BMP280/BMP280_lib/serial_read.py}
    \item Modifique el puerto COM para que coincida con el puerto en el que esta conectado el Arduino UNO R4 MINIMA
    \item Modifique el código para que cada toma de datos se almacene en un archivo $.csv$ y para que se genere una gráfica en tiempo real.
    \item Realice tres tomas de datos de 60 segundos cada una
\end{enumerate}

\begin{figure}[H]
    \tikzset{comm/.style={muxdemux, muxdemux def={Lh=5, Rh=5, NL=2, NB=0, NR=0, w=2}}}
    \tikzset{BMP280/.style={muxdemux, muxdemux def={Lh=5, Rh=5, NL=0, NB=0, NR=6, w=2}}}
    \centering
    \begin{circuitikz} 
        \draw (0,3.5) node[comm] (m){\rotatebox{90}{\small communication}}
        (0,0) node[comm] (p){\rotatebox{90}{\small power}}
        ;
        \draw (m.blpin 1) node[above left]{\small SDA};
        \draw (m.blpin 2) node[above left]{\small SCL};
        \draw (p.blpin 1) node[above left]{\small GND};
        \draw (p.blpin 2) node[above left]{\small +3V3};
        \draw (-5,-3) node[BMP280,rotate=90] (s){\rotatebox{-90}{\small BMP280}}
        (s.brpin 1) node[above right,rotate=90]{\scriptsize SDO}
        (s.brpin 2) node[above right,rotate=90]{\scriptsize CSB}
        (s.brpin 3) node[above right,rotate=90]{\scriptsize SDA}
        (s.brpin 4) node[above right,rotate=90]{\scriptsize SCL}
        (s.brpin 5) node[above right,rotate=90]{\scriptsize GND}
        (s.brpin 6) node[above right,rotate=90]{\scriptsize VCC}
        ;
        \draw[blue]
        (p.blpin 2)
        -|
        (s.brpin 6)
        (p.blpin 2)
        -|
        (s.brpin 1)
        ;
        \draw[green]
        (p.blpin 1)
        -|
        (s.brpin 5)
        ;
        \draw[red]
        (m.blpin 2)
        -| 
        (s.brpin 4)
        ;
        \draw[brown]
        (m.blpin 1)
        -| 
        (s.brpin 3)
        ;
        \draw[dashed,blue]
        (-2.5,6) -- (1,6)node[midway, below]{UNO R4 MINIMA} -- (1,-1.5) -- (-2.5,-1.5) -- cycle;
    \end{circuitikz}
    \caption{Conexión de sensor ultrasónico de distancia}
    \label{fig:nive1}
\end{figure}

\subsection{Análisis}
\begin{enumerate}
    \item Determine la presión atmosférica local usando el promedio de las mediciones que las tres tomas de datos, incluya la precisión de la medición 
    \item Determine la altitud local usando el valor de presión calculado ¿Es este valor cercano al esperado para la altitud del campus (1407 msnm)?
\end{enumerate}

\section{Actividad 2}
\subsection{\pro}
\begin{enumerate}
    \item Usando la misma configuración anterior, realice una nueva toma de datos de 60s en la que lleve el sensor lo mas cerca del piso posible y luego lo mas alto posible lentamente y repita durante los 60 segundos de la toma
\end{enumerate}
\subsection{Análisis}
\begin{enumerate}
    \item ¿Se puede observar el cambio de altura en las mediciones?
    \item Según la hoja de datos del sensor ¿Debería detectar estos cambios?
    \item ¿El tiempo entre mediciones es realmente un segundo?
\end{enumerate}
