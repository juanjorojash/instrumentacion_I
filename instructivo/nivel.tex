\chapter{Mediciones de nivel}
\section{\obj}
\capacidad
\begin{itemize}
\item Utilizar un sensor para medir nivel.
\item Determinar la curva de calibración para medición de volumen.
\end{itemize}

\section{\mat}
\begin{itemize}
\item 1 Arduino UNO R4 MINIMA
\item 1 sensor ultrasónico HC-SR04
\item 1 mini breadboard de 55 pines
\item 1 beaker de 250 ml
\item Jumpers macho-macho y hembra-hembra
\end{itemize}

\section{Actividad 1}
\subsection{\pro}
\begin{enumerate}
    \item Conecte el sensor HC-SR04 tal como se muestra en la Figura \ref{fig:nive1}
    \item Copie el siguiente código en el IDE de Arduino (el codigo se puede encontrar en: \href{https://github.com/juanjorojash/instrumentacion_I/blob/master/algoritmos/nivel/nivel.ino}{GitHub})
    \lstinputlisting[language=Arduino,numbers=none]{algoritmos/nivel/nivel.ino}  
    \item Analice cada linea de código y cuando entienda su funcionamiento corra el \emph{sketch} usando el botón \emph{Upload}
    \item Fije el sensor en el aditamento plástico y coloquelo en la parte superior del beaker  
    \item Realice una medición de distancia para poder calibrar el valor cero.
    \item Calibre el sensor en cero de forma que pueda medir la altura desde la base el beaker
    \item Llene el beaker hasta 50ml, tome 10 datos y utilice el promedio para relacionar altura y volumen en ese punto
    \item Llene el beaker hasta 150ml, tome 10 datos y utilice el promedio para relacionar altura y volumen en ese punto
    \item Calcule la curva de calibración de volumen usando esos dos puntos
    \item Vacie el beaker
    \item Llene el beaker a 50, 100, 150 y 200ml y en cada uno de esos puntos tome el promedio de 10 datos de volumen como valor medido
\end{enumerate}

\begin{figure}[H]
    \tikzset{comm/.style={muxdemux, muxdemux def={Lh=5, Rh=5, NL=2, NB=0, NR=0, w=2}}}
    \tikzset{sensor/.style={muxdemux, muxdemux def={Lh=5, Rh=5, NL=0, NB=0, NR=4, w=2}}}
    \centering
    \begin{circuitikz} 
        \draw (0,3.5) node[comm] (m){\rotatebox{90}{\small digital}}
        (0,0) node[comm] (p){\rotatebox{90}{\small power}}
        ;
        \draw (m.blpin 1) node[above left]{\small 9};
        \draw (m.blpin 2) node[above left]{\small 10};
        \draw (p.blpin 1) node[above left]{\small GND};
        \draw (p.blpin 2) node[above left]{\small 5V};
        \draw (-5,-3) node[sensor,rotate=90] (s){\rotatebox{-90}{\small HC-SR04}}
        (s.brpin 1) node[above right,rotate=90]{\scriptsize VCC}
        (s.brpin 2) node[above right,rotate=90]{\scriptsize Trig}
        (s.brpin 3) node[above right,rotate=90]{\scriptsize Echo}
        (s.brpin 4) node[above right,rotate=90]{\scriptsize GND}
        ;
        \draw[blue]
        (p.blpin 2)
        -|
        (s.brpin 1)
        ;
        \draw[green]
        (p.blpin 1)
        -|
        (s.brpin 4)
        ;
        \draw[red]
        (m.blpin 2)
        -|
        (s.brpin 3)
        ;
        \draw[brown]
        (m.blpin 1)
        -| 
        (s.brpin 2)
        ;
        \draw[dashed,blue]
        (-2.5,6) -- (1,6)node[midway, below]{UNO R4 MINIMA} -- (1,-1.5) -- (-2.5,-1.5) -- cycle;
    \end{circuitikz}
    \caption{Conexión de sensor ultrasónico de distancia}
    \label{fig:nive1}
\end{figure}

\subsection{Análisis}
\begin{enumerate}
    \item Compare los valores medidos con los valores teóricos para 50, 100, 150 y 200ml 
    \item Determine la incertidumbre aproximada de la medición de volumen
\end{enumerate}

\section{Actividad 2}
\subsection{\pro}
\begin{enumerate}
    \item Usando la misma configuración anterior, realice una nueva toma de datos de 15 segundos en la que inicie con el beaker vacío y lo llene hasta aproximadamente 200ml
\end{enumerate}
\subsection{Análisis}
\begin{enumerate}
    \item Incluya una gráfica del proceso de llenado
    \item ¿Se observar algún tipo de distorsión en la medición? Si es así, ¿a que se debe?
\end{enumerate}
