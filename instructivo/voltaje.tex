\chapter{Medición de voltaje}
\section{\obj}
\capacidad
\begin{itemize}
\item Utilizar la fuente de tensión en voltaje continuo.
\item Utilizar un registrador para tomar datos de sensores.
\item Utilizar sensores para medir corriente y voltaje.
\item Aplicar los conocimientos relacionados a los instrumentos de medición, para el cálculo de valores tales como: exactitud, precisión, error porcentual, desviación estándar, incertidumbre, repetibilidad.
\end{itemize}

\section{\mat}
\textbf{A suministrar por la Escuela:}
\begin{itemize}
\item 1 ARDUINO UNO R4 MINIMA

\end{itemize}
\textbf{A suministrar por el estudiante:}
\begin{itemize}
\item 1 breadboard
\item 1 potenciometro de \SI{10}{\kilo\ohm}
\item 1 resistor de \SI{220}{\ohm} y \SI{0.25}{\watt}
\item Cables de interconexión (jumpers) macho-macho
\end{itemize}


\section{\pro}
\begin{enumerate}
\item Realice las conexiones del circuito tal y como se indica en la Figura \ref{fig:L1F1}.
\item Abra el programa Arduino IDE

\begin{figure}[H]
    \tikzset{dig/.style={muxdemux, muxdemux def={Lh=5, Rh=5, NL=2, NB=0, NR=0, w=2}}}
    \centering
    \begin{circuitikz} 
        \draw 
        (0,3.5) 
        node[dig] (p){\rotatebox{90}{\small POWER}}
        (0,0) 
        node[dig] (m){\rotatebox{90}{\small ANALOG IN}};
        \draw (m.blpin 1) node[above left]{\small A0};
        \draw (m.blpin 2) node[above left]{\small A1};
        \draw (p.blpin 1) node[above left]{\small 5V};
        \draw (p.blpin 2) node[above left]{\small GND};
        \draw
        (-4,6)
            to[american potentiometer,n=mypot, l_=0$\sim$\SI{10}{\kilo\ohm},i=$i$]
        (-4,4) 
        (mypot.wiper) |- (p.blpin 1)
        (-4,4)
            to[R, l_=\SI{4.7}{\kilo\ohm}, v^=$v_1$]
        (-4,1) 
            to[R, l_=\SI{2.2}{\kilo\ohm}, v^=$v_2$]
        (-4,-2)
        -- ++(1.8,0)
        |-
        (p.blpin 2)
        ;
        \draw[orange]
        (-4,3.8) -- ++ (1.5,0)
        |-
        (m.blpin 1)
        ;
        \draw[brown]
        (-4,0.8) -- ++ (1.2,0)
        |-
        (m.blpin 2)
        ;
        \draw[dashed,blue]
        (-2,6) -- (1.5,6)node[midway, below, align=center]{ARDUINO UNO\\ R4 MINIMA} -- (1.5,-1.5) -- (-2,-1.5) -- cycle;
    \end{circuitikz}
    \caption{Conexión de circuito para el Laboratorio 1}
    \label{fig:L1F1}
\end{figure}

\item Incluya el siguiente código
{\scriptsize 
    \lstinputlisting[language=Arduino, numbers=none, showstringspaces=false]{algoritmos/L1/L1.ino}
}

\item Modifique el programa de forma que se imprima \lstinline{a0value} una vez por segundo aproximadamente.
\item Modifique el programa de forma que se calcule y se imprima el voltaje $v_1$ y el voltaje $v_2$ que se muestra en la Figura \ref{fig:L1F1}.
\item Modifique el programa de forma que se incremente una variable con el valor de los segundos y se imprima.
\item Calcule $i$ usando:
\begin{equation*}
    i = \dfrac{v_2}{\SI{2.2}{\kilo\ohm}}
\end{equation*}
\item Modifique el programa de forma que se imprima el valor de los segundos, el valor del voltaje $v_1$, el valor del voltaje $v_2$, y el valor calculado de $i$, todos separados por comas y con un salto de linea después de cada medición.
\item Modifique el programa para que cuando se cumplan 20 segundos se detenga la ejecución
\item Realice una medición de 20 segundos y modifique suave pero constantemente el valor del potenciómetro, después de cada medición copie los valores obtenidos en un archivo $.csv$
\item Repita la medición tres veces siguiendo el mismo procedimiento, tome en cuenta que los resultados no tienen que ser iguales, complete la Tabla \ref{tab:L1T1}
\item Calcule indirectamente (Ley de Ohm) para cada caso el valor de $R$.
\item Para el conjunto de valores de $R$, calcule lo siguiente e incluya en una tabla:
    \begin{itemize}
        \item El valor promedio, $\overline{R}$ 
        \item La desviación estándar, $\sigma$
        \item El valor promedio de error relativo tomando \SI{4.7}{\kilo\ohm} como valor real
        \item El valor de la incertidumbre estándar, $\sigma_x = \sigma / \sqrt{n}$
        \item La exactitud
        \item La precisión
        \item La repetibilidad
    \end{itemize}
    
\item Tome en cuenta lo siguiente:
    \begin{itemize}
        \item La exactitud se calculará como el mayor error relativo obtenido en cualquier punto de los datos en todas las corridas de medición.
        \item La precisión se calculará como la mayor desviación estándar entre todas las corridas de medición.
        \item La repetibilidad se calculará de la siguiente manera:
        \begin{equation*}
            \mathrm{repetibilidad} = \sqrt{\dfrac{\sum_{i=1}^n(\mathrm{error\,absoluto})^2}{n}}
        \end{equation*}
    \end{itemize}


\section{Resultados}

\begin{table}[H]
    \centering
    \caption{Mediciones tomadas a una frecuencia de \SI{1}{\hertz}}
    \vspace{0.5cm}
    \begin{tabular}{ccccccc}%
    \toprule
    \bfseries &  \multicolumn{2}{c}{\textbf{corrida 1}} & \multicolumn{2}{c}{\textbf{corrida 2}} & \multicolumn{2}{c}{\textbf{corrida 3}}\\
    \bfseries $t$ & \bfseries $i$ & \bfseries $v_1$ & \bfseries $i$ & \bfseries $v_1$ & \bfseries $i$ & \bfseries $v_1$\\
    {[\si{\second}]} & [\si{\ampere}] & [\si{\volt}] & [\si{\ampere}] & [\si{\volt}] & [\si{\ampere}] & [\si{\volt}]\\
    \midrule
    \csvreader[
        late after line=\\,
        late after last line=,
        before reading={\catcode`\#=12},
        after reading={\catcode`\#=6}]%
        {data.csv}{1=\t,2=\ci,3=\vi,4=\cii,5=\vii,6=\ciii,7=\viii}{\t &\ci & \vi &\cii & \vii &\ciii & \viii}\\
        \bottomrule
    \end{tabular}
    \label{tab:L1T1}
\end{table}
\end{enumerate}