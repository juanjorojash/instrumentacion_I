\documentclass[12pt]{article}
\usepackage[margin=1in]{geometry} 
\usepackage{amsmath}
\usepackage{amssymb}
\usepackage{siunitx}
\usepackage{float}
\usepackage{tikz}
\usepackage{url}
\usepackage[siunitx,american,RPvoltages]{circuitikz}
\ctikzset{capacitors/scale=0.7}
\ctikzset{diodes/scale=0.7}
\usepackage{tabularx}
\newcolumntype{C}{>{\centering\arraybackslash}X}
\renewcommand\tabularxcolumn[1]{m{#1}}% for vertical centering text in X column
\usepackage{tabu}
\usepackage[spanish,es-tabla,activeacute]{babel}
\usepackage{babelbib}
\usepackage{booktabs}
\usepackage{pgfplots}
\usepackage{hyperref}
\hypersetup{colorlinks = true,
            linkcolor = black,
            urlcolor  = blue,
            citecolor = blue,
            anchorcolor = blue}
\usepgfplotslibrary{units, fillbetween} 
\pgfplotsset{compat=1.16}
\usepackage{bm}
\usetikzlibrary{arrows, arrows.meta, shapes, 3d, perspective, positioning}
\renewcommand{\sin}{\sen} %change from sin to sen
\usepackage{bohr}
\setbohr{distribution-method = quantum,insert-missing = true}
\usepackage{elements}
\usepackage{verbatim}
\usepackage{amsmath}
\usepackage{tcolorbox}
\usepackage{amssymb}
\usepackage{amsthm}
\usepackage{lastpage}
\usepackage{fancyhdr}
\usepackage{accents}
\usepackage{siunitx}
\pagestyle{fancy}
\setlength{\headheight}{42pt}

\begin{document}
\input{comunes/postamble}

\lhead{Ingeniería en Mantenimiento Industrial \\ Escuela de Ingeniría Electromecánica \\ Tecnológico de Costa Rica} 
\rhead{Electricidad I \\ Trabajo de investigación \\ Entrega: Semana 17} 
\cfoot{\thepage\ de \pageref{LastPage}}
\noindent Tomando en cuenta las siguiente asignación de los temas por grupos:
\begin{itemize}
    \item Grupo 1: Energía Solar Térmica
    \item Grupo 2: Energía Solar Fotovoltaica
    \item Grupo 3: Energía Hidroeléctrica
    \item Grupo 4: Energía Eólica
    \item Grupo 5: Energía de Biomasa
    \item Grupo 6: Energía de las olas 
    \item Grupo 7: Energía de las mareas
    \item Grupo 8: Energía por gradientes de temperatura en océanos
    \item Grupo 9: Energía por gradientes de salinidad
    \item Grupo 10: Energía geotérmica
    \item Grupo 11: Energía piezoeléctrica y triboeléctrica
\end{itemize}
y tomando en cuenta las siguientes definiciones:
\begin{itemize}
    \item Sistema: objeto o grupo de objetos cuyas propiedades queremos estudiar
    \item Entradas: variables que afectan el comportamiento del sistema
    \item Salidas: variables que son definidas por el sistema
\end{itemize}

\noindent Realice una presentación en vídeo de no mas de 15 minutos donde desarrolle las siguientes preguntas: 
\begin{itemize}
    \item Cuál es el principio de funcionamiento de esta fuente de energía?
    \item Cuales consideran que son sus entradas? 
    \item Cuales consideran son sus salidas? 
    \item Cómo se define su eficiencia? 
    \item Que tipos de tecnologías hay -si aplica- y como varia la eficiencia de estas diferentes tecnologías?
    \item Como se puede comparar su huella de carbono -en caso de no ser despreciable- con la de otras fuente de energía? Investigue sobre Análisis de Ciclo de Vida para esta fuente de energía para poder responder esto. 
    \item Que implicaciones sociales tiene el uso a gran escala de esta fuente de energía? Por ejemplo: se requiere relocalización de poblaciones enteras, implica sobre explotación de recursos de zonas en las que hay conflictos debido a estos mismo recursos, genera mas o menos empleos que otro tipo de fuente, tiene un impacto positivo o negativo en la salud, etc.
\end{itemize}
\end{document}