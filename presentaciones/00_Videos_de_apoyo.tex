%%%%%%%%%%%%%%%%%%%%%%%%%%%%%%%%%%%%%%%%%
% Tecnologico de Costa Rica/Instructivo de Laboratorio de Electricidad I
% LaTeX Template
% Version 3.1 (25/3/14)
%
% This template has been downloaded from:
% http://www.LaTeXTemplates.com
%
% Original author:
% Linux and Unix Users Group at Virginia Tech Wiki 
% (https://vtluug.org/wiki/Example_LaTeX_chem_lab_report)
%
% License:
% CC BY-NC-SA 3.0 (http://creativecommons.org/licenses/by-nc-sa/3.0/)
%
%%%%%%%%%%%%%%%%%%%%%%%%%%%%%%%%%%%%%%%%%

%----------------------------------------------------------------------------------------
%	PACKAGES AND DOCUMENT CONFIGURATIONS
%----------------------------------------------------------------------------------------

\documentclass[12pt,letterpaper]{report}
\usepackage{float}
\usepackage[spanish,activeacute]{babel} %Permite la escritura intiutiva en español
\usepackage{siunitx} % Provides the \SI{}{} and \si{} command for typesetting SI units
\sisetup{output-decimal-marker = {,}}
\usepackage{graphicx} % Required for the inclusion of images
\usepackage[siunitx,american,RPvoltages]{circuitikz} %Este se usa para hacer circuitos
%\usepackage{natbib} % Required to change bibliography style to APA
\usepackage{amsmath} % Required for some math elements 
\usepackage{array}
\usepackage{geometry}  
\usepackage{booktabs}
\usepackage{tabularx}
\newcolumntype{C}{>{\centering\arraybackslash}X}
\geometry{left=18mm,right=18mm,top=21mm,bottom=21mm,headheight=15pt} % Tamaño del área de escritura de la página
\usepackage{hyperref}
\hypersetup{
    colorlinks=true,
    linkcolor=blue,
    filecolor=magenta,      
    urlcolor=blue,
    citecolor=blue,
}
\setlength\parindent{0pt} % Removes all indentation from paragraphs

\renewcommand{\labelenumi}{\alph{enumi}.} % Make numbering in the enumerate environment by letter rather than number (e.g. section 6)
\usepackage{fancyhdr}
\pagestyle{fancy}
%\lhead{Videos de apoyo: Electricidad I}
\rhead{\begin{picture}(0,0) \put(-60,0){\includegraphics[width=20mm]{fig/logo.png}} \end{picture}}


\begin{document}
{\Huge Electricidad I: Videos de apoyo}

\vspace{1cm}

{\Large Leyes básicas}

\vspace{0.5cm}

{\Large Métodos de análisis}

\vspace{0.5cm}

{\Large Teoremas de circuitos}
\begin{itemize}
    \item \href{https://youtu.be/LQFKj_J37cA}{LTSpice: Máxima transferencia de potencia}
    \item \href{https://youtu.be/kjDn_0aK00A}{Problema 4.69: Máxima transferencia de potencia}
\end{itemize}

{\Large Capacitores e Inductores}
\begin{itemize}
    \item \href{https://youtu.be/Icl9_5sV-6E}{Clase asincrónica}
\end{itemize}

{\Large Respuestas Naturales}
\begin{itemize}
    \item \href{https://youtu.be/bWRGyjY0niY}{Clase asincrónica}
    \item \href{https://youtu.be/37KDKfhaRO8}{Problema 7.10: Respuesta Natural RC}
\end{itemize}

{\Large Funciones de singularidad}
\begin{itemize}
    \item \href{https://youtu.be/-DqJL51fJ6w}{Clase asincrónica}
\end{itemize}

\end{document}