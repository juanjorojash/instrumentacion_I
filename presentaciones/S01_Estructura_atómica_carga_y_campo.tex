\documentclass[aspectratio=169]{beamer}
\usetheme{Bruno}
\usepackage{amsmath}
\usepackage{amssymb}
\usepackage{siunitx}
\usepackage{float}
\usepackage{tikz}
\usepackage{url}
\usepackage[siunitx,american,RPvoltages]{circuitikz}
\ctikzset{capacitors/scale=0.7}
\ctikzset{diodes/scale=0.7}
\usepackage{tabularx}
\newcolumntype{C}{>{\centering\arraybackslash}X}
\renewcommand\tabularxcolumn[1]{m{#1}}% for vertical centering text in X column
\usepackage{tabu}
\usepackage[spanish,es-tabla,activeacute]{babel}
\usepackage{babelbib}
\usepackage{booktabs}
\usepackage{pgfplots}
\usepackage{hyperref}
\hypersetup{colorlinks = true,
            linkcolor = black,
            urlcolor  = blue,
            citecolor = blue,
            anchorcolor = blue}
\usepgfplotslibrary{units, fillbetween} 
\pgfplotsset{compat=1.16}
\usepackage{bm}
\usetikzlibrary{arrows, arrows.meta, shapes, 3d, perspective, positioning}
\renewcommand{\sin}{\sen} %change from sin to sen
\usepackage{bohr}
\setbohr{distribution-method = quantum,insert-missing = true}
\usepackage{elements}
\usepackage{verbatim}
\title{Electricidad I: \\ \emph{Estructura atómica,}\\ \emph{carga eléctrica y}\\ \emph{campo eléctrico}}
\author{
    Juan J. Rojas
}
\institute{Instituto Tecnológico de Costa Rica}
\date{\today}
\background{fig/background.jpg}
\begin{document}
\input{comunes/postamble}
\maketitle

\begin{frame}{Estructura atómica}
    \begin{columns}[onlytextwidth]
    \begin{column}{0.6\textwidth}
        \textbf{\elementname{H}}\\[8pt]
        Número atómico: \atomicnumber{H}\\[8pt]
        Configuración electrónica:\\[8pt] 
        \elconf{H}\\[8pt]  
        \end{column}
    \begin{column}{0.4\textwidth}
    \begin{center}
        Modelo de Bohr:\\[8pt]
        \bohr{}{H}
    \end{center}
\end{column}
\end{columns}
\end{frame}

\begin{frame}{Estructura atómica}
    \begin{columns}[onlytextwidth]
    \begin{column}{0.6\textwidth}
        \textbf{\elementname{Cu}}\\[8pt]
        Número atómico: \atomicnumber{Cu}\\[8pt]
        Configuración electrónica:\\[8pt] 
        \elconf{Cu}\\[8pt]  
        \end{column}
    \begin{column}{0.4\textwidth}
    \begin{center}
        Modelo de Bohr:\\[8pt]
        \bohr{}{Cu}
    \end{center}
\end{column}
\end{columns}
\end{frame}

\begin{frame}{Estructura atómica}
    \begin{columns}[onlytextwidth]
    \begin{column}{0.6\textwidth}
        \textbf{\elementname{Si}}\\[8pt]
        Número atómico: \atomicnumber{Si}\\[8pt]
        Configuración electrónica:\\[8pt] 
        \elconf{Si}\\[8pt]  
        \end{column}
    \begin{column}{0.4\textwidth}
    \begin{center}
        Modelo de Bohr:\\[8pt]
        \bohr{}{Si}
    \end{center}
\end{column}
\end{columns}
\end{frame}

\begin{frame}{Materiales aislantes, semiconductores y conductores}
    \begin{center}
        \begin{tikzpicture}[scale=0.9, transform shape]
            \begin{scope}
                \fill[green!30] (0,3) rectangle +(4,1) node[black,midway]{Banda de conducción}; 
                \fill[red!30] (0,0) rectangle +(4,1) node[black,midway]{Banda de valencia}; 
                \draw[dashed,thick,red] (0,2)node[above]{\scriptsize$E_f$} --++ (4,0);
                \draw[latex-latex] (4.2,1) --++ (0,2) node[midway,right]{$E_g$};
                \fill[blue!30] (0,-1.5) rectangle +(4,1) node[black,midway]{Bandas llenas};
                \fill[white] (0,-2.5) rectangle +(4,1) node[black,midway]{\textbf{Aislante}};
            \end{scope}
            \begin{scope}[xshift=5.2cm]
                \fill[green!30] (0,2) rectangle +(4,1) node[black,midway]{Banda de conducción}; 
                \fill[red!30] (0,0) rectangle +(4,1) node[black,midway]{Banda de valencia};
                \draw[dashed,thick,red] (0,1.5)node[above]{\scriptsize$E_f$} --++ (4,0);
                \draw[latex-latex] (4.2,1) --++ (0,1) node[midway,right]{$E_g$};
                \fill[blue!30] (0,-1.5) rectangle +(4,1) node[black,midway]{Bandas llenas};
                \fill[white] (0,-2.5) rectangle +(4,1) node[black,midway]{\textbf{Semiconductor}};
            \end{scope}
            \begin{scope}[xshift=10.4cm]
                \shade[top color=green!30,bottom color=red!30] (0,0) rectangle +(4,1.8);
                \draw (2,1.35) node[black]{Banda de conducción};
                \draw (2,0.45) node[black]{Banda de valencia};
                %\fill[red!30] (0,0) rectangle +(4,1) node[black,midway]{Banda de valencia};
                \fill[blue!30] (0,-1.5) rectangle +(4,1) node[black,midway]{Bandas llenas};
                \fill[white] (0,-2.5) rectangle +(4,1) node[black,midway]{\textbf{Conductor}};
            \end{scope}
        \end{tikzpicture}
    \end{center}
\end{frame}

\begin{frame}{Densidad de estados}
    \begin{center}
        \includegraphics[width=0.7\linewidth]{fig/band_structure.png}
    \end{center}
\end{frame}

\begin{frame}{Carga eléctrica} 
    \begin{columns}[onlytextwidth]
    \begin{column}{0.6\textwidth}
        La carga eléctrica es una propiedad de las particulas atómicas de las que se compone la materia, se mide en coulombs (\si{\coulomb})\\[4pt]
        La carga elemental es la que presenta un solo electrón y equivale a \SI{-1.602e-19}{C}
    \end{column}
    \begin{column}{0.4\textwidth}
        \begin{center}
            \begin{tikzpicture}[scale=1]
                \clip (-2.5,-3) rectangle (2.5,3);
                \coordinate (e) at (-2,0);
                \coordinate (p) at (2,0);
                \shade[ball color=gray!10!] (e) circle (.35) node{\Large$-$};
                \shade[ball color=gray!10!] (p) circle (3) node{\Large$+$};
            \end{tikzpicture}
        \end{center}
    \end{column}
    \end{columns}
\end{frame}

\begin{frame}{Fuerza eléctrica} 
    \begin{columns}[onlytextwidth]
    \begin{column}{0.6\textwidth}
        La fuerza eléctrica es la fuerza de atracción o repulsión entre dos cuerpos o partículas cargadas 
        \begin{equation*}
            F = k\dfrac{q_1q_2}{r^2}
        \end{equation*}
        donde $F$ es la fuerza eléctrica en newtons (\si{\newton}), $k$ es la constante de Coulomb ($\SI{9e9}{\newton \meter\squared / \coulomb \squared}$), $q_1$ y $q_2$ son dos cargas en coulombs (\si{\coulomb}), y $r$ es la distancia que separa a las cargas en metros (\si{\meter})   
    \end{column}
    \begin{column}{0.4\textwidth}
        \begin{center}
            \begin{tikzpicture}[scale=1]
                \shade[ball color=gray!10!] (0,0) coordinate(q1) circle (.4) node{$+q_1$};
                \shade[ball color=gray!10!] (2.5,0) coordinate(q2) circle (.4) node{$+q_2$};
                \draw[-latex,thick] (q1)++(-0.4,0)--+(-0.5,0);
                \draw[-latex,thick] (q2)++(0.4,0)--++(0.5,0);
                \draw[latex-,thick] (q1)++(0,1)--++(1,0);
                \draw[latex-,thick] (q2)++(0,1)--++(-1,0);
                \draw (q1)++(0,1)++(1.25,0)node{$r$};
            \end{tikzpicture}\\[4pt]
            \begin{tikzpicture}[scale=1]
                \shade[ball color=gray!10!] (0,0) coordinate(q1) circle (.4) node{$+q_1$};
                \shade[ball color=gray!10!] (2.5,0) coordinate(q2) circle (.4) node{$-q_2$};
                \draw[-latex,thick] (q1)++(0.4,0)--+(0.5,0);
                \draw[-latex,thick] (q2)++(-0.4,0)--++(-0.5,0);
            \end{tikzpicture}\\[4pt]
            \begin{tikzpicture}[scale=1]
                \shade[ball color=gray!10!] (0,0) coordinate(q1) circle (.4) node{$-q_1$};
                \shade[ball color=gray!10!] (2.5,0) coordinate(q2) circle (.4) node{$-q_2$};
                \draw[-latex,thick] (q1)++(-0.4,0)--+(-0.5,0);
                \draw[-latex,thick] (q2)++(0.4,0)--++(0.5,0);
            \end{tikzpicture}
        \end{center}
    \end{column}
    \end{columns}
\end{frame}

\begin{frame}{Campo eléctrico} 
    \begin{columns}[onlytextwidth]
    \begin{column}{0.55\textwidth}
        El campo eléctrico es la región del espacio en el que un cuerpo o partícula cargada tiene influencia, dentro de esta regíon se ejercería una fuerza sobre otros cuerpos o partículas cargadas.  
        \begin{equation*}
            E = k\dfrac{q}{r^2}
        \end{equation*}
        donde $E$ es el campo eléctrico en newtons por coulomb (\si{\newton/\coulomb}) o voltios por metro (\si{\volt/\meter}), $k$ es la constante de Coulomb ($\SI{9e9}{\newton \meter\squared / \coulomb \squared}$), $q$ es la carga en coulombs (\si{\coulomb}), y $r$ es la distancia entre el punto en el que se mide el campo y la carga en (\si{\meter})   
    \end{column}
    \begin{column}{0.45\textwidth}
        \begin{center}
            \includegraphics[scale=0.55]{fig/campo_unados.png}
            \includegraphics[scale=0.55]{fig/campo_iguales.png}
        \end{center}
    \end{column}
    \end{columns}
\end{frame}

% \begin{frame}{Referencias}

% \bibliographystyle{ieeetr}

% \bibliography{referencias}

% \end{frame}

\end{document}