\documentclass[aspectratio=169]{beamer}
\usetheme{Bruno}
\usepackage{amsmath}
\usepackage{amssymb}
\usepackage{siunitx}
\usepackage{float}
\usepackage{tikz}
\def\checkmark{\tikz\fill[scale=0.4](0,.35) -- (.25,0) -- (1,.7) -- (.25,.15) -- cycle;} 
\usepackage{url}
\usepackage[siunitx,american,RPvoltages]{circuitikz}
\ctikzset{capacitors/scale=0.7}
\ctikzset{diodes/scale=0.7}
\usepackage{tabularx}
\newcolumntype{C}{>{\centering\arraybackslash}X}
\renewcommand\tabularxcolumn[1]{m{#1}}% for vertical centering text in X column
\usepackage{tabu}
\usepackage[spanish,es-tabla,activeacute]{babel}
\usepackage{babelbib}
\usepackage{booktabs}
\usepackage{pgfplots}
\usepackage{hyperref}
\hypersetup{colorlinks = true,
            linkcolor = black,
            urlcolor  = blue,
            citecolor = blue,
            anchorcolor = blue}
\usepgfplotslibrary{units, fillbetween} 
\pgfplotsset{compat=1.16}
\usepackage{bm}
\usetikzlibrary{arrows, arrows.meta, shapes, 3d, perspective, positioning,mindmap,trees,backgrounds}
\renewcommand{\sin}{\sen} %change from sin to sen
\usepackage{bohr}
\setbohr{distribution-method = quantum,insert-missing = true}
\usepackage{elements}
\usepackage{verbatim}
\usepackage[edges]{forest}
\usepackage{etoolbox}
\usepackage{schemata}
\usepackage{appendix}
\usepackage{listings}

\definecolor{color_mate}{RGB}{255,255,128}
\definecolor{color_plas}{RGB}{255,128,255}
\definecolor{color_text}{RGB}{128,255,255}
\definecolor{color_petr}{RGB}{255,192,192}
\definecolor{color_made}{RGB}{192,255,192}
\definecolor{color_meta}{RGB}{192,192,255}
\newcommand\diagram[2]{\schema{\schemabox{#1}}{\schemabox{#2}}}

\definecolor{codegreen}{rgb}{0,0.6,0}
\definecolor{codegray}{rgb}{0.5,0.5,0.5}
\definecolor{codepurple}{rgb}{0.58,0,0.82}
\definecolor{backcolour}{rgb}{0.95,0.95,0.92}

\lstdefinestyle{mystyle}{
    backgroundcolor=\color{backcolour},   
    commentstyle=\color{codegreen},
    keywordstyle=\color{magenta},
    numberstyle=\tiny\color{codegray},
    stringstyle=\color{codepurple},
    basicstyle=\ttfamily\footnotesize,
    breakatwhitespace=false,         
    breaklines=true,                 
    captionpos=b,                    
    keepspaces=true,                 
    numbers=left,                    
    numbersep=5pt,                  
    showspaces=false,                
    showstringspaces=false,
    showtabs=false,                  
    tabsize=2
}

\lstset{style=mystyle}
\title{Electricidad I: \\ \emph{Conceptos básicos}}
\author{
    Juan J. Rojas
}
\institute{Instituto Tecnológico de Costa Rica}
\date{\today}
\background{fig/background.jpg}
\begin{document}
\sisetup{unit-math-rm=\mathrm,math-rm=\mathrm} % change sinitx font
\sisetup{output-decimal-marker = {,}}
\maketitle

\begin{frame}{Corriente}
    \begin{columns}[onlytextwidth]
        \begin{column}{0.6\textwidth}
            En un material conductor los electrones libres se mueven en todas direcciones, el efecto neto es cero\\[8pt]
            No existe una fuerza externa que los obligue a moverse en una dirección particular y por lo tanto no hay corriente
        \end{column}
        \begin{column}{0.4\textwidth}
            \begin{center}
                \begin{tikzpicture}
                    \def \yaw {30}
                    \def \pitch {50};
                    \begin{scope}[3d view={\yaw}{\pitch}]
                        \draw[canvas is zy plane at x=-2](0,0) ++ (\pitch:1cm) arc (\pitch:\pitch-180:1cm);
                        \draw[canvas is zy plane at x=-2,dashed](0,0) ++ (\pitch:1cm) arc (\pitch:\pitch+180:1cm);
                        %\draw[canvas is zy plane at x=-2](0,0) circle (1cm);
                        \path[canvas is zy plane at x=-2](0,0) ++ (\pitch:1cm) coordinate (tl);
                        \path[canvas is zy plane at x=-2](0,0) ++ (\pitch:-1cm) coordinate (bl);
                        \draw[canvas is zy plane at x=2](0,0) circle (1cm);
                        \fill[canvas is zy plane at x=2,color = gray!10, opacity = 0.8](0,0) circle (1cm);
                        \path[canvas is zy plane at x=2](0,0) ++ (\pitch:1cm) coordinate (tr);
                        \path[canvas is zy plane at x=2](0,0) ++ (\pitch:-1cm) coordinate (br);
                        \draw (tl) -- (tr);
                        \draw (bl) -- (br);
                        \coordinate (e1) at (1,0.5,0.5);
                        \coordinate (e2) at (1,-0.5,-0.5);
                        \coordinate (e3) at (-1,0.5,-0.5);
                        \coordinate (e4) at (-1,-0.5,0.5);
                        \draw[canvas is zy plane at x=0, dashed](0,0) circle (1cm) ;
                        \fill[canvas is zy plane at x=0, dashed, color = gray!40, opacity = 0.8](0,0) circle (1cm);
                        \draw[-latex] (e1)  --++ (-1.5,0,0);
                        \shade[ball color = gray!40, opacity = 0.8] (e1) circle (4pt) node{-};
                        \draw[-latex] (e2)  --++ (-1.5,0,0);
                        \shade[ball color = gray!40, opacity = 0.8] (e2) circle (4pt) node{-};
                        \draw[-latex] (e3)  --++ (1.5,0,0);
                        \shade[ball color = gray!40, opacity = 0.8] (e3) circle (4pt) node{-};
                        \draw[-latex] (e4)  --++ (1.5,0,0);
                        \shade[ball color = gray!40, opacity = 0.8] (e4) circle (4pt) node{-};
                    \end{scope}
                \end{tikzpicture}
            \end{center}
        \end{column}
    \end{columns}
\end{frame}

\begin{frame}{Corriente}
    \begin{columns}[onlytextwidth]
        \begin{column}{0.6\textwidth}
            La corriente eléctrica es la taza de flujo de carga eléctrica a través de un punto o una región.
            \begin{equation*}
                i = \frac{dq}{dt} 
            \end{equation*}
            donde $i$ es la corriente en amperes (\si{\ampere}), $q$ es la carga en coulombs (\si{\coulomb}), y $t$ es el tiempo en segundos (\si{\second})
            \begin{equation*}
                \SI{1}{\ampere} = \SI{1}{\coulomb \second^{-1}} 
            \end{equation*}
            Por convención, la dirección de la corriente es opuesta a la dirección en la que fluyen los electrones
        \end{column}
        \begin{column}{0.4\textwidth}
            \begin{center}
                \begin{circuitikz}
                    \def \yaw {30}
                    \def \pitch {50};
                    \begin{scope}[3d view={\yaw}{\pitch}]
                        \draw[canvas is zy plane at x=-2](0,0) ++ (\pitch:1cm) arc (\pitch:\pitch-180:1cm);
                        \draw[canvas is zy plane at x=-2,dashed](0,0) ++ (\pitch:1cm) arc (\pitch:\pitch+180:1cm);
                        %\draw[canvas is zy plane at x=-2](0,0) circle (1cm);
                        \path[canvas is zy plane at x=-2](0,0) ++ (\pitch:1cm) coordinate (tl);
                        \path[canvas is zy plane at x=-2](0,0) ++ (\pitch:-1cm) coordinate (bl);
                        \draw[canvas is zy plane at x=2](0,0) circle (1cm);
                        \fill[canvas is zy plane at x=2,color = gray!10, opacity = 0.8](0,0) circle (1cm);
                        \path[canvas is zy plane at x=2](0,0) ++ (\pitch:1cm) coordinate (tr);
                        \path[canvas is zy plane at x=2](0,0) ++ (\pitch:-1cm) coordinate (br);
                        \draw (tl) -- (tr);
                        \draw (bl) -- (br);
                        \coordinate (e1) at (-1,0.5,0.5);
                        \coordinate (e2) at (-1,-0.5,-0.5);
                        \coordinate (e3) at (-1,0.5,-0.5);
                        \coordinate (e4) at (-1,-0.5,0.5);
                        \draw[canvas is zy plane at x=0, dashed](0,0) circle (1cm) ;
                        \fill[canvas is zy plane at x=0, dashed, color = gray!40, opacity = 0.8](0,0) circle (1cm);
                        \draw[-latex] (e1)  --++ (1.5,0,0);
                        \shade[ball color = gray!40, opacity = 0.8] (e1) circle (4pt) node{-};
                        \draw[-latex] (e2)  --++ (1.5,0,0);
                        \shade[ball color = gray!40, opacity = 0.8] (e2) circle (4pt) node{-};
                        \draw[-latex] (e3)  --++ (1.5,0,0);
                        \shade[ball color = gray!40, opacity = 0.8] (e3) circle (4pt) node{-};
                        \draw[-latex] (e4)  --++ (1.5,0,0);
                        \shade[ball color = gray!40, opacity = 0.8] (e4) circle (4pt) node{-};
                        \draw[canvas is xz plane at y=0,transform shape](0,0) ++ (-2,0) --++ (-1,0) to [V,i={\Large$i$},v=$V$] ++ (0,3) --++ (6,0) --++ (0,-3) --++ (-1,0)
                        ; 
                    \end{scope}
                \end{circuitikz}
            \end{center}
        \end{column}
    \end{columns}
\end{frame}

\begin{frame}{Corriente directa y alterna}
    \begin{columns}[onlytextwidth]
    \begin{column}{0.6\textwidth}
        \begin{itemize}
            \item La corriente directa permanece constante en el tiempo\\[8pt]
            \item La corriente alterna invierte periódicamente su dirección y cambia su magnitud continuamente a lo largo del tiempo
        \end{itemize}
        \end{column}
    \begin{column}{0.4\textwidth}
    \begin{center}
    \vskip0pt plus 1filll
        \begin{tikzpicture}
            \begin{axis}[
                width= \linewidth,
                height= 0.7\linewidth,
                domain=0:4*pi,
                samples=100,
                ymin=-1.2, ymax=1.2,
                xmin=0, xmax=4*pi+0.5,
                axis x line=center,
                axis y line=left,
                ytick=\empty,
                %yticklabels={$-V_p$,$0$,$V_p$},
                xtick=\empty,
                %xticklabels={$\pi$,$2\pi$,$3\pi$,$4\pi$},
                xlabel=$t$, % Set the labels
                ylabel=$i(t)$,
                x unit=, 
                y unit=,
                axis lines=center,
                x label style={anchor=west},
                y label style={anchor=south},
                ]
                \addplot[color=blue,mark=none,thick] {1}
            	;
            \end{axis}
        \end{tikzpicture}
        \begin{tikzpicture}
            \begin{axis}[
                width= \linewidth,
                height= 0.7\linewidth,
                domain=0:4*pi,
                samples=100,
                ymin=-1.2, ymax=1.2,
                xmin=0, xmax=4*pi+0.5,
                axis x line=center,
                axis y line=left,
                ytick=\empty,
                %yticklabels={$-V_p$,$0$,$V_p$},
                xtick=\empty,
                %xticklabels={$\pi$,$2\pi$,$3\pi$,$4\pi$},
                xlabel=$t$, % Set the labels
                ylabel=$i(t)$,
                x unit=, 
                y unit=,
                axis lines=center,
                x label style={anchor=west},
                y label style={anchor=south},
                ]
                \addplot[color=blue,mark=none,thick] {sin(deg(x))}
            	;
            \end{axis}
        \end{tikzpicture}
    \end{center}
\end{column}
\end{columns}
\end{frame}


\begin{frame}{Voltaje}
    \begin{columns}[onlytextwidth]
        \begin{column}{0.6\textwidth}
            Voltaje es la diferencia de potencial eléctrico entre dos puntos $a$ y $b$. Físicamente, es la energía que se requiere para mover una carga unitaria desde $a$ hasta $b$
            \begin{equation*}
                v = \dfrac{dw}{dq}
            \end{equation*}            
            donde $v$ es el voltaje en voltios (\si{\volt}), $w$ es la energía en joules (\si{\joule}) y $q$ es la carga en coulombs (\si{\coulomb})\\[4pt]
            \begin{equation*}
                \SI{1}{\volt} = \SI{1}{\joule\coulomb^{-1}}
            \end{equation*} 
        \end{column}
        \begin{column}{0.4\textwidth}
            \begin{center}
                    \begin{circuitikz} [scale=1]\draw
                    (1,0)
                        to[short,o-]
                    (0,0)	
                        to[generic]
                    (0,3)
                        to[short,-o]
                    (1,3)
                        to[open,v^=$v$]
                    (1,0)
                    ;
                    \end{circuitikz}
            \end{center}
        \end{column}
    \end{columns}
\end{frame}

\begin{frame}{Representaciones del voltaje}
    \begin{center}
            \begin{circuitikz} [scale=1]
                \begin{scope}
                    \draw
                    (1,0)
                        to[short,o-]
                    (0,0)	
                        to[generic]
                    (0,3)
                        to[short,-o]
                    (1,3)
                        to[open,v^=$v_x$]
                    (1,0)
                    ;
                    \draw (0.5,-1.5)node[above] {$v_x=v_+-v_-$};
                \end{scope}
                \begin{scope}[xshift=4cm]
                    \draw
                    (1,0)
                        to[short,o-]
                    (0,0)	
                        to[generic]
                    (0,3)
                        to[short,-o]
                    (1,3)node[below right]{$a$}
                    (1,0)node[above right]{$b$}
                    ;
                    \draw (0.5,-1.5)node[above] {$v_{ab}=v_a-v_b$};
                \end{scope}
                \begin{scope}[xshift=8cm]
                    \draw
                    (0,0)node[ground]{}
                        to[generic,v>=$v_x$]
                    (0,3)
                        to[short,-o]
                    (1,3)node[below right]{$a$}
                    ;
                    \draw (0.5,-1.5)node[above] {$v_x=v_a-0$};
                \end{scope}
            \end{circuitikz}
    \end{center}
\end{frame}

\begin{frame}{Potencia} 
    \begin{columns}[onlytextwidth]
    \begin{column}{0.6\textwidth}
        La potencia es la taza de transferencia de energía eléctrica en un circuito. Se obtiene al multiplicar la corriente por el voltaje
        \begin{equation*}
            p = \dfrac{dw}{dt} = \dfrac{dw}{dq} \cdot \dfrac{dq}{dt} =  v\cdot i
        \end{equation*}
        donde $p$ es la potencia en watts (\si{\watt})
        \begin{equation*}
            \SI{1}{\watt} = \SI{1}{\joule\second^{-1}}
        \end{equation*}
    \end{column}
    \begin{column}{0.4\textwidth}
        \begin{center}
            \begin{circuitikz} [scale=1]\draw
                (1,0)
                    to[short,o-]
                (0,0)	
                    to[generic]
                (0,3)
                    to[short,-o,i<=$i$]
                (1,3)
                    to[open,v^=$v$]
                (1,0)
                ;
            \draw (0.5,-1)node[above] {$p=v\cdot i$};
            \end{circuitikz}
        \end{center}
    \end{column}
    \end{columns}
\end{frame}

\begin{frame}{Convención pasiva de potencia}
    \begin{tabularx}{\linewidth}{X X}
        \centering
        \begin{circuitikz} [scale=1]\draw
            (1,0)
                to[short,o-]
            (0,0)	
                to[generic]
            (0,3)
                to[short,-o,i<=$i$]
            (1,3)
                to[open,v^=$v$]
            (1,0)
            ;
        \draw (0.5,4)node[above] {Pasivos};
        \draw (0.5,-1)node[above] {$p=v\cdot i$};
        \end{circuitikz}
        &
        \centering
        \begin{circuitikz} [scale=1]\draw
            (1,0)
                to[short,o-]
            (0,0)	
                to[generic]
            (0,3)
                to[short,-o,i=$i$]
            (1,3)
                to[open,v^=$v$]
            (1,0)
            ;
        \draw (0.5,4)node[above] {Activos};
        \draw (0.5,-1)node[above] {$p=-v\cdot i$};
        \end{circuitikz}
    \end{tabularx}
\end{frame}

\begin{frame}{Energía} 
    \begin{columns}[onlytextwidth]
    \begin{column}{0.6\textwidth}
        La energía es la capacidad para realizar trabajo
        \begin{equation*}
            w = \int_{t_0}^{t}p\,dt = \int_{t_0}^{t}v\cdot i\,dt
        \end{equation*}
        donde $w$ es la energía en joules (\si{\joule}), $p$ es la potencia en watts (\si{\watt}) y $t$ es el tiempo de segundos (\si{\second})
        \begin{equation*}
            \SI{3600}{\joule} = \SI{1}{\watt\hour}
        \end{equation*}
    \end{column}
    \begin{column}{0.4\textwidth}
        \begin{center}
            \begin{circuitikz} [scale=1]\draw
                (1,0)
                    to[short,o-]
                (0,0)	
                    to[generic]
                (0,3)
                    to[short,-o,i<=$i$]
                (1,3)
                    to[open,v^=$v$]
                (1,0)
                ;
            \draw (0.5,-1)node[above] {$p=v\cdot i$};
            \end{circuitikz}
        \end{center}
    \end{column}
    \end{columns}
\end{frame}

\begin{frame}{Resistencia y resistividad} 
    \begin{columns}[onlytextwidth]
    \begin{column}{0.6\textwidth}
        La resistencia es una medida de la oposición al flujo de la corriente que exhibe un material 
        \begin{equation*}
            R = \rho \dfrac{l}{A}
        \end{equation*}
        donde $R$ es la resistencia en ohms (\si{\ohm}), $\rho$ es la resistividad en ohm-metro (\si{\ohm\meter}), $l$ es la longitud en metros (\si{\meter}) y $A$ es el area en metros cuadrados (\si{\meter\squared})
    \end{column}
    \begin{column}{0.4\textwidth}
        \begin{center}
            \begin{circuitikz}
                \def \yaw {30}
                \def \pitch {50};
                \def \radio {0.7cm};
                \begin{scope}[3d view={\yaw}{\pitch}]
                    \draw[canvas is zy plane at x=-2](0,0) ++ (\pitch:\radio) arc (\pitch:\pitch-180:\radio);
                    \draw[canvas is zy plane at x=-2,dashed](0,0) ++ (\pitch:\radio) arc (\pitch:\pitch+180:\radio);
                    \path[canvas is zy plane at x=-2](0,0) ++ (\pitch:\radio) coordinate (tl);
                    \path[canvas is zy plane at x=-2](0,0) ++ (\pitch:-\radio) coordinate (bl);
                    \draw[canvas is zy plane at x=2](0,0) circle (\radio);
                    \fill[canvas is zy plane at x=2,color = gray!10, opacity = 0.8](0,0) circle (\radio);
                    \path[canvas is zy plane at x=2](0,0) ++ (\pitch:\radio) coordinate (tr);
                    \path[canvas is zy plane at x=2](0,0) ++ (\pitch:-\radio) coordinate (br);
                    \draw (tl) -- (tr);
                    \draw (bl) -- (br);
                    %\draw[canvas is zy plane at x=0, dashed](0,0) circle (\radio) ;
                    %\fill[canvas is zy plane at x=0, dashed, color = gray!40, opacity = 0.8](0,0) circle (\radio);
                    \draw[canvas is xz plane at y=0,dashed](-2,0) --++ (0,2)(2,0) --++ (0,2);
                    \draw[canvas is xz plane at y=0,latex-latex] (-2,2) --++ (4,0)node[midway, above]{$l$}; 
                    \draw[canvas is yz plane at x=2, latex-]plot [smooth] coordinates {(0.2,0.2) (0.7,0.9) (1.5,0.5)};
                    \draw[canvas is yz plane at x=2] (1.8,0.5)node{$A$};
                    \draw[canvas is yz plane at x=1.5, latex-]plot [smooth] coordinates {(-0.2,-0.2) (-0.7,-0.9) (-1.5,-0.5)};
                    \draw[canvas is yz plane at x=2] (-2.5,0.5)node[align=right,left]{$\rho$ \\ {\scriptsize (del material)}};
                \end{scope}
            \end{circuitikz}
        \end{center}
    \end{column}
    \end{columns}
\end{frame}

\begin{frame}{Ley de Ohm}
    \begin{columns}[onlytextwidth]
    \begin{column}{0.6\textwidth}
        Existe una relación de proporcionalidad directa entre el voltaje y la corriente en un resistor ideal.
        
        La constante de proporcionalidad es la resistencia y la relación matemática se conoce como la ley de Ohm
        \begin{equation*}
            v = i\cdot R
        \end{equation*}
        
        \begin{equation*}
            \SI{1}{\ohm} = \SI{1}{\volt\ampere^{-1}}
        \end{equation*}
    \end{column}
    \begin{column}{0.4\textwidth}
        \begin{center}
            \begin{circuitikz} [scale=1]\draw
                (1,0)
                    to[short,o-]
                (0,0)	
                    to[R,l=$R$]
                (0,3)
                    to[short,-o,i<=$i$]
                (1,3)
                    to[open,v=$v$]
                (1,0)
                ;
            \draw (0.5,-1)node[above] {$v=i\cdot R$};
            \end{circuitikz}
        \end{center}
    \end{column}
    \end{columns}
\end{frame}

\begin{frame}{Corto circuito}
    \begin{columns}[onlytextwidth]
    \begin{column}{0.6\textwidth}
        Un cortocircuito es un elemento que tiene una resistencia cero, por lo tanto el voltaje entre sus terminales es también igual a cero
    \end{column}
    \begin{column}{0.4\textwidth}
        \begin{center}
            \begin{circuitikz} [scale=1]\draw
                (1,0)
                    to[short,o-]
                (0,0)	
                    to[short,-o]
                (0,0.5)
                    to[short]
                (0,2.5)
                    to[short,o-]
                (0,3)
                    to[short,-o,i<=$i$]
                (1,3)
                    to[open,v^=$v$]
                (1,0)
                ;
            \draw (0.5,-1)node[above] {$v=0$};
            \end{circuitikz}
        \end{center}
    \end{column}
    \end{columns}
\end{frame}

\begin{frame}{Circuito abierto}
    \begin{columns}[onlytextwidth]
    \begin{column}{0.6\textwidth}
        Un circuito abierto es un elemento que tiene una resistencia infinita, por lo tanto la corriente no fluye a traves de él
    \end{column}
    \begin{column}{0.4\textwidth}
        \begin{center}
            \begin{circuitikz} [scale=1]\draw
                (1,0)
                    to[short,o-]
                (0,0)
                    to[short,-o]
                (0,0.5)
                    to[open]
                (0,2.5)
                    to[short,o-]
                (0,3)
                    to[short,-o,i<=$i$]
                (1,3)
                    to[open,v^=$v$]
                (1,0)
                ;
            \draw (0.5,-1)node[above] {$i=0$};
            \end{circuitikz}
        \end{center}
    \end{column}
    \end{columns}
\end{frame}



% \begin{frame}{Referencias}

% \bibliographystyle{ieeetr}

% \bibliography{referencias}

% \end{frame}

\end{document}