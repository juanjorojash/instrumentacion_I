\documentclass[aspectratio=169]{beamer}
\usetheme{Bruno}
\usepackage{amsmath}
\usepackage{amssymb}
\usepackage{siunitx}
\usepackage{float}
\usepackage{tikz}
\usepackage{url}
\usepackage[siunitx,american,RPvoltages]{circuitikz}
\ctikzset{capacitors/scale=0.7}
\ctikzset{diodes/scale=0.7}
\usepackage{tabularx}
\newcolumntype{C}{>{\centering\arraybackslash}X}
\renewcommand\tabularxcolumn[1]{m{#1}}% for vertical centering text in X column
\usepackage{tabu}
\usepackage[spanish,es-tabla,activeacute]{babel}
\usepackage{babelbib}
\usepackage{booktabs}
\usepackage{pgfplots}
\usepackage{hyperref}
\hypersetup{colorlinks = true,
            linkcolor = black,
            urlcolor  = blue,
            citecolor = blue,
            anchorcolor = blue}
\usepgfplotslibrary{units, fillbetween} 
\pgfplotsset{compat=1.16}
\usepackage{bm}
\usetikzlibrary{arrows, arrows.meta, shapes, 3d, perspective, positioning}
\renewcommand{\sin}{\sen} %change from sin to sen
\usepackage{bohr}
\setbohr{distribution-method = quantum,insert-missing = true}
\usepackage{elements}
\usepackage{verbatim}
\title{Electricidad I: \\ \emph{Nodos y supernodos}}
\author{
    Juan J. Rojas
}
\institute{Instituto Tecnológico de Costa Rica}
\date{\today}
\background{fig/background.jpg}
\begin{document}
\input{comunes/postamble}
\maketitle

\begin{frame}{Análisis de nodos}
    \vspace{0.1cm}
    \only<2>{1. Seleccionar un nodo como referencia. Asignar voltajes a los demás nodos}
    \only<3-7>{2. Aplique la LCK a cada uno de los nodos de  no referencia}
    \only<8-9>{3. Reordene las ecuaciones}
    \only<10->{4. Resuelva el sistema de ecuaciones simultaneas obtenido}
    \vskip0pt plus 1filll
    \begin{columns}[onlytextwidth]
        \begin{column}{0.5\textwidth}
            \only<3-4>{
                \begin{gather*}
                    \onslide<3->{i_1 - i_{R_1} - i_{R_2} - i_2 = 0\\
                    \\}
                    \onslide<4>{
                        i_{R_1} = \frac{v_a - 0}{R_1}\\  
                        i_{R_2} = \frac{v_a - v_b}{R_2}
                    }
                \end{gather*}
            }
            \only<5-6>{
                \begin{gather*}
                    \onslide<5->{i_2 + i_{R_2} - i_{R_3} = 0\\
                    \\}
                    \onslide<6>{
                        i_{R_2} = \frac{v_a - v_b}{R_2}\\
                        i_{R_3} = \frac{v_b - 0}{R_3}
                    }
                \end{gather*}
            }
            \only<7>{
                \begin{gather*}
                    -i_1 + \frac{v_a - 0}{R_1} + \frac{v_a - v_b}{R_2} + i_2 = 0\\
                    \\
                    -i_2 - \frac{v_a - v_b}{R_2} + \frac{v_b - 0}{R_3} = 0
                \end{gather*}
            }
            \only<8-9>{
                \begin{gather*}
                     \left(\frac{1}{R_1}+\frac{1}{R_2}\right) \cdot v_a - \frac{1}{R_2} \cdot v_b = i_1 - i_2\\
                    \\
                    - \frac{1}{R_2} \cdot v_a + \left(\frac{1}{R_2} + \frac{1}{R_3}\right)  \cdot v_b = i_2
                \end{gather*}
                \vspace{0.2cm}
                \onslide<9>{
                \begin{equation*}
                    \begin{bmatrix} 
                        \frac{1}{R_1}+\frac{1}{R_2} & -\frac{1}{R_2} \\ 
                        - \frac{1}{R_2} & \frac{1}{R_2} + \frac{1}{R_3}\\
                    \end{bmatrix}
                    \begin{bmatrix} 
                        v_a \\ 
                        v_b 
                    \end{bmatrix}
                        =
                    \begin{bmatrix}
                        i_1 - i_2\\
                        i_2
                    \end{bmatrix}
                \end{equation*}
                }
            }
            \only<10->{
                \begin{align*}
                    \onslide<10->{R_1 &= \SI{1}{\ohm} & R_2 &= \SI{2}{\ohm} & R_3 &= \SI{3}{\ohm}\\
                    i_1 &= \SI{1}{\ampere} & & & i_2 &=\SI{2}{\ampere}}
                \end{align*}
                \onslide<11->{
                    \begin{equation*}
                    \begin{bmatrix} 
                        \frac{3}{2} & -\frac{1}{2} \\ 
                        -\frac{1}{2} & \frac{5}{6}\\
                    \end{bmatrix}
                    \begin{bmatrix} 
                      v_a \\ 
                      v_b 
                    \end{bmatrix}
                    =
                    \begin{bmatrix}
                        -1 \\
                        2
                    \end{bmatrix}
                    \end{equation*}
                }
                \begin{align*}
                    \onslide<12>{v_a &= \SI{0.167}{\volt} & v_b &= \SI{2.5}{\volt}}
                \end{align*}
            }
        \end{column}
        \begin{column}{0.5\textwidth}
        \centering
        \begin{circuitikz} [scale=0.85,transform shape]\draw
            (0,0)
                to[I,l=$i_1$]
            (0,3)
                to[short]
            (1.5,3)
                to[R,l=$R_1$]
            (1.5,0)
                to[short]
            (0,0)
            (1.5,3)
                to[R,l=$R_2$]
            (4.5,3)
                to[R,l=$R_3$]
            (4.5,0)
                --
            (0,0)
            (1.5,3) 
                --
            (1.5,4.5)
                to[I,l=$i_2$]
            (4.5,4.5)
                --
            (4.5,3)
            ;
            \draw [white](-1,-0.7) rectangle (5.7,5.5);
            \draw<2-> (1.5,0) node[ground]{} (1.5,3) node[above left]{$v_a$} (4.5,3)                 node[above right]{$v_b$};
            \draw<3-4> (1.5,3) to[open,i=$i_{R_1}$] (1.5,2.5);
            \draw<3-4> (1.5,3) to[open,i=$i_{R_2}$] (2,3);
            \fill<3-4> [red] (1.5,3) circle(3pt);
            \draw<5-6> (4.5,3) to[open,i_<=$i_{R_2}$] (4,3);
            \draw<5-6> (4.5,3) to[open,i=$i_{R_3}$] (4.5,2.5);
            \fill<5-6> [red] (4.5,3) circle(3pt);
        \end{circuitikz}
        \end{column}
        \end{columns}
\end{frame}

\begin{frame}{Análisis de nodos por inspección}
    \vspace{0.1cm}
    \only<2>{1. Seleccionar un nodo como referencia. Asignar voltajes a los demás nodos}
    \only<3>{2. Construir el sistema simultaneo de ecuaciones en el orden final pero sin coeficientes}
    \only<4>{3. Sumar los inversos de las resistencias que tocan al nodo $a$}
    \only<5>{4. Sumar los inversos de las resistencias que el nodo $a$ comparte con el nodo $b$ y agregar un negativo}
    \only<6>{5. Sumar las fuentes de corriente que se conectan al nodo $a$, tomar como positivas las fuentes cuya corriente entra al nodo}
    \only<7>{6. Sumar los inversos de las resistencias que el nodo $b$ comparte con el nodo $a$ y agregar un negativo}
    \only<8>{7. Sumar los inversos de las resistencias que tocan al nodo $b$}
    \only<9>{8. Sumar las fuentes de corriente que se conectan al nodo $b$, tomar como positivas las fuentes cuya corriente entra al nodo}
    \only<10->{9. Resolver el sistema de ecuaciones simultaneas obtenido}
    \vskip0pt plus 1filll
    \begin{columns}[onlytextwidth]
        \begin{column}{0.5\textwidth}
            \only<1-9>{
            \centering
            \begin{tikzpicture}
                \draw [white](-0.6,1) rectangle (7,5);
                \draw<3-> (1.5,4) node[right]{$v_a$};
                \draw<4>[red,thick]  (-0.5,4.5) rectangle (1.5,3.5);
                \draw<4-9> (1.5,4) node[left]{$\left(\frac{1}{R_1}+\frac{1}{R_2}\right)$}node{$\cdot$};
                \draw<3-> (4.5,4) node[right]{$v_b=$};
                \draw<5-9> (4.5,4) node[left]{$-\frac{1}{R_2}$}node{$\cdot$};
                \draw<5>[red,thick]  (2.2,4.5) rectangle (4.5,3.5);
                \draw<6-9> (6.8,4) node[left]{$i_1-i_2$};
                \draw<6>[red,thick]  (5.5,4.5) rectangle (6.8,3.5);
                \draw<3-> (1.5,2) node[right]{$v_a$};
                \draw<7-9> (1.5,2) node[left]{$-\frac{1}{R_2}$}node{$\cdot$};
                \draw<7>[red,thick]  (-0.5,2.5) rectangle (1.5,1.5);
                \draw<3-> (4.5,2) node[right]{$v_b=$};
                \draw<8-9> (4.5,2) node[left]{$+\left(\frac{1}{R_2}+\frac{1}{R_3}\right)$}node{$\cdot$};
                \draw<8>[red,thick]  (2.2,2.5) rectangle (4.5,1.5);
                \draw<9> (6.8,2) node[left]{$i_2$};
                \draw<9>[red,thick]  (5.5,2.5) rectangle (6.8,1.5);
            \end{tikzpicture}
            }
            \only<10->{
                \begin{align*}
                    \onslide<10->{R_1 &= \SI{1}{\ohm} & R_2 &= \SI{2}{\ohm} & R_3 &= \SI{3}{\ohm}\\
                    i_1 &= \SI{1}{\ampere} & & & i_2 &=\SI{2}{\ampere}}
                \end{align*}
                \onslide<11->{
                    \begin{equation*}
                    \begin{bmatrix} 
                        \frac{3}{2} & -\frac{1}{2} \\ 
                        -\frac{1}{2} & \frac{5}{6}\\
                    \end{bmatrix}
                    \begin{bmatrix} 
                      v_a \\ 
                      v_b 
                    \end{bmatrix}
                    =
                    \begin{bmatrix}
                        -1 \\
                        2
                    \end{bmatrix}
                    \end{equation*}
                }
                \begin{align*}
                    \onslide<12>{v_a &= \SI{0.167}{\volt} & v_b &= \SI{2.5}{\volt}}
                \end{align*}
            }
        \end{column}
        \begin{column}{0.5\textwidth}
            \centering
            \begin{circuitikz} [scale=0.85,transform shape]\draw
                (0,0)
                    to[I,l=$i_1$]
                (0,3)
                    to[short]
                (1.5,3)
                    to[R,l=$R_1$]
                (1.5,0)
                    to[short]
                (0,0)
                (1.5,3)
                    to[R,l=$R_2$]
                (4.5,3)
                    to[R,l=$R_3$]
                (4.5,0)
                    --
                (0,0)
                (1.5,3) 
                    --
                (1.5,4.5)
                    to[I,l=$i_2$]
                (4.5,4.5)
                    --
                (4.5,3)
                ;
                \draw [white](-1,-0.7) rectangle (5.7,5.5);
                \draw<2-> (1.5,0) node[ground]{} (1.5,3) node[above left]{$v_a$} (4.5,3)                 node[above right]{$v_b$};
                \fill<4-6> [red] (1.5,3) circle(3pt);
                \fill<7-9> [red] (4.5,3) circle(3pt);
            \end{circuitikz}
        \end{column}
    \end{columns}
\end{frame}




\begin{frame}{Supernodos}
    \only<1-2>{Se da cuando dos nodos de no referencia tienen una fuente de voltaje en común}
    \only<3>{1. Realizar inspección en el nodo $a$}
    \only<4>{2. Realizar inspección en el supernodo $b$-$c$}
    \only<5>{3. Incluir la ecuación auxiliar que relaciona los voltajes del supernodo con la fuente de voltaje}
    \only<6->{4. Resolver el sistema de ecuaciones simultaneas obtenido}
    \vskip0pt plus 1filll 
    \begin{columns}[onlytextwidth]
        \begin{column}{0.6\textwidth}
            \only<2>{A la unión de ambos nodos se le llama supernodo, este incluye a la fuente de voltaje y cualquier elemento conectado en paralelo}
            \only<3-5>{
                \onslide<3->{
                \begin{equation*}
                    v_a + 0\cdot v_b + 0\cdot v_c= v_1
                \end{equation*}
                }
                \onslide<4->{
                \begin{multline*}
                     -\left(\frac{1}{R_1}+\frac{1}{R_2}\right) \cdot v_a + \left(\frac{1}{R_2} + \frac{1}{R_3}\right) \cdot v_b \\+ \left(\frac{1}{R_1} + \frac{1}{R_4}\right) \cdot v_c= 0
                \end{multline*}
                \vspace{0.2cm}
                }
                \onslide<5>{
                \begin{equation*}
                    0\cdot v_a + v_b - v_c = v_2
                \end{equation*}
                }
            }
            \only<6->{
                \begin{align*}
                    \onslide<6->{R_1 &= \SI{1}{\ohm} & R_2 &= \SI{2}{\ohm} & R_3 &= \SI{3}{\ohm}\\
                     R_4 &= \SI{4}{\ohm} & v_1 &= \SI{10}{\volt} & v_2 &=\SI{20}{\volt}
                    }
                \end{align*}
                \onslide<7->{
                    \begin{equation*}
                    \begin{bmatrix} 
                        1 & 0 & 0\\
                        -\frac{3}{2} & \frac{5}{6} & \frac{5}{4} \\ 
                        0 & 1 & -1\\
                    \end{bmatrix}
                    \begin{bmatrix} 
                      v_a \\ 
                      v_b \\
                      v_c
                    \end{bmatrix}
                    =
                    \begin{bmatrix}
                        10 \\
                        0 \\
                        20
                    \end{bmatrix}
                    \end{equation*}
                }
                \begin{align*}
                    \onslide<8>{v_a &= \SI{10}{\volt} & v_b &= \SI{19.2}{\volt} & v_c &= \SI{-0.8}{\volt}}
                \end{align*}
            }
        \end{column}
        \begin{column}{0.4\textwidth}
            \centering
            \begin{circuitikz} [scale=0.8, transform shape]\draw
                (0,0)
                    to[V,l=$v_1$]
                    node[left]{$v_a$}
                (0,3)
                    to[short]
                (0,5)
                    to[R,l=$R_1$]
                    %node[ground]{}
                (5,5)
                    to[short]
                    node[right]{$v_c$}
                (5,3)
                    to[R,l=$R_4$]
                (5,0)
                    to[short]
                    node[ground]{}
                (2.5,0)
                    to[short]
                (0,0)
                (0,3)
                     to[R,l=$R_2$]
                     node[above]{$v_b$}
                (2.5,3)
                    to[R,l=$R_3$]
                (2.5,0)
                % (5,3)
                %     to[V,l_=$v_2$]
                % (2.5,3)
                ;
                \draw [white](-1,-0.9) rectangle (6,6.1);
                \draw<1-4,6-> (5,3) to[V,l_=$v_2$] (2.5,3);
                \draw<5> (5,3) to[V,l_=$v_2$,color=red] (2.5,3);
                \draw <2-3,5->[dashed, blue] (2.3,3.5) -- (5.2,3.5) -- (5.2,2.5) -- (2.3,2.5) -- cycle;
                \draw<4> [dashed, red] (2.3,3.5) -- (5.2,3.5) -- (5.2,2.5) -- (2.3,2.5) -- cycle;
                \fill<3> [red] (0,3) circle(3pt);
                \fill<4> [red] (2.5,3) circle(3pt) (5,3) circle(3pt);
            \end{circuitikz}
        \end{column}
    \end{columns}
\end{frame}

% \begin{frame}{Referencias}

% \bibliographystyle{ieeetr}

% \bibliography{referencias}

% \end{frame}


\end{document}