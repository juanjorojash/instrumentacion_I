\documentclass[aspectratio=169]{beamer}
\usetheme{Bruno}
\usepackage{amsmath}
\usepackage{amssymb}
\usepackage{siunitx}
\usepackage{float}
\usepackage{tikz}
\def\checkmark{\tikz\fill[scale=0.4](0,.35) -- (.25,0) -- (1,.7) -- (.25,.15) -- cycle;} 
\usepackage{url}
\usepackage[siunitx,american,RPvoltages]{circuitikz}
\ctikzset{capacitors/scale=0.7}
\ctikzset{diodes/scale=0.7}
\usepackage{tabularx}
\newcolumntype{C}{>{\centering\arraybackslash}X}
\renewcommand\tabularxcolumn[1]{m{#1}}% for vertical centering text in X column
\usepackage{tabu}
\usepackage[spanish,es-tabla,activeacute]{babel}
\usepackage{babelbib}
\usepackage{booktabs}
\usepackage{pgfplots}
\usepackage{hyperref}
\hypersetup{colorlinks = true,
            linkcolor = black,
            urlcolor  = blue,
            citecolor = blue,
            anchorcolor = blue}
\usepgfplotslibrary{units, fillbetween} 
\pgfplotsset{compat=1.16}
\usepackage{bm}
\usetikzlibrary{arrows, arrows.meta, shapes, 3d, perspective, positioning,mindmap,trees,backgrounds}
\renewcommand{\sin}{\sen} %change from sin to sen
\usepackage{bohr}
\setbohr{distribution-method = quantum,insert-missing = true}
\usepackage{elements}
\usepackage{verbatim}
\usepackage[edges]{forest}
\usepackage{etoolbox}
\usepackage{schemata}
\usepackage{appendix}
\usepackage{listings}

\definecolor{color_mate}{RGB}{255,255,128}
\definecolor{color_plas}{RGB}{255,128,255}
\definecolor{color_text}{RGB}{128,255,255}
\definecolor{color_petr}{RGB}{255,192,192}
\definecolor{color_made}{RGB}{192,255,192}
\definecolor{color_meta}{RGB}{192,192,255}
\newcommand\diagram[2]{\schema{\schemabox{#1}}{\schemabox{#2}}}

\definecolor{codegreen}{rgb}{0,0.6,0}
\definecolor{codegray}{rgb}{0.5,0.5,0.5}
\definecolor{codepurple}{rgb}{0.58,0,0.82}
\definecolor{backcolour}{rgb}{0.95,0.95,0.92}

\lstdefinestyle{mystyle}{
    backgroundcolor=\color{backcolour},   
    commentstyle=\color{codegreen},
    keywordstyle=\color{magenta},
    numberstyle=\tiny\color{codegray},
    stringstyle=\color{codepurple},
    basicstyle=\ttfamily\footnotesize,
    breakatwhitespace=false,         
    breaklines=true,                 
    captionpos=b,                    
    keepspaces=true,                 
    numbers=left,                    
    numbersep=5pt,                  
    showspaces=false,                
    showstringspaces=false,
    showtabs=false,                  
    tabsize=2
}

\lstset{style=mystyle}
\title{Electricidad I: \\ \emph{Superposición y transformación} \\ \emph{de fuentes}}
\author{
    Juan J. Rojas
}
\institute{Instituto Tecnológico de Costa Rica}
\date{\today}
\background{fig/background.jpg}
\begin{document}
\sisetup{unit-math-rm=\mathrm,math-rm=\mathrm} % change sinitx font
\sisetup{output-decimal-marker = {,}}}
\maketitle

\begin{frame}{Superposición}
    \emph{El teorema de superposición establece que en cualquier circuito lineal con más de una fuente independiente, la respuesta a través de cualquier elemento es la suma de las respuestas obtenidas debido a cada fuente independiente considerada por separado y con todas las demás fuentes independientes apagadas.}
\end{frame}

\begin{frame}{Superposición: pasos}
    \begin{columns}[onlytextwidth,T]
    \begin{column}{0.5\textwidth}
        \begin{enumerate}
            \item<+-> Se deja solamente una fuente independiente encendida. Las fuentes dependientes se dejan intactas.
            \item<+-> Se calcula la contribución de la fuente independiente analizada en el valor total de la corriente o el voltaje buscado.
        \end{enumerate}
    \end{column}
    \begin{column}{0.5\textwidth}
        \begin{enumerate}
        \setcounter{enumi}{2}
            \item<+-> Se repiten los pasos 1 y 2 para cada una de las fuentes independientes.
            \item<+-> Se suman las contribuciones de todas las fuentes independientes para determinar el valor total de la corriente o voltaje buscado. 
        \end{enumerate}
    \end{column}
    \end{columns}
\end{frame}

\begin{frame}{Superposición}
    \only<1>{Usando el teorema de superposición averigue $v_x$}
    \only<2>{Primero averiguamos la respuesta usando nodos}
    \only<3>{1. Se deja solamente la fuente de \SI{25}{\volt} encendida.}
    \only<4>{2. Se deja solamente la fuente de \SI{5}{\ampere} encendida.}
    \only<5>{3. Se suman las contribuciones de ambas fuentes.}
    \vskip0pt plus 1filll
    \begin{columns}
    \begin{column}{0.5\textwidth}
        \begin{circuitikz} [scale=0.8,transform shape]
        \draw<1-2,5>
        (0,0)
            to[V,l=\SI{25}{\volt}]
        (0,3)	
            to[R, l=\SI{20}{\ohm}]
        (2,3) -- (4,3)node[above]{$v_x$} 
            to[R,l=\SI{4}{\ohm}]
        (4,0) -- (0,0)
        (2,0)
            to[I,l=\SI{5}{\ampere}]
        (2,3)
        (4,0)node[ground]{} -- (6,0)
            to[cI,l_=$0.1v_x$]
        (6,3) -- (4,3)
        ;
        \draw<3>
        (0,0)
            to[V,l=\SI{25}{\volt}]
        (0,3)	
            to[R, l=\SI{20}{\ohm}]
        (2,3) -- (4,3)node[above]{$v_{x_1}$} 
            to[R,l=\SI{4}{\ohm}]
        (4,0) -- (0,0)
        (4,0)node[ground]{} -- (6,0)
            to[cI,l_=$0.1v_{x_1}$]
        (6,3) -- (4,3)
        ;
        \draw<4>
        (0,0) -- (0,3)	
            to[R, l=\SI{20}{\ohm}]
        (2,3) -- (4,3)node[above]{$v_{x_2}$} 
            to[R,l=\SI{4}{\ohm}]
        (4,0) -- (0,0)
        (2,0)
            to[I,l=\SI{5}{\ampere}]
        (2,3)
        (4,0)node[ground]{} -- (6,0)
            to[cI,l_=$0.1v_{x_2}$]
        (6,3) -- (4,3)
        ;
        \draw<1-> [white](-1.5,-1.5) rectangle (7.7,4.5);
        \end{circuitikz}
    \end{column}
    \begin{column}{0.5\textwidth}
        \only<2>{
            \begin{gather*}
            \left(\frac{1}{20}+\frac{1}{4}\right)v_x-\left(\frac{1}{20}\right)25 = 0.1v_x + 5\\[5pt]
            0.2v_x = 6.25\\[5pt]
            v_x = \SI{31.25}{\volt}
            \end{gather*}
        }
        \only<3>{
            \begin{gather*}
            \left(\frac{1}{20}+\frac{1}{4}\right)v_{x_1}-\left(\frac{1}{20}\right)25 = 0.1v_{x_1}\\[5pt]
            0.2v_{x_1} = 1.25\\[5pt]
            v_{x_1} = \SI{6.25}{\volt}
            \end{gather*}
        }
        \only<4>{
            \begin{gather*}
            \left(\frac{1}{20}+\frac{1}{4}\right)v_{x_2} = 0.1v_{x_2} + 5\\[5pt]
            0.2v_{x_2} = 5\\[5pt]
            v_{x_2} = \SI{25}{\volt}
            \end{gather*}
        }
        \only<5>{
            \begin{gather*}
            v_x = v_{x_1} + v_{x_2}\\[5pt]
            v_x = 6.25 + 25\\[5pt]
            v_x = \SI{31.25}{\volt}
            \end{gather*}
        }
    \end{column}
    \end{columns}
\end{frame}

\begin{frame}{Transformación de fuentes}
        \vfill
        \centering
        \begin{circuitikz} [scale=0.8]
            \draw<1>
            (4,0)
                to[short,o-]
            (0,0)
                to[V, v=$v_1$]
            (0,4)	
                to[R, l=$R_1$, -o]
            (4,4)
            (12,0)
                to[short,o-]
            (8,0)
                to[I, l=$i_1$]
            (8,4)
                to[short, -o]
            (12,4)
            (10,0)
                to[R, l_=$R_1$]
            (10,4)
            ;
            \draw<2>
            (4,0)
                to[short,o-]
            (0,0)
                to[cV, v=$v_1$]
            (0,4)	
                to[R, l=$R_1$, -o]
            (4,4)
            (12,0)
                to[short,o-]
            (8,0)
                to[cI, l=$i_1$]
            (8,4)
                to[short, -o]
            (12,4)
            (10,0)
                to[R, l_=$R_1$]
            (10,4)
            ;
            \draw[thick, >=triangle 45, <->] (4.5,2) -- (6.5,2);
            \draw(2,-1)node{$v_1=i_1\cdot R_1$};
            \draw(10,-1)node{$i_1=v_1 / R_1$};
            \draw[white](-2,-2) rectangle (15,5);
        \end{circuitikz}
\end{frame}


\begin{frame}{Transformación de fuentes}
    \centering
    \begin{tikzpicture}
        \draw<1,3>[ultra thick,-{Latex[length=4mm,width=4mm]}] (0,0) arc (135:45:3);
        \draw<2,4>[ultra thick,{Latex[length=4mm,width=4mm]}-] (0,0) arc (135:45:3);
    \end{tikzpicture}
    \begin{columns}
    \begin{column}{0.5\textwidth}
        \begin{circuitikz} [scale=0.65,transform shape]
        \draw (1,-1) [white] rectangle (3,4);
        \draw<1>
        (0,0)
            to[V,l=\SI{25}{\volt}]
        (0,3)	
            to[R, l=\SI{20}{\ohm}]
        (2,3) -- (4,3)node[above]{$v_x$} 
            to[R,l=\SI{4}{\ohm}]
        (4,0) -- (0,0)
        (2,0)
            to[I,l=\SI{5}{\ampere}]
        (2,3)
        (4,0)node[ground]{} -- (6,0)
            to[cI,l_=$0.1v_x$]
        (6,3) -- (4,3)
        ;
        \draw<2-3>
        (0,3)	
            to[R, l_=\SI{20}{\ohm}]
        (0,0)
        (0,3) -- (4,3)node[above]{$v_x$} 
            to[R,l=\SI{4}{\ohm}]
        (4,0) -- (0,0)
        (2,0)
            to[I,l=\SI{6.25}{\ampere}]
        (2,3)
        (4,0)node[ground]{} -- (6,0)
            to[cI,l_=$0.1v_x$]
        (6,3) -- (4,3)
        ;
        \draw<4,5>
        (2,3)	
            to[R, l_=$\frac{10}{3}$\si{\ohm}]
        (2,0)
        (2,3) -- (4,3)node[above]{$v_x$} 
        (4,0) -- (2,0)
        (4,0)node[ground]{} -- (6,0)
            to[cI,l_=$6.25 + 0.1v_x$]
        (6,3) -- (4,3)
        ;
        \end{circuitikz}
    \end{column}
    \begin{column}{0.5\textwidth}
    \begin{circuitikz} [scale=0.65,transform shape]
        \draw (1,-1) [white] rectangle (3,4);
        \draw<1-2>
        (-2,0)
            to[I,l=\SI{1.25}{\ampere}]
        (-2,3) -- (0,3)	
            to[R, l_=\SI{20}{\ohm}]
        (0,0) -- (-2,0)
        (0,3) -- (4,3)node[above]{$v_x$} 
            to[R,l=\SI{4}{\ohm}]
        (4,0) -- (0,0)
        (2,0)
            to[I,l=\SI{5}{\ampere}]
        (2,3)
        (4,0)node[ground]{} -- (6,0)
            to[cI,l_=$0.1v_x$]
        (6,3) -- (4,3)
        ;
        \draw<3,4>
        (2,3)	
            to[R, l_=\SI{20}{\ohm}]
        (2,0)
        (2,3) -- (4,3)node[above]{$v_x$} 
            to[R,l=\SI{4}{\ohm}]
        (4,0) -- (2,0)
        (4,0)node[ground]{} -- (6,0)
            to[cI,l_=$6.25 + 0.1v_x$]
        (6,3) -- (4,3)
        ;
        \draw<5>
        (3,3) node{\Large $v_x=\frac{10}{3} \left(6.25+0.1v_x\right)$}	
        (3,1.5) node{\Large $v_x= 20.8333 + 0.333v_x$}
        (3,0) node{\Large $v_x=$\SI{31.23}{\volt}}
        ;
        \end{circuitikz}
    \end{column}
    \end{columns}
\end{frame}

\end{document}