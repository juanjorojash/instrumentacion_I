\documentclass[aspectratio=169]{beamer}
\usetheme{Bruno}
\usepackage{amsmath}
\usepackage{amssymb}
\usepackage{siunitx}
\usepackage{float}
\usepackage{tikz}
\usepackage{url}
\usepackage[siunitx,american,RPvoltages]{circuitikz}
\ctikzset{capacitors/scale=0.7}
\ctikzset{diodes/scale=0.7}
\usepackage{tabularx}
\newcolumntype{C}{>{\centering\arraybackslash}X}
\renewcommand\tabularxcolumn[1]{m{#1}}% for vertical centering text in X column
\usepackage{tabu}
\usepackage[spanish,es-tabla,activeacute]{babel}
\usepackage{babelbib}
\usepackage{booktabs}
\usepackage{pgfplots}
\usepackage{hyperref}
\hypersetup{colorlinks = true,
            linkcolor = black,
            urlcolor  = blue,
            citecolor = blue,
            anchorcolor = blue}
\usepgfplotslibrary{units, fillbetween} 
\pgfplotsset{compat=1.16}
\usepackage{bm}
\usetikzlibrary{arrows, arrows.meta, shapes, 3d, perspective, positioning}
\renewcommand{\sin}{\sen} %change from sin to sen
\usepackage{bohr}
\setbohr{distribution-method = quantum,insert-missing = true}
\usepackage{elements}
\usepackage{verbatim}
\title{Electricidad I: \\ \emph{Teoremas de Thevenin, Norton y } \\ \emph{Máxima transferencia} \\ \emph{de potencia}}
\author{
    Juan J. Rojas
}
\institute{Instituto Tecnológico de Costa Rica}
\date{\today}
\background{fig/background.jpg}
\begin{document}
\input{comunes/postamble}
\maketitle

\begin{frame}{Teorema de Thevenin}
        \vspace{0.2cm}
        Cualquier circuito lineal de dos terminales puede reemplazarse por un circuito equivalente compuesto de una fuente de voltaje $v_{Th}$ en serie con un resistor $R_{Th}$
        \vskip0pt plus 1filll
        \centering
        \begin{circuitikz} [scale=0.8]
            \draw [white](0.5,-1) rectangle (18,5);
            \draw 
            (1,-0.5)
                rectangle node[align=center] {\small Circuito\\\small lineal}
            (3,4.5)
            (3,4)	
                to[short,-o]
            (4,4)node[above]{$a$}
                to[short]
            (6,4)
                to[generic, l=\small{Carga}, i>^=$i$]
            (6,0)
                to[short, -o]
                node[below]{$b$}
            (4,0)
                to[short]
            (3,0)
            (4,4)
                to[open, v^=$v$]
            (4,0)
            (11,0)
                to[V,l=$v_{Th}$]
            (11,4)
                to[R,l=$R_{Th}$,-o]
                node[above]{$a$}
            (14,4)
                to[short]
            (16,4)
                to[generic, l=\small{Carga}, i>^=$i$]
            (16,0)
                to[short,-o]
                node[below]{$b$}
            (14,0)
                to[short]
            (11,0)
            (14,4)
                to[open, v^=$v$]
            (14,0)
            ;
            \draw[thick, >=triangle 45, ->] (8,2) -- (9.5,2);
        \end{circuitikz}
\end{frame}

\begin{frame}{Teorema de Thevenin}
        \vspace{0.2cm}
        Se elimina la carga y podemos encontrar $v_{Th}$ 
        \vskip0pt plus 1filll
        \centering
        \begin{circuitikz} [scale=0.8]
            \draw [white](0.5,-1) rectangle (18,5);
            \draw
            (1,-0.5)
                rectangle node[align=center] {\small Circuito\\\small lineal}
            (3,4.5)
            (3,4)	
                to[short,-o]
            (4,4)node[above]{$a$}
            (4,0)node[below]{$b$}
                to[short,o-]
            (3,0)
            (4,4)
                to[open, v^=$v_{Th}$]
            (4,0)
            (11,0)
                to[V,l=$v_{Th}$]
            (11,4)
                to[R,l=$R_{Th}$,-o]
            (14,4)node[above]{$a$}
                to[open, v^=$v_{Th}$]
            (14,0)node[below]{$b$}
                to[short,o-]
            (11,0)
            ;
        \end{circuitikz}
\end{frame}

\begin{frame}{Teorema de Thevenin}
        \vspace{0.2cm}
        Si no hay fuentes dependientes, se apagan las fuentes independientes y podemos encontrar $R_{Th}$
        \vskip0pt plus 1filll
        \centering
        \begin{circuitikz} [scale=0.8]
            \draw [white](0.5,-1) rectangle (18,5);
            \draw
            (1,-0.5)
                rectangle node[align=center] {\small Circuito\\\small lineal\\\small con\\\small fuentes\\\small indep.\\\small apagadas}
            (3,4.5)
            (3,4)	
                to[short,-o]
            (4,4)node[above]{$a$}
            (4,0)node[below]{$b$}
                to[short,o-]
            (3,0)
            (4,4)
                to[open, l=$R_{Th}$]
            (4,0)
            (11,0)
                to[short]
            (11,4)
                to[R,l=$R_{Th}$,-o]
            (14,4)node[above]{$a$}
                to[open, l=$R_{Th}$]
            (14,0)node[below]{$b$}
                to[short,o-]
            (11,0)
            ;
            \draw[thick, >=triangle 45, <-] (4,1.6) -- (5,1.6);
            \draw[thick, >=triangle 45, <-] (14,1.6) -- (15,1.6);
        \end{circuitikz}
\end{frame}

\begin{frame}{Teorema de Thevenin}
        \vspace{0.2cm}
        Si hay fuentes dependientes, se apagan las fuentes independientes y usamos una fuente de prueba para encontrar $R_{Th}$
        \vskip0pt plus 1filll
        \begin{tabularx}{\linewidth}{X X}
            \centering
            \begin{circuitikz} [scale=0.8]
                \draw [white](0.5,-1) rectangle (18,5);
                \draw
                (1,-0.5)
                    rectangle node[align=center] {\small Circuito\\\small lineal\\\small con\\\small fuentes\\\small indep.\\\small apagadas}
                (3,4.5)
                (3,4)	
                    to[short,-o]
                (4,4)node[above]{$a$}
                    to[short]
                (6,4)
                    to[V, l=$v_{o}$, i<^=$i_o$, invert]
                (6,0)
                    to[short, -o]
                (4,0)node[below]{$b$}
                    to[short,o-]
                (3,0)
                (6.5,4)node[right]{$R_{Th} = \frac{v_o}{i_o}$}
                ;
            \end{circuitikz}
            &
            \centering
            \begin{circuitikz} [scale=0.8]\draw
                (1,-0.5)
                    rectangle node[align=center] {\small Circuito\\\small lineal\\\small con\\\small fuentes\\\small indep.\\\small apagadas}
                (3,4.5)
                (3,4)	
                    to[short,-o]
                (4,4)node[above]{$a$}
                    to[short]
                (6,4)
                    to[I, l=$i_{o}$, invert]
                (6,0)
                    to[short, -o]
                (4,0)node[below]{$b$}
                    to[short,o-]
                (3,0)
                (6.5,4)node[right]{$R_{Th} = \frac{v_o}{i_o}$}
                (4,4)
                    to[open, v^=$v_o$]
                (4,0)
                ;
            \end{circuitikz}
        \end{tabularx}
\end{frame}

\begin{frame}{Teorema de Norton}
        \vspace{0.2cm}
        Cualquier circuito lineal de dos terminales puede reemplazarse por un circuito equivalente compuesto de una fuente de corriente $i_{N}$ en paralelo con un resistor $R_{N}$
        \vskip0pt plus 1filll
        \centering
        \begin{circuitikz} [scale=0.8]
            \draw [white](0.5,-1) rectangle (18,5);
            \draw
            (1,-0.5)
                rectangle node[align=center] {\small Circuito\\\small lineal}
            (3,4.5)
            (3,4)	
                to[short,-o]
            (4,4)node[above]{$a$}
                to[short]
            (6,4)
                to[generic, l=\small{Carga}, i>^=$i$]
            (6,0)
                to[short, -o]
                node[below]{$b$}
            (4,0)
                to[short]
            (3,0)
            (4,4)
                to[open, v^=$v$]
            (4,0)
            (11,0)
                to[I,l=$i_{N}$]
            (11,4)
                to[short,-o]
                node[above]{$a$}
            (14,4)
                to[short]
            (16,4)
                to[generic, l=\small{Carga}, i>^=$i$]
            (16,0)
                to[short,-o]
                node[below]{$b$}
            (14,0)
                to[short]
            (11,0)
            (13,4)
                to[R, l_=$R_N$]
            (13,0)
            (14,4)
                to[open, v^=$v$]
            (14,0)
            ;
            \draw[thick, >=triangle 45, ->] (8,2) -- (9.5,2);
        \end{circuitikz}
\end{frame}

\begin{frame}{Teorema de Norton}
        \vspace{0.2cm}
        La carga se pone en cortocircuito y podemos encontrar $i_N$ 
        \vskip0pt plus 1filll
        \centering
        \begin{circuitikz} [scale=0.8]
            \draw [white](0.5,-1) rectangle (18,5);
            \draw
            (1,-0.5)
                rectangle node[align=center] {\small Circuito\\\small lineal}
            (3,4.5)
            (3,4)	
                to[short,-o]
            (4,4)node[above]{$a$}
                to[short]
            (6,4)
                to[short, i>^=$i_N$]
            (6,0)
                to[short, -o]
                node[below]{$b$}
            (4,0)
                to[short,o-]
            (3,0)
            (11,0)
                to[I,l=$i_{N}$]
            (11,4)
                to[short,-o]
                node[above]{$a$}
            (14,4)
                to[short]
            (16,4)
                to[short, i>^=$i_N$]
            (16,0)
                to[short,-o]
                node[below]{$b$}
            (14,0)
                to[short]
            (11,0)
            (13,4)
                to[R, l_=$R_N$]
            (13,0)
            ;
        \end{circuitikz}
\end{frame}

\begin{frame}{Teorema de Norton}
    La resistencia equivalente de Norton es igual a la de Thevenin y se puede encontrar usando el mismo método.\\
    \begin{equation*}
        R_{Th} = R_N
    \end{equation*}
    Luego se pueden relacionar por medio de la Ley de Ohm:
    \begin{equation*}
        i_N = \frac{v_{Th}}{R_{Th}}
    \end{equation*}
    \begin{equation*}
        v_{th} = i_N \cdot R_N
    \end{equation*}
\end{frame}

\begin{frame}{Teorema de máxima transferencia de potencia}
    Para transferir la máxima potencia a una carga, esta debe tener una resistencia igual a la resistencia de Thevenin.\\
    \vfill
    \centering
    \begin{circuitikz}[scale=0.8]\draw
        (0,0)
            to[V,l=$v_{Th}$]
        (0,4)
            to[R,l=$R_{Th}$,-o]
            node[above]{$a$}
        (3,4)
            to[short]
        (5,4)
            to[vR, mirror, l=$R_c$]
        (5,0)
            to[short,-o]
            node[below]{$b$}
        (3,0)
            to[short]
        (0,0)
        (3,4)
            to[open, v^=$v$]
        (3,0)
        (8,4)node[right]{$p=\left(\frac{v_{Th}}{R_{Th} + R_c}\right)^2 \cdot R_c$}
        (8,2)node[right]{$p_{max} \rightarrow R_c = R_{Th}$}
        (8,0)node[right]{$p_{max}=\frac{v_{Th}^2}{4\cdot R_{Th}}$}
        ;
    \end{circuitikz}
\end{frame}

\begin{frame}{Puente de Wheatstone}
    \vskip0pt plus 1filll
    \begin{columns}
    \begin{column}{0.5\textwidth}
    \only<1>{Si no fluye corriente por el instrumento de medición: 
        \begin{gather*}
            R_x = \frac{R_3}{R_1} R_2
        \end{gather*}
    se puede usar una resistencia calibrada en $R_2$ para medir $R_x$
    }\only<2>{Si las resistencias son fijas y no se  cumple que: 
        \begin{gather*}
            R_4 = \frac{R_3}{R_1} R_2
        \end{gather*}
    entonces el puente esta desequilibrado y existe una corriente fluyendo por el instrumento de medición
    }\only<3>{para medir la corriente que fluye por el instrumento se obtiene el equivalente de Thevenin entre $a$ y $b$, así la corriente será
        \begin{gather*}
            i = \frac{v_{Th}}{R_{Th} + R_{sh}}
        \end{gather*}
    donde $R_{sh}$ es la resistencia interna del instrumento de medición
    }
    % \only<4>{en instrumentación moderna se utiliza un amplificador operacional que amplifica el valor del voltaje que cae en $R_{sh}$, estos resitores se conocen como shunts y son de alta precisión.\\[4pt]
    % Por ejemplo un shunt de \SI{2}{\milli\ohm} podria usarse en conjunto con un {\color{blue}\href{https://www.ti.com/product/INA210}{INA210}} de Texas Instruments ($v_{cc}=\SI{5}{\volt}$, $k=200$, $v_{ref}=\SI{2.5}{\volt}$)
    % }\only<5>{si en nuestro microcontrolador medimos que $v_s=\SI{4}{\volt}$ entonces tenemos que:
    %     \begin{gather*}
    %         V_{ref} + k \cdot i_s \cdot R_{sh} = v_s\\[4pt]
    %         i_s = \frac{v_s-v{ref}}{k\cdot R_{sh}}\\[4pt]
    %         i_s = \frac{4-2,5}{200 \cdot \num{2e-3}}\\[4pt]
    %         i_s = \SI{3.75}{\ampere}
    %     \end{gather*}
    % }
    \end{column}
    \begin{column}{0.5\textwidth}
        \centering
        \begin{circuitikz} [scale=0.8,transform shape]
        \draw[white](-1,-1) rectangle (7,7);
        \draw <1>
            (0,0)
                to[V,l=$v_1$]
            (0,6)
                --
            (2,6)
                to[R,l=$R_1$]
            (2,3)
                to[vR,l=$R_2$]
            (2,0)
                to[short]
            (0,0)
            (2,6)
                to[short]
            (6,6)
                to[R,l=$R_3$]
            (6,3)
                to[R,l=$R_x$]
            (6,0)
                to[short]
            (2,0)
            (2,3)
                to[short,-o]node[above]{$a$}
            (3,3)
                to[qiprobe, l=\SI{0}{\ampere}]
            (5,3)
                node[above]{$b$}to[short,o-]
            (6,3)
            ;
        \draw <2-3>
            (0,0)
                to[V,l=$v_1$]
            (0,6)
                --
            (2,6)
                to[R,l=$R_1$]
            (2,3)
                to[R,l=$R_2$]
            (2,0)
                to[short]
            (0,0)
            (2,6)
                to[short]
            (6,6)
                to[R,l=$R_3$]
            (6,3)
                to[R,l=$R_4$]
            (6,0)
                to[short]
            (2,0)
            (2,3)
                to[short,-o]node[above]{$a$}
            (3,3)
                to[qiprobe, l=$\neq \SI{0}{\ampere}$]
            (5,3)
                node[above]{$b$}to[short,o-]
            (6,3)
            ;
        % \draw <4-5>
        %     (0,0)
        %         to[V,l=$v_1$]
        %     (0,6)
        %         --
        %     (2,6)
        %         to[R,l=$R_1$]
        %     (2,3)
        %         to[R,l=$R_2$]
        %     (2,0)
        %         to[short]
        %     (0,0)
        %     (2,6)
        %         to[short]
        %     (6,6)
        %         to[R,l=$R_3$]
        %     (6,3)
        %         to[R,l=$R_4$]
        %     (6,0)
        %         to[short]
        %     (2,0)
        %     (2,3)
        %         to[short,-o,i=$i_s$]node[below]{$a$}
        %     (3,3)node(a){}
        %         to[R, l=$R_{sh}$]
        %     (5,3)node(b){}
        %         node[below]{$b$}to[short,o-]
        %     (6,3)
        %     ;
        % \draw<4-5>
        %     (4,4.5) node[op amp, scale=0.6, rotate=90, yscale=-1] (opamp) {}
        %     (opamp.+) -| (a.center)
        %     (opamp.-) -| (b.center)
        %     (opamp.out) -- ++ (0,1) node[above]{$v_s=v_{ref} + k\cdot v_{ab}$}
        %     ;
        \end{circuitikz}
    \end{column}
    \end{columns}
\end{frame}
\end{document}