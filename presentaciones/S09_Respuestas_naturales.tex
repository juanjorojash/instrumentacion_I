\documentclass[aspectratio=169]{beamer}
\usetheme{Bruno}
\usepackage{amsmath}
\usepackage{amssymb}
\usepackage{siunitx}
\usepackage{float}
\usepackage{tikz}
\usepackage{url}
\usepackage[siunitx,american,RPvoltages]{circuitikz}
\ctikzset{capacitors/scale=0.7}
\ctikzset{diodes/scale=0.7}
\usepackage{tabularx}
\newcolumntype{C}{>{\centering\arraybackslash}X}
\renewcommand\tabularxcolumn[1]{m{#1}}% for vertical centering text in X column
\usepackage{tabu}
\usepackage[spanish,es-tabla,activeacute]{babel}
\usepackage{babelbib}
\usepackage{booktabs}
\usepackage{pgfplots}
\usepackage{hyperref}
\hypersetup{colorlinks = true,
            linkcolor = black,
            urlcolor  = blue,
            citecolor = blue,
            anchorcolor = blue}
\usepgfplotslibrary{units, fillbetween} 
\pgfplotsset{compat=1.16}
\usepackage{bm}
\usetikzlibrary{arrows, arrows.meta, shapes, 3d, perspective, positioning}
\renewcommand{\sin}{\sen} %change from sin to sen
\usepackage{bohr}
\setbohr{distribution-method = quantum,insert-missing = true}
\usepackage{elements}
\usepackage{verbatim}
\title{Electricidad I: \\ \emph{Respuestas naturales}}
\author{
    Juan J. Rojas
}
\institute{Instituto Tecnológico de Costa Rica}
\date{\today}
\background{fig/background.jpg}
\begin{document}
\input{comunes/postamble}
\maketitle
\begin{frame}{Resumen}
    \begin{center}
        \begin{tabularx}{14cm}{C C C}
        \toprule
        Variable & Ecuación $v-i$ & Ecuación $i-v$ \\
        \midrule
       Capacitor & $v=\frac{1}{C}\int_{t_0}^{t} i(t)dt + v(t_0) $ & $i = C\,\frac{dv}{dt}$ \\[5pt]
        Inductor & $v = L\,\frac{di}{dt}$ & $i=\frac{1}{L}\int_{t_0}^{t} v(t)dt + i(t_0) $\\[5pt]
        \bottomrule
        \end{tabularx}   
    \end{center}
\end{frame}

\begin{frame}{Circuito RC: respuesta natural}
    \begin{columns}[onlytextwidth]
    \begin{column}{0.5\textwidth}
        \centering
        \only<1>{
            \begin{circuitikz}[scale=0.8]\draw
            (0,2.5) node[spdt, rotate = 90, anchor=in] (Sw) {}
            (Sw.in) 
                to [C, l=$C$, voltage shift=3]
            (0,1) -- (0,0)
            (Sw.out 1) |- (-1,4) to [R,l=$R_f$,o-] (-4,4)
                to[V, l_=$V$, invert]
            (-4,0)
            (Sw.out 2) |- (1.5,4) to[short,o-] (3,4)
                to[R,l=$R$]
            (3,0) to[short,-o] (1.5,0) to[short,-o] (-1,0) -- (-4,0)
            (-1,4)
                to[open,v=$v$,voltage shift=-3]
            (-1,0)
            ;
        \end{circuitikz}
        }
        \only<2>{
            \begin{circuitikz}[scale=0.8]\draw
            (0,2.5) node[spdt, rotate = 90, yscale = -1, anchor=in] (Sw) {}
            (Sw.in) 
                to [C, l=$C$, voltage shift=3]
            (0,1) -- (0,0)
            (Sw.out 2) |- (-1,4) to [R,l=$R_f$,o-] (-4,4)
                to[V, l_=$V$, invert]
            (-4,0)
            (Sw.out 1) |- (1.5,4) to[short,o-] (3,4)
                to[R,l=$R$]
            (3,0) to[short,-o] (1.5,0) to[short,-o] (-1,0) -- (-4,0)
            (1.5,4)
                to[open,v=$v$,voltage shift=-3]
            (1.5,0)
            ;
        \end{circuitikz}
        }
    \end{column}
    \begin{column}{0.5\textwidth}
        \only<1>{
        Después de mucho tiempo...
        \begin{equation*}
            v = V
        \end{equation*}
        }
        \only<2>{
        Si se cambia la posición del interruptor en $t_0$
        \begin{equation*}
            v(t_0) = V
        \end{equation*}
        }
    
    \end{column}
\end{columns}
\end{frame}

\begin{frame}{Circuito RC: respuesta natural}
    \begin{columns}[onlytextwidth]
    \begin{column}{0.5\textwidth}
        \centering
        \only<1-12>{
            \begin{circuitikz}[scale=0.8]\draw
            (0,4)
                to [C, l=$C$, i>^=$i_C$]
            (0,0)
            (0,4) -- (2,4) to[short,o-] (4,4)
                to[R,l=$R$, i>^=$i_R$]
            (4,0) to[short,-o] (2,0) -- (0,0)
            (2,4)
                to[open,v=$v$,voltage shift=-3]
            (2,0)
            ;
        \end{circuitikz}
        }
        \only<13-15>{
            \begin{tikzpicture}
            \begin{axis}[
                %yshift=-1cm,
                width= 0.8\linewidth,
                height= 0.8\linewidth,
                %grid=major, % Display a grid
                %grid style={dashed,gray!30}, % Set the style
                %every axis plot/.append style={thick},
                ymin=0, ymax=1.1,
                xmin=-0.5, xmax=5.5,
                axis x line=bottom,
                axis y line=left,
                axis line style={-},
                ytick={0,0.368,1},
                % yticklabels={$0$,$\frac{\pi}{2}$,$\pi$,$\frac{3\pi}{2}$},
                xtick={0,1,2,3,4,5},
                % yticklabels=\empty,
                xlabel=$\frac{t-t_0}{\tau}$, % Set the labels
                ylabel=$\frac{v(t)}{v(t_0)}$,
                x unit=, 
                y unit=,
                yticklabel style={/pgf/number format/precision=3}
                %legend pos=north east,
                %legend style={cells={anchor=west}}
                %legend style={at={(1,0.7)},anchor=west}, % Put the legend below the plot
                %smooth,
                ]
                \addplot [mark=none, color=black,forget plot, thick]       
                table[x=t,y=mag,col sep=comma] {progras/RLC/resp_nat.csv};
                \addplot[color=red,dashed,mark=none] coordinates {
            		(-0.5,0.368)
            		(1,0.368)
            		(1,0)
            	};
            	\addplot[color=blue,dashed,mark=none] coordinates {
            		(0,1)
            		(1,0)
            	};
            \end{axis}
            \end{tikzpicture}
        }
    \end{column}
    \begin{column}{0.5\textwidth}
    \centering
    \only<1-3>{
        \onslide<1-3>{
        aplicando LCK tenemos que:
        \begin{equation*}
            i_C + i_R = 0
        \end{equation*}
        }
        \onslide<2-3>{
        y sabemos que: 
        \begin{equation*}
            i_C = C\,\frac{dv}{dt},\, i_R=\frac{v}{R}
        \end{equation*}
        }
        \onslide<3>{
        entonces:
        \begin{equation*}
            C\,\frac{dv}{dt} + \frac{v}{R} = 0
        \end{equation*}
        }
    }
    \only<4-6>{
        \onslide<4-6>{
        dividiendo entre $C\,$ y reordenando:
        \begin{equation*}
            \frac{dv}{dt} = -\frac{v}{RC}
        \end{equation*}
        }
        \onslide<5-6>{
        reordenando nuevamente:
        \begin{equation*}
            \frac{1}{v} dv = -\frac{1}{RC} dt
        \end{equation*}
        }
        \onslide<6>{
        integrando a ambos lados:
        \begin{equation*}
            \int\limits_{v(t_0)}^{v(t)}\frac{1}{v} dv = \int\limits_{t_0}^{t}-\frac{1}{RC} dt
        \end{equation*}
        }
    }
    \only<7-9>{
        \onslide<7-9>{
        tenemos que:
        \begin{equation*}
            \ln v\,\bigg|_{v(t_0)}^{v(t)} = -\frac{t}{RC}\,\bigg|_{t_0}^{t}
        \end{equation*}
        }
        \onslide<8-9>{
        evaluando:
        \begin{equation*}
            \ln v(t) - \ln v(t_0) = -\frac{1}{RC}\left(t - t_0\right)
        \end{equation*}
        }
        \onslide<9>{
        entonces:
        \begin{equation*}
            \ln\frac{v(t)}{v(t_0)} = -\frac{1}{RC}\left(t - t_0\right)
        \end{equation*}
        }
    }
    \only<10-12>{
        \onslide<10-12>{
        eliminamos el logaritmo natural:
        \begin{equation*}
            e\,^{\ln \left( v(t)/v(t_0)\right)} = e\,^{-(t-t_0)/RC}
        \end{equation*}
        }
        \onslide<11-12>{
        y obtenemos:
        \begin{equation*}
            \frac{v(t)}{v(t_0)} =  e\,^{-(t-t_0)/RC}
        \end{equation*}
        }
        \onslide<12>{
        pero la constante de tiempo es $\tau=RC\,$, entonces:
        \begin{equation*}
            \frac{v(t)}{v(t_0)} = e\,^{-(t-t_0)/\tau}
        \end{equation*}
        }
    }
    \only<13-15>{
        \onslide<13-15>{
        $\tau \rightarrow$ es la pendiente en $t(0)$, ver linea azul\\
        }
        
        \vspace{0.5cm}
                
        \onslide<14-15>{
        $\tau \rightarrow$ es el tiempo que se requiere para que el voltaje se reduzca en un $36.8\%$, ver linea roja
        }
                
        \vspace{0.5cm}
        
        \onslide<15>{
        En $5 \tau \rightarrow$ el voltaje se reduce a un $0.67\%$ de su valor original\\
        }
    }
    \end{column}
\end{columns}
\end{frame}

\begin{frame}{Circuito RL: respuesta natural}
    \begin{columns}[onlytextwidth]
    \begin{column}{0.5\textwidth}
        \centering
        \only<1>{
            \begin{circuitikz}[scale=0.8]\draw
            (0,2.5) node[spdt, rotate = 90, anchor=in] (Sw) {}
            (Sw.in) 
                to [L, l=$L$]
            (0,1) -- (0,0)
            (Sw.out 1) |- (-1,4) to [short,o-] (-4,4)
                to[I, l_=$I$]
            (-4,0)
            (Sw.out 2) |- (1.5,4) to[short,o-] (3,4)
                to[R,l=$R$]
            (3,0) to[short,-o] (1.5,0) -- (0,0)
            (-4,0) -- (-2,0) -- (-1,0) to[short,o-,i=$i$] (0,0)
            (-2,4)
                to[R,l_=$R_f$]
            (-2,0)
            ;
            \draw[white] (-5,-1) rectangle (5,5)
            ;
        \end{circuitikz}
        }
        \only<2>{
            \begin{circuitikz}[scale=0.8]\draw
            (0,2.5) node[spdt, rotate = 90, yscale = -1, anchor=in] (Sw) {}
            (Sw.in) 
                to [L, l=$L$]
            (0,1) -- (0,0)
            (Sw.out 2) |- (-1,4) to [short,o-] (-4,4)
                to[I, l_=$I$]
            (-4,0)
            (Sw.out 1) |- (1.5,4) to[short,o-,i=$i$] (3,4)
                to[R,l=$R$]
            (3,0) to[short,-o] (1.5,0) to[short,-o] (-1,0) -- (-4,0)
            (-2,4)
                to[R,l_=$R_f$]
            (-2,0)
            ;
            \draw[white] (-5,-1) rectangle (5,5)
            ;
        \end{circuitikz}
        }
    \end{column}
    \begin{column}{0.5\textwidth}
        \only<1>{
        Después de mucho tiempo...
        \begin{equation*}
            i = I
        \end{equation*}
        }
        \only<2>{
        Si se cambia la posición del interruptor en $t_0$
        \begin{equation*}
            i(t_0) = I
        \end{equation*}
        }
    \end{column}
\end{columns}
\end{frame}

\begin{frame}{Circuito RL: respuesta natural}
    \begin{columns}[onlytextwidth]
    \begin{column}{0.5\textwidth}
        \centering
        \only<1-12>{
            \begin{circuitikz}[scale=0.8]\draw
            (0,4)
                to [L, l=$L$, v_>=$v_L$]
            (0,0)
            (0,4) -- (2,4) to[short,o-,i=$i$] (4,4)
                to[R,l=$R$, v=$v_R$]
            (4,0) to[short,-o] (2,0) -- (0,0)
            ;
        \end{circuitikz}
        }
        \only<13-15>{
            \begin{tikzpicture}
            \begin{axis}[
                %yshift=-1cm,
                width= 0.8\linewidth,
                height= 0.8\linewidth,
                %grid=major, % Display a grid
                %grid style={dashed,gray!30}, % Set the style
                %every axis plot/.append style={thick},
                ymin=0, ymax=1.1,
                xmin=-0.5, xmax=5.5,
                axis x line=bottom,
                axis y line=left,
                axis line style={-},
                ytick={0,0.368,1},
                % yticklabels={$0$,$\frac{\pi}{2}$,$\pi$,$\frac{3\pi}{2}$},
                xtick={0,1,2,3,4,5},
                % yticklabels=\empty,
                xlabel=$\frac{t-t_0}{\tau}$, % Set the labels
                ylabel=$\frac{i(t)}{i(t_0)}$,
                x unit=, 
                y unit=,
                yticklabel style={/pgf/number format/precision=3}
                %legend pos=north east,
                %legend style={cells={anchor=west}}
                %legend style={at={(1,0.7)},anchor=west}, % Put the legend below the plot
                %smooth,
                ]
                \addplot [mark=none, color=black,forget plot, thick]       
                table[x=t,y=mag,col sep=comma] {progras/RLC/resp_nat.csv};
                \addplot[color=red,dashed,mark=none] coordinates {
            		(-0.5,0.368)
            		(1,0.368)
            		(1,0)
            	};
            	\addplot[color=blue,dashed,mark=none] coordinates {
            		(0,1)
            		(1,0)
            	};
            \end{axis}
            \end{tikzpicture}
        }
    \end{column}
    \begin{column}{0.5\textwidth}
    \centering
    \only<1-3>{
        \onslide<1-3>{
        aplicando LVK tenemos que:
        \begin{equation*}
            v_L + v_R = 0
        \end{equation*}
        }
        \onslide<2-3>{
        y sabemos que: 
        \begin{equation*}
            v_L = L\,\frac{di}{dt},\, v_R=iR
        \end{equation*}
        }
        \onslide<3>{
        entonces:
        \begin{equation*}
            L\,\frac{di}{dt} + iR = 0
        \end{equation*}
        }
    }
    \only<4-6>{
        \onslide<4-6>{
        dividiendo entre $L\,$ y reordenando:
        \begin{equation*}
            \frac{di}{dt} = -\frac{iR}{L}
        \end{equation*}
        }
        \onslide<5-6>{
        reordenando nuevamente:
        \begin{equation*}
            \frac{1}{i} di = -\frac{R}{L} dt
        \end{equation*}
        }
        \onslide<6>{
        integrando a ambos lados:
        \begin{equation*}
            \int\limits_{i(t_0)}^{i(t)}\frac{1}{i} di = \int\limits_{t_0}^{t}-\frac{R}{L} dt
        \end{equation*}
        }
    }
    \only<7-9>{
        \onslide<7-9>{
        tenemos que:
        \begin{equation*}
            \ln i\,\bigg|_{i(t_0)}^{i(t)} = -\frac{tR}{L}\,\bigg|_{t_0}^{t}
        \end{equation*}
        }
        \onslide<8-9>{
        evaluando:
        \begin{equation*}
            \ln i(t) - \ln i(t_0) = -\frac{R}{L}\left(t - t_0\right)
        \end{equation*}
        }
        \onslide<9>{
        entonces:
        \begin{equation*}
            \ln\frac{i(t)}{i(t_0)} = -\frac{R}{L}\left(t - t_0\right)
        \end{equation*}
        }
    }
    \only<10-12>{
        \onslide<10-12>{
        eliminamos el logaritmo natural:
        \begin{equation*}
            e\,^{\ln \left( i(t)/i(t_0)\right)} = e\,^{-(t-t_0)R/L}
        \end{equation*}
        }
        \onslide<11-12>{
        y obtenemos:
        \begin{equation*}
            \frac{i(t)}{i(t_0)} =  e\,^{-(t-t_0)R/L}
        \end{equation*}
        }
        \onslide<12>{
        pero la constante de tiempo es $\tau=L/R\,$, entonces:
        \begin{equation*}
            \frac{i(t)}{i(t_0)} = e\,^{-(t-t_0)/\tau}
        \end{equation*}
        }
    }
    \only<13-15>{
        \onslide<13-15>{
        $\tau \rightarrow$ es la pendiente en $t(0)$, ver linea azul\\
        }
        
        \vspace{0.5cm}
        
        \onslide<14-15>{
        $\tau \rightarrow$ es el tiempo que se requiere para que la corriente se reduzca a un $36.8\%$ de su valor original, ver linea roja\\
        }
        
        \vspace{0.5cm}
        
        \onslide<15>{
        En $5 \tau \rightarrow$ la corriente se reduce a un $0.67\%$ de su valor original\\
        }
    }
    \end{column}
\end{columns}
\end{frame}




% \begin{frame}{Referencias}

% \bibliographystyle{ieeetr}

% \bibliography{referencias}

% \end{frame}

\end{document}