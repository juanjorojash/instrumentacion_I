\documentclass[aspectratio=169]{beamer}
\usetheme{Bruno}
\usepackage{amsmath}
\usepackage{amssymb}
\usepackage{siunitx}
\usepackage{float}
\usepackage{tikz}
\def\checkmark{\tikz\fill[scale=0.4](0,.35) -- (.25,0) -- (1,.7) -- (.25,.15) -- cycle;} 
\usepackage{url}
\usepackage[siunitx,american,RPvoltages]{circuitikz}
\ctikzset{capacitors/scale=0.7}
\ctikzset{diodes/scale=0.7}
\usepackage{tabularx}
\newcolumntype{C}{>{\centering\arraybackslash}X}
\renewcommand\tabularxcolumn[1]{m{#1}}% for vertical centering text in X column
\usepackage{tabu}
\usepackage[spanish,es-tabla,activeacute]{babel}
\usepackage{babelbib}
\usepackage{booktabs}
\usepackage{pgfplots}
\usepackage{hyperref}
\hypersetup{colorlinks = true,
            linkcolor = black,
            urlcolor  = blue,
            citecolor = blue,
            anchorcolor = blue}
\usepgfplotslibrary{units, fillbetween} 
\pgfplotsset{compat=1.16}
\usepackage{bm}
\usetikzlibrary{arrows, arrows.meta, shapes, 3d, perspective, positioning,mindmap,trees,backgrounds}
\renewcommand{\sin}{\sen} %change from sin to sen
\usepackage{bohr}
\setbohr{distribution-method = quantum,insert-missing = true}
\usepackage{elements}
\usepackage{verbatim}
\usepackage[edges]{forest}
\usepackage{etoolbox}
\usepackage{schemata}
\usepackage{appendix}
\usepackage{listings}

\definecolor{color_mate}{RGB}{255,255,128}
\definecolor{color_plas}{RGB}{255,128,255}
\definecolor{color_text}{RGB}{128,255,255}
\definecolor{color_petr}{RGB}{255,192,192}
\definecolor{color_made}{RGB}{192,255,192}
\definecolor{color_meta}{RGB}{192,192,255}
\newcommand\diagram[2]{\schema{\schemabox{#1}}{\schemabox{#2}}}

\definecolor{codegreen}{rgb}{0,0.6,0}
\definecolor{codegray}{rgb}{0.5,0.5,0.5}
\definecolor{codepurple}{rgb}{0.58,0,0.82}
\definecolor{backcolour}{rgb}{0.95,0.95,0.92}

\lstdefinestyle{mystyle}{
    backgroundcolor=\color{backcolour},   
    commentstyle=\color{codegreen},
    keywordstyle=\color{magenta},
    numberstyle=\tiny\color{codegray},
    stringstyle=\color{codepurple},
    basicstyle=\ttfamily\footnotesize,
    breakatwhitespace=false,         
    breaklines=true,                 
    captionpos=b,                    
    keepspaces=true,                 
    numbers=left,                    
    numbersep=5pt,                  
    showspaces=false,                
    showstringspaces=false,
    showtabs=false,                  
    tabsize=2
}

\lstset{style=mystyle}
\title{Electricidad I: \\ \emph{Funciones de singularidad}}
\author{
    Juan J. Rojas
}
\institute{Instituto Tecnológico de Costa Rica}
\date{\today}
\background{fig/background.jpg}
\begin{document}
\sisetup{unit-math-rm=\mathrm,math-rm=\mathrm} % change sinitx font
\sisetup{output-decimal-marker = {,}}
\maketitle

\begin{frame}{Funciones de singularidad}
    \begin{columns}[onlytextwidth]
    \begin{column}{0.6\textwidth}
    \begin{itemize}
        \item La función, su derivada o ambas son discontinuas
        \item Permiten modelar matemáticamente operaciones de conmutación
    \end{itemize}
    \footnotesize
    \vspace{0.5cm}
    \begin{tabularx}{\textwidth}{X l}
        \toprule
        Función & Nombre\\
        \midrule
        $u(t-t_0)$ & escalón unitario o escalón de Heaviside \\[5pt]
        $\delta(t-t_0)$ & impulso unitario o delta de Dirac \\[5pt]
        $r(t-t_0)$ & rampa unitaria  \\[5pt]
        \bottomrule
    \end{tabularx}
    \normalsize 
    \end{column}
    \begin{column}{0.4\textwidth}
    \centering
        \begin{tikzpicture}[scale=0.8]
            \begin{axis}[%
                width= 0.9\linewidth,
                height= 0.5\linewidth,
                axis lines=center,
                xlabel=$t$,
                ylabel=$u(t-t_0)$,
                x label style={anchor=west},
                y label style={anchor=south},
                ytick={1},
                ymax=1.5, % or enlarge y limits=upper
                xticklabels={$t_0$},
                xtick={2},
                xmin=-1,
                ]
            \addplot+[no marks, thick] coordinates {(-1,0) (2,0) (2,1) (4,1)};
            \end{axis}
        \end{tikzpicture}
        \begin{tikzpicture}[scale=0.8]
            \begin{axis}[%
                width= 0.9\linewidth,
                height= 0.5\linewidth,
                axis lines=center,
                xlabel=$t$,
                ylabel=$\delta(t-t_0)$,
                x label style={anchor=west},
                y label style={anchor=south},
                ytick=false,
                ymax=1.5, % or enlarge y limits=upper
                xticklabels={$t_0$},
                xtick={2},
                xmin=-1,
                ]
            \addplot+[no marks, thick] coordinates {(-1,0) (2,0) (2,1) (2,0) (4,0)};
            \draw[->,blue,thick](2,1)--(2,1.3);
            \end{axis}
        \end{tikzpicture}
        \begin{tikzpicture}[scale=0.8]
            \begin{axis}[%
                width= 0.9\linewidth,
                height= 0.5\linewidth,
                axis lines=center,
                xlabel=$t$,
                ylabel=$r(t-t_0)$,
                x label style={anchor=west},
                y label style={anchor=south},
                ytick={1},
                ymax=1.5, % or enlarge y limits=upper
                xticklabels={$t_0$,$t_0+1$},
                xtick={1,3},
                xmin=-1,
                xmax=4,
                ]
            \addplot+[no marks, thick] coordinates {(-1,0) (1,0) (3,1) (5,2)};
            \draw[dashed, gray] (0,1) -- (3,1) -- (3,0);
            \end{axis}
        \end{tikzpicture}
    \end{column}
\end{columns}
\end{frame}

\begin{frame}{Función escalón unitario}
    \begin{columns}[onlytextwidth]
    \begin{column}{0.5\textwidth}
        \centering
        \begin{tikzpicture}[scale=1]
            \begin{axis}[%
                width= 0.9\linewidth,
                height= 0.5\linewidth,
                axis lines=center,
                xlabel=$t$,
                ylabel=$u(t-t_0)$,
                x label style={anchor=west},
                y label style={anchor=south},
                ytick={1},
                ymax=1.5, % or enlarge y limits=upper
                xticklabels={$t_0$},
                xtick={2},
                xmin=-1,
                ]
            \addplot+[no marks, thick] coordinates {(-1,0) (2,0) (2,1) (4,1)};
            \end{axis}
        \end{tikzpicture}
    \end{column}
    \centering
    \begin{column}{0.5\textwidth}
        \onslide<+->{
            \begin{equation*}
                u(t-t_0) =
                \begin{cases}
                    0, & t < t_0\\
                    1, & t > t_0
                \end{cases}
            \end{equation*}
            \vspace{0.5cm}
        }
        \onslide<+->{
            \begin{equation*}
                \frac{d}{dt} u(t-t_0) = \delta(t-t_0)
            \end{equation*}
            \vspace{0.5cm}
        }
        \onslide<+->{
            \begin{equation*}
                \int_{-\infty}^{t} u(t-t_0)dt = r(t-t_0)
            \end{equation*}
        }
    \end{column}
\end{columns}
\end{frame}

\begin{frame}{Función impulso unitario}
    \begin{columns}[onlytextwidth]
    \begin{column}{0.5\textwidth}
        \centering
        \begin{tikzpicture}[scale=1]
            \begin{axis}[%
                width= 0.9\linewidth,
                height= 0.5\linewidth,
                axis lines=center,
                xlabel=$t$,
                ylabel=$\delta(t-t_0)$,
                x label style={anchor=west},
                y label style={anchor=south},
                ytick=false,
                ymax=1.5, % or enlarge y limits=upper
                xticklabels={$t_0$},
                xtick={2},
                xmin=-1,
                ]
            \addplot+[no marks, thick] coordinates {(-1,0) (2,0) (2,1) (2,0) (4,0)};
            \draw[->,thick,blue](2,1)--(2,1.3);
            \end{axis}
        \end{tikzpicture}
    \end{column}
    \centering
    \begin{column}{0.5\textwidth}
        \onslide<+->{
            \begin{equation*}
                \delta(t-t_0) =
                \begin{cases}
                    0, & t \ne t_0\\
                    \infty, & t = t_0
                \end{cases}
            \end{equation*}
        \vspace{0.5cm}
        }
        \onslide<+->{
            \begin{equation*}
                \int_{t_0^-}^{t_0^+} \delta(t-t_0)dt = 1
            \end{equation*}
            \vspace{0.5cm}
        }
        \onslide<+->{
            \begin{equation*}
                \int_{-\infty}^{t} \delta(t-t_0)dt = u(t-t_0)
            \end{equation*}
        }
    \end{column}
\end{columns}
\end{frame}

\begin{frame}{Función rampa unitaria}
    \begin{columns}[onlytextwidth]
    \begin{column}{0.5\textwidth}
        \centering
        \begin{tikzpicture}[scale=1]
            \begin{axis}[%
                width= 0.9\linewidth,
                height= 0.5\linewidth,
                axis lines=center,
                xlabel=$t$,
                ylabel=$r(t-t_0)$,
                x label style={anchor=west},
                y label style={anchor=south},
                ytick={1},
                ymax=1.5, % or enlarge y limits=upper
                xticklabels={$t_0$,$t_0+1$},
                xtick={1,3},
                xmin=-1,
                xmax=4,
                ]
            \addplot+[no marks, thick] coordinates {(-1,0) (1,0) (3,1) (5,2)};
            \draw[dashed, gray] (0,1) -- (3,1) -- (3,0);
            \end{axis}
        \end{tikzpicture}
    \end{column}
    \centering
    \begin{column}{0.5\textwidth}
        \onslide<+->{
        \begin{equation*}
            r(t-t_0) =
            \begin{cases}
                0, & t \leq t_0\\
                t-t_0, & t \geq t_0
            \end{cases}
        \end{equation*}
        \vspace{0.5cm}
        }
        \onslide<+->{
        \begin{equation*}
            \frac{d}{dt} r(t-t_0) = u(t-t_0)
        \end{equation*}
        \vspace{0.5cm}
        }
    \end{column}
\end{columns}
\end{frame}


\begin{frame}{Conmutación de fuente de voltaje}
    \begin{columns}[onlytextwidth]
    \begin{column}{0.5\textwidth}
    \centering
            \begin{circuitikz}[scale=0.8]
            \draw[white](-3,-1) rectangle (3.5,5.5);
            \draw
            (0,3.6) node[cute spdt up arrow, anchor=out 1, rotate=180] (Sw) {}
            (Sw.in) -| (2.5,3.5)
            (Sw.out 2) -| (-2,3.5)
                to[V, l_=$V$, invert]
            (-2,0) -| (Sw.cout 1) |- (2.5,0)
            (Sw.in)node[above, yshift=0.3cm]{$t = t_0$}
            (2.5,3.5) to[R,l=$R$,v=$v_R$, voltage shift=3] (2.5,0)
            ;
            \end{circuitikz}
    \end{column}
    \centering
    \begin{column}{0.5\textwidth}
        \only<+>{
            \begin{equation*}
                v_R =
                \begin{cases}
                    0, & t < t_0\\
                    V, & t > t_0
                \end{cases}
            \end{equation*}
            \vspace{0.5cm}
        }
        \only<+>{
            \begin{circuitikz}[scale=0.8]
                \draw[white](-5,-1) rectangle (1,5);
                \draw
                (-2,4)
                    to[V, l_=$V\cdot u(t-t_0)$, invert]
                (-2,0)
                (-2,4) -- (1,4)
                to [R,l=$R$,v=$v_R$, voltage shift=3]
                (1,0) -- (-2,0)
                ;
            \end{circuitikz}
        }
    \end{column}
    \end{columns}
\end{frame}

\begin{frame}{Conmutación de fuente de corriente}
    \begin{columns}[onlytextwidth]
    \begin{column}{0.5\textwidth}
    \centering
            \begin{circuitikz}[scale=0.8]
            \draw[white](-3,-1) rectangle (3.5,5.5);
            \draw
            (0,3.6) node[cute spdt down arrow, anchor=out 2] (Sw) {}
            (Sw.out 1) -| (2.5,3.5)
            (Sw.in) -| (-2.5,3.5)
                to[I, l_=$I$, invert]
            (-2.5,0) -| (Sw.cout 2) |- (2.5,0)
            (Sw.in)node[above, yshift=0.3cm]{$t = t_0$}
            (2.5,3.5) to[R,l=$R$,i>^=$i_R$] (2.5,0)
            ;
            \end{circuitikz}
    \end{column}
    \centering
    \begin{column}{0.5\textwidth}
        \only<+>{
            \begin{equation*}
                i_R =
                \begin{cases}
                    0, & t < t_0\\
                    I, & t > t_0
                \end{cases}
            \end{equation*}
            \vspace{0.5cm}
        }
        \only<+>{
            \begin{circuitikz}[scale=0.8]
                \draw[white](-5,-1) rectangle (1,5);
                \draw
                (-2,4)
                    to[I, l_=$I\cdot u(t-t_0)$, invert]
                (-2,0)
                (-2,4) -- (1,4)
                to [R,l=$R$,i>^=$i_R$]
                (1,0) -- (-2,0)
                ;
            \end{circuitikz}
        }
    \end{column}
    \end{columns}
\end{frame}

\begin{frame}{Ejemplos: compuerta}
    \begin{columns}[onlytextwidth]
    \begin{column}{0.5\textwidth}
        \centering
        \begin{tikzpicture}[scale=1]
            \begin{axis}[%
                width= 0.9\linewidth,
                height= 0.5\linewidth,
                axis lines=center,
                xlabel=$t$,
                ylabel=$g(t)$,
                x label style={anchor=west},
                y label style={anchor=south},
                ytick={1},
                ymax=1.5, % or enlarge y limits=upper
                xticklabels={$t_0$,$t_1$},
                xtick={1,3},
                xmin=-1,
                xmax=4,
                ]
            \addplot+[no marks, thick] coordinates {(-1,0) (1,0) (1,1) (3,1) (3,0) (4,0)};
            \end{axis}
        \end{tikzpicture}
        \begin{equation*}
            g(t) = u(t-t_0) - u(t-t_1)
        \end{equation*}
    \end{column}
    \begin{column}{0.5\textwidth}
    \centering
        \begin{tikzpicture}[scale=0.9]
            \begin{axis}[%
                width= 0.9\linewidth,
                height= 0.5\linewidth,
                axis lines=center,
                xlabel=$t$,
                ylabel=$u(t-t_0)$,
                x label style={anchor=west},
                y label style={anchor=south},
                ytick={1},
                ymax=1.5, % or enlarge y limits=upper
                xticklabels={$t_0$, $t_1$},
                xtick={1,3},
                xmin=-1,
                xmax=4
                ]
            \addplot+[no marks, thick] coordinates {(-1,0) (1,0) (1,1) (4,1)};
            \end{axis}
        \end{tikzpicture}
        \begin{tikzpicture}[scale=0.9]
            \begin{axis}[%
                width= 0.9\linewidth,
                height= 0.5\linewidth,
                axis lines=center,
                xlabel=$t$,
                ylabel=$u(t-t_1)$,
                x label style={anchor=west},
                y label style={anchor=south},
                ytick={1},
                ymax=1.5, % or enlarge y limits=upper
                xticklabels={$t_0$, $t_1$},
                xtick={1,3},
                xmin=-1,
                xmax=4
                ]
            \addplot+[no marks, thick] coordinates {((-1,0) (3,0) (3,1) (4,1)};
            \end{axis}
        \end{tikzpicture}
    \end{column}
\end{columns}
\end{frame}

\begin{frame}{Ejemplos: triangulo}
    \begin{columns}[onlytextwidth]
    \begin{column}{0.5\textwidth}
        \centering
        \begin{tikzpicture}[scale=1]
            \begin{axis}[%
                width= 0.9\linewidth,
                height= 0.5\linewidth,
                axis lines=center,
                xlabel=$t$,
                ylabel=$h(t)$,
                x label style={anchor=west},
                y label style={anchor=south},
                ytick={1},
                ymax=1.5, % or enlarge y limits=upper
                xticklabels={$t_0$,$t_1$},
                xtick={1,2},
                xmin=-1,
                xmax=4
                ]
            \addplot+[no marks, thick] coordinates {(-1,0) (1,0) (2,1) (2,0) (4,0)};
            \end{axis}
        \end{tikzpicture}
        \begin{align*}
            h(t) &= r(t-t_0) \cdot \left(1- u(t-t_1)\right)\\
            & = r(t-t_0) - r(t-t_0)\cdot u(t-t_1)
        \end{align*}
    \end{column}
    \begin{column}{0.5\textwidth}
    \centering
        \begin{tikzpicture}[scale=0.8]
            \begin{axis}[%
                width= 0.9\linewidth,
                height= 0.55\linewidth,
                axis lines=center,
                xlabel=$t$,
                ylabel=$r(t-t_0)$,
                x label style={anchor=west},
                y label style={anchor=south},
                ytick={1,2},
                ymax=2.2, % or enlarge y limits=upper
                xticklabels={$t_0$,$t_1$},
                xtick={1,2},
                xmin=-1,
                xmax=4
                ]
            \addplot+[no marks, thick] coordinates {(-1,0) (1,0) (2,1) (3,2) (4,3)};
            \end{axis}
        \end{tikzpicture}
        \begin{tikzpicture}[scale=0.8]
            \begin{axis}[%
                width= 0.9\linewidth,
                height= 0.55\linewidth,
                axis lines=center,
                xlabel=$t$,
                ylabel=$r(t-t_0)\cdot u(t-t_1)$,
                x label style={anchor=west},
                y label style={anchor=south},
                ytick={1,2},
                ymax=2.2, % or enlarge y limits=upper
                xticklabels={$t_0$,$t_1$},
                xtick={1,2},
                xmin=-1,
                xmax=4
                ]
            \addplot+[no marks, thick] coordinates {(-1,0) (1,0) (2,0) (2,1) (3,2) (4,3)};
            \end{axis}
        \end{tikzpicture}
    \end{column}
\end{columns}
\end{frame}

\begin{frame}{Ejemplos: triangulo (de nuevo) }
    \begin{columns}[onlytextwidth]
    \begin{column}{0.5\textwidth}
        \centering
        \begin{tikzpicture}[scale=1]
            \begin{axis}[%
                width= 0.9\linewidth,
                height= 0.5\linewidth,
                axis lines=center,
                xlabel=$t$,
                ylabel=$h(t)$,
                x label style={anchor=west},
                y label style={anchor=south},
                ytick={1},
                ymax=1.5, % or enlarge y limits=upper
                xticklabels={$t_0$,$t_1$},
                xtick={1,2},
                xmin=-1,
                xmax=4
                ]
            \addplot+[no marks, thick] coordinates {(-1,0) (1,0) (2,1) (2,0) (4,0)};
            \end{axis}
        \end{tikzpicture}
        \begin{align*}
            h(t) &= r(t-t_0) \cdot \left(1- u(t-t_1)\right)\\
            & = r(t-t_0) \cdot u(-t+t_1)
        \end{align*}
    \end{column}
    \begin{column}{0.5\textwidth}
    \centering
        \begin{tikzpicture}[scale=0.8]
            \begin{axis}[%
                width= 0.9\linewidth,
                height= 0.55\linewidth,
                axis lines=center,
                xlabel=$t$,
                ylabel=$r(t-t_0)$,
                x label style={anchor=west},
                y label style={anchor=south},
                ytick={1,2},
                ymax=2.2, % or enlarge y limits=upper
                xticklabels={$t_0$,$t_1$},
                xtick={1,2},
                xmin=-1,
                xmax=4
                ]
            \addplot+[no marks, thick] coordinates {(-1,0) (1,0) (2,1) (3,2) (4,3)};
            \end{axis}
        \end{tikzpicture}
        \begin{tikzpicture}[scale=0.8]
            \begin{axis}[%
                width= 0.9\linewidth,
                height= 0.55\linewidth,
                axis lines=center,
                xlabel=$t$,
                ylabel=$u(-t+t_1)$,
                x label style={anchor=west},
                y label style={anchor=south},
                ytick={1,2},
                ymax=2.2, % or enlarge y limits=upper
                xticklabels={$t_0$,$t_1$},
                xtick={1,2},
                xmin=-1,
                xmax=4
                ]
            \addplot+[no marks, thick] coordinates {(-1,1) (1,1) (2,1) (2,0) (3,0) (4,0)};
            \end{axis}
        \end{tikzpicture}
    \end{column}
\end{columns}
\end{frame}

% \begin{frame}{Referencias}

% \bibliographystyle{ieeetr}

% \bibliography{referencias}

% \end{frame}

\end{document}