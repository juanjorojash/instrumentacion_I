\documentclass[aspectratio=169]{beamer}
\usetheme{Bruno}
\usepackage{amsmath}
\usepackage{amssymb}
\usepackage{siunitx}
\usepackage{float}
\usepackage{tikz}
\usepackage{url}
\usepackage[siunitx,american,RPvoltages]{circuitikz}
\ctikzset{capacitors/scale=0.7}
\ctikzset{diodes/scale=0.7}
\usepackage{tabularx}
\newcolumntype{C}{>{\centering\arraybackslash}X}
\renewcommand\tabularxcolumn[1]{m{#1}}% for vertical centering text in X column
\usepackage{tabu}
\usepackage[spanish,es-tabla,activeacute]{babel}
\usepackage{babelbib}
\usepackage{booktabs}
\usepackage{pgfplots}
\usepackage{hyperref}
\hypersetup{colorlinks = true,
            linkcolor = black,
            urlcolor  = blue,
            citecolor = blue,
            anchorcolor = blue}
\usepgfplotslibrary{units, fillbetween} 
\pgfplotsset{compat=1.16}
\usepackage{bm}
\usetikzlibrary{arrows, arrows.meta, shapes, 3d, perspective, positioning}
\renewcommand{\sin}{\sen} %change from sin to sen
\usepackage{bohr}
\setbohr{distribution-method = quantum,insert-missing = true}
\usepackage{elements}
\usepackage{verbatim}
\title{Electricidad I: \\ \emph{Resolución de circuitos RLC}\\ \emph{con transformada de Laplace}}
\author{
    Juan J. Rojas
}
\institute{Instituto Tecnológico de Costa Rica}
\date{\today}
\background{fig/background.jpg}

\begin{document}
\input{postamble}
\maketitle

\begin{frame}{Resistor en el dominio \emph{s}}
    \begin{columns}
        \begin{column}{0.5\textwidth}
            En el dominio del tiempo:
            \begin{equation*}
                v(t)=R \cdot i(t)
            \end{equation*}
            aplicando la transformada de Laplace:
            \begin{equation*}
                \mathcal{L}[v(t)]=\mathcal{L}[R\cdot i(t)]=R\cdot \mathcal{L}[i(t)]
            \end{equation*}
            obtenemos:
            \begin{equation*}
               V(s)=R\cdot I(s)
            \end{equation*}            
        \end{column}
        \begin{column}{0.5\textwidth}
        \centering
        Equivalente\\
            \begin{circuitikz}[scale=1]
                \draw
                (0,5)node[above]{$a$}
                    to[short,o-,i=$I(s)$]
                (0,4)
                    to[R,l=$R$]
                (0,1)
                    to[short,-o]
                (0,0)node[below]{$b$}
                (0,5)
                    to[open,v=$V(s)$,voltage shift=2]
                (0,0)
                ;
            \end{circuitikz}
        \end{column}
    \end{columns}
\end{frame}

\begin{frame}{Inductor en el dominio \emph{s}}
    \begin{columns}
        \begin{column}{0.5\textwidth}
            En el dominio del tiempo:
            \begin{equation*}
                v(t)=L \cdot \dfrac{di(t)}{dt}
            \end{equation*}
            aplicando la transformada de Laplace:
            \begin{equation*}
                \mathcal{L}[v(t)]=\mathcal{L}[L\cdot i'(t)] = L\cdot \mathcal{L}[i'(t)]
            \end{equation*}
            obtenemos:
            \begin{equation*}
                V(s)=L\cdot [sI(s)+I_0] = sL\cdot I(s) - LI_0
            \end{equation*}  
            \begin{equation*}
                I(s)=\dfrac{V(s)}{sL} + \dfrac{I_0}{s}
            \end{equation*} 
        \end{column}
        \begin{column}{0.30\textwidth}
        \centering
        Eq. en paralelo\\
            \begin{circuitikz}[scale=1]
                \draw
                (0,5)node[above]{$a$}
                    to[short,o-,i=$I(s)$]
                (0,4)
                    to[L,l=$sL$]
                (0,1)
                    to[short,-o]
                (0,0)node[below]{$b$}
                (0,4)
                    --
                (1.5,4)
                    to[I, l=$\dfrac{I_0}{s}$]
                (1.5,1)
                    --
                (0,1)
                (0,5)
                    to[open,v=$V(s)$,voltage shift=2]
                (0,0)
                ;
            \end{circuitikz}
        \end{column}
        \begin{column}{0.20\textwidth}
        \centering
        Eq. en serie\\
            \begin{circuitikz}[scale=1]
                \draw
                (0,5)node[above]{$a$}
                    to[short,o-,i=$I(s)$]
                (0,4)
                    to[L,l=$sL$]
                (0,2.5)
                    to[V, l=$LI_0$]
                (0,1)
                    to[short,-o]
                (0,0)node[below]{$b$}
                (0,5)
                    to[open,v=$V(s)$,voltage shift=2]
                (0,0)
                ;
            \end{circuitikz}
        \end{column}
    \end{columns}
\end{frame}

\begin{frame}{Capacitor en el dominio \emph{s}}
    \begin{columns}
        \begin{column}{0.5\textwidth}
            En el dominio del tiempo:
            \begin{equation*}
                i(t)=C \cdot \dfrac{dv(t)}{dt}
            \end{equation*}
            aplicando la transformada de Laplace:
            \begin{equation*}
                \mathcal{L}[i(t)]=\mathcal{L}[C\cdot v'(t)] = C\cdot \mathcal{L}[v'(t)]
            \end{equation*}
            obtenemos:
            \begin{equation*}
                I(s)=C\cdot [sV(s)+V_0] = sC\cdot V(s) - CV_0
            \end{equation*}  
            \begin{equation*}
                V(s)=\dfrac{I(s)}{sC} + \dfrac{V_0}{s}
            \end{equation*} 
        \end{column}
        \begin{column}{0.20\textwidth}
        \centering
        Eq. en serie\\
            \begin{circuitikz}[scale=1]
                \draw
                (0,5)node[above]{$a$}
                    to[short,o-,i=$I(s)$]
                (0,4)
                    to[C,l=$\dfrac{1}{sC}$]
                (0,2.5)
                    to[V, l=$\dfrac{V_0}{s}$,invert]
                (0,1)
                    to[short,-o]
                (0,0)node[below]{$b$}
                (0,5)
                    to[open,v=$V(s)$,voltage shift=2]
                (0,0)
                ;
            \end{circuitikz}
        \end{column}
        \begin{column}{0.30\textwidth}
        \centering
        Eq. en paralelo\\
            \begin{circuitikz}[scale=1]
                \draw
                (0,5)node[above]{$a$}
                    to[short,o-,i=$I(s)$]
                (0,4)
                    to[L,l=$\dfrac{1}{sC}$]
                (0,1)
                    to[short,-o]
                (0,0)node[below]{$b$}
                (0,4)
                    --
                (1.5,4)
                    to[I, l=$CV_0$,invert]
                (1.5,1)
                    --
                (0,1)
                (0,5)
                    to[open,v=$V(s)$,voltage shift=2]
                (0,0)
                ;
            \end{circuitikz}
        \end{column}
    \end{columns}
\end{frame}

\begin{frame}{Fuentes constantes en el dominio \emph{s}}
    \begin{columns}
        \begin{column}{0.5\textwidth}
            En el dominio del tiempo:
            \begin{align*}
               i(t) &= I_k & v(t) &= V_k
            \end{align*}
            aplicando la transformada de Laplace:
            \begin{align*}
               \mathcal{L}[i(t)]&=\mathcal{L}[I_k] & I(s)&= \dfrac{I_k}{s}
            \end{align*}
            \begin{align*}
               \mathcal{L}[v(t)]&=\mathcal{L}[V_k] & V(s)&= \dfrac{V_k}{s}
            \end{align*}
        \end{column}
        \begin{column}{0.25\textwidth}
        \centering
        Eq. voltaje\\
            \begin{circuitikz}[scale=1]
                \draw
                (0,5)node[above]{$a$}
                    to[short,o-,i<=$I(s)$]
                (0,4)
                    to[V, l=$\dfrac{V_k}{s}$,invert]
                (0,1)
                    to[short,-o]
                (0,0)node[below]{$b$}
                (0,5)
                    to[open,v=$V(s)$,voltage shift=2]
                (0,0)
                ;
            \end{circuitikz}
        \end{column}
        \begin{column}{0.25\textwidth}
        \centering
       Eq. corriente\\
            \begin{circuitikz}[scale=1]
                \draw
                (0,5)node[above]{$a$}
                    to[short,o-,i<=$I(s)$]
                (0,4)
                    to[I, l=$\dfrac{I_k}{s}$,invert]
                (0,1)
                    to[short,-o]
                (0,0)node[below]{$b$}
                (0,5)
                    to[open,v=$V(s)$,voltage shift=2]
                (0,0)
                ;
            \end{circuitikz}
        \end{column}
    \end{columns}
\end{frame}

\begin{frame}{Fuentes en escalón en el dominio \emph{s}}
    \begin{columns}
        \begin{column}{0.5\textwidth}
            En el dominio del tiempo:
            \begin{align*}
               i(t) &= I_k\cdot u(t-t_0) & v(t) &= V_k\cdot u(t-t_0)
            \end{align*}
            aplicando la transformada de Laplace:
            \begin{align*}
               \mathcal{L}[i(t)]&=\mathcal{L}[I_k\cdot u(t-t_0)] & I(s)&= I_k\cdot\dfrac{e^{-t_0s}}{s}
            \end{align*}
            \begin{align*}
               \mathcal{L}[v(t)]&=\mathcal{L}[V_k\cdot u(t-t_0)] & V(s)&= V_k \cdot\dfrac{e^{-t_0s}}{s}
            \end{align*}
        \end{column}
        \begin{column}{0.25\textwidth}
        \centering
        Eq. voltaje\\
            \begin{circuitikz}[scale=1]
                \draw
                (0,5)node[above]{$a$}
                    to[short,o-,i<=$I(s)$]
                (0,4)
                    to[V, l=$V_k \dfrac{e^{-t_0s}}{s}$,invert]
                (0,1)
                    to[short,-o]
                (0,0)node[below]{$b$}
                (0,5)
                    to[open,v=$V(s)$,voltage shift=2]
                (0,0)
                ;
            \end{circuitikz}
        \end{column}
        \begin{column}{0.25\textwidth}
        \centering
       Eq. corriente\\
            \begin{circuitikz}[scale=1]
                \draw
                (0,5)node[above]{$a$}
                    to[short,o-,i<=$I(s)$]
                (0,4)
                    to[I, l=$I_k\dfrac{e^{-t_0s}}{s}$,invert]
                (0,1)
                    to[short,-o]
                (0,0)node[below]{$b$}
                (0,5)
                    to[open,v=$V(s)$,voltage shift=2]
                (0,0)
                ;
            \end{circuitikz}
        \end{column}
    \end{columns}
\end{frame}

\begin{frame}{Circuito RC: respuesta a un escalón}
    \begin{columns}
    \begin{column}{0.5\textwidth}
        \centering
        \only<1>{
            \begin{circuitikz}[scale=0.8]
            \draw
            (0,0)
                to[V, l=$v_f(t)$]
            (0,4) 
                to[R, l=$R$]
            (5,4)
                to[C,l=$C$, v=$v(t)$,voltage shift=3]
            (5,0) 
                to[short]
            (0,0)
            ;
            \draw [thin, <-, >=triangle 45,blue] (2,2)node{$i(t)$}  ++(-60:0.7) arc (-60:170:0.7);
            \node [right] at (-2,-1) {$v_f(t)=V_0\cdot u(t_0-t)+V_\infty\cdot u(t-t_0)$};
            \draw [white](-2,-2) rectangle (7.2,6.2);
        \end{circuitikz}
        }
        \only<2->{
            \begin{circuitikz}[scale=0.8]
            \draw
            (0,0)
                to[V, l=$V_F(s)$]
            (0,4) 
                to[R, l=$R$]
            (5,4)
                to[C,l=$\dfrac{1}{sC}$]
            (5,2) 
                to[V, l=$\dfrac{V_0}{s}$,invert]
            (5,0)
                to[short]
            (0,0)
            (5,4)
                to[open, v=$V(s)$,voltage shift=3]
            (5,0)
            ;
            \draw [thin, <-, >=triangle 45,blue] (2,2)node{$I(s)$}  ++(-60:0.7) arc (-60:170:0.7);
            \node [right] at (-2,-1) {$V_F(s)=\dfrac{V_0}{s}\cdot (1-e^{-t_0s})+\dfrac{V_\infty}{s}\cdot e^{-t_0s}$};
            \draw [white](-2,-2) rectangle (7.2,6.2);
        \end{circuitikz}
        }
    \end{column}
    \begin{column}{0.5\textwidth}
        \only<1>
        {\begin{tikzpicture}[scale=1]
            \begin{axis}[%
                width= 0.9\linewidth,
                height= 0.5\linewidth,
                ymin=0, ymax=1.5,
                xmin=0, xmax=4,
                axis lines=center,
                xlabel=$t$,
                ylabel=$v_f$,
                x label style={anchor=west},
                y label style={anchor=south},
                ytick={0.5,1},
                yticklabels={$V_0$,$V_\infty$},
                ymax=1.5, % or enlarge y limits=upper
                xticklabels={$t_0$},
                xtick={1},
                ]
            \addplot+[no marks, thick] coordinates {(-1,0.5) (1,0.5) (1,1) (4,1)};
            \end{axis}
        \end{tikzpicture}\\
        se asume que ha pasado mucho tiempo antes de $t_0$ y por lo tanto el capacitor tiene un voltaje inicial
        \begin{equation*}
            v(t_0) = V_0
        \end{equation*}
        }\only<2>
            {Para resolver con Laplace:
            \begin{itemize}
                \item se transforman todos los elementos del circuito por sus equivalentes en el dominio $s$
                \item se incluyen las condiciones iniciales según convenga
            \end{itemize}
        }\only<3>
            {\onslide<3->
                {aplicando LVK tenemos que:
                \begin{multline*}
                    -\dfrac{V_o}{s}\cdot (1-e^{-t_0s}) - \dfrac{V_\infty}{s}\cdot e^{-t_0s} \\ 
                    + I(s)\left[R+\dfrac{1}{sC}\right] + \dfrac{V_0}{s} = 0
                \end{multline*}
            }
        }\only<4-5>
            {\onslide<4->
                {simplificando obtenemos: 
                \begin{equation*}
                    I(s) = \dfrac{V_\infty - V_o}{R}\cdot\dfrac{e^{-t_0s}}{s+1/\tau}
                \end{equation*}
            }\onslide<5->
                {aplicando la transformada inversa tenemos que:
                \begin{equation*}
                    i(t) = \dfrac{V_\infty - V_o}{R}\cdot e^{-(t-t_0)/\tau}\cdot u(t-t_0)
                \end{equation*}
                }
        }\only<6-8>
            {\onslide<6->
                {podemos calcular el voltaje pues:
                \begin{equation*}
                    V(s) = I(s)\cdot \dfrac{1}{sC} + \dfrac{V_0}{s}
                \end{equation*}
            }\onslide<7->
                {sustituyendo $I(s)$ tenemos que:
                \begin{equation*}
                    V(s) = \dfrac{V_\infty - V_o}{RC}\cdot\dfrac{e^{-t_0s}}{s(s+1/\tau)} + \dfrac{V_0}{s}
                \end{equation*}
            }\onslide<8->
                {aplicando la transformada inversa tenemos que:
                \begin{equation*}
                    v(t) = (V_\infty - V_o) \cdot (1 - e^{-(t-t_0)/\tau}) \cdot u(t-t_0) + V_0
                \end{equation*}
            }
        }\only<9-12>
            {\onslide<9->
            {retomando, tenemos
                \begin{equation*}
                    v(t) = (V_\infty - V_o) \cdot (1 - e^{-(t-t_0)/\tau}) \cdot u(t-t_0) + V_0
                \end{equation*}
            }\onslide<10->
                {si $t\leq t_0$ tenemos que $u(t-t_0) = 0$...
                \begin{equation*}
                     v(t) = V_0
                \end{equation*}
            }\onslide<11->
                {si $t\geq t_0$ tenemos que $u(t-t_0) = 1$...
                \begin{equation*}
                     v(t) = (V_\infty - V_o) \cdot (1 - e^{-(t-t_0)/\tau}) + V_0
                \end{equation*}
            }\onslide<12->
                {reordenando: 
                \begin{equation*}
                     v(t) = V_\infty + (V_o - V_\infty) \cdot e^{-(t-t_0)/\tau}
                \end{equation*}
            }
        }
    \end{column}
\end{columns}
\end{frame}

\end{document}