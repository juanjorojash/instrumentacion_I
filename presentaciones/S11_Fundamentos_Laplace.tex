\documentclass[aspectratio=169]{beamer}
\usetheme{Bruno}
\usepackage{amsmath}
\usepackage{amssymb}
\usepackage{siunitx}
\usepackage{float}
\usepackage{tikz}
\def\checkmark{\tikz\fill[scale=0.4](0,.35) -- (.25,0) -- (1,.7) -- (.25,.15) -- cycle;} 
\usepackage{url}
\usepackage[siunitx,american,RPvoltages]{circuitikz}
\ctikzset{capacitors/scale=0.7}
\ctikzset{diodes/scale=0.7}
\usepackage{tabularx}
\newcolumntype{C}{>{\centering\arraybackslash}X}
\renewcommand\tabularxcolumn[1]{m{#1}}% for vertical centering text in X column
\usepackage{tabu}
\usepackage[spanish,es-tabla,activeacute]{babel}
\usepackage{babelbib}
\usepackage{booktabs}
\usepackage{pgfplots}
\usepackage{hyperref}
\hypersetup{colorlinks = true,
            linkcolor = black,
            urlcolor  = blue,
            citecolor = blue,
            anchorcolor = blue}
\usepgfplotslibrary{units, fillbetween} 
\pgfplotsset{compat=1.16}
\usepackage{bm}
\usetikzlibrary{arrows, arrows.meta, shapes, 3d, perspective, positioning,mindmap,trees,backgrounds}
\renewcommand{\sin}{\sen} %change from sin to sen
\usepackage{bohr}
\setbohr{distribution-method = quantum,insert-missing = true}
\usepackage{elements}
\usepackage{verbatim}
\usepackage[edges]{forest}
\usepackage{etoolbox}
\usepackage{schemata}
\usepackage{appendix}
\usepackage{listings}

\definecolor{color_mate}{RGB}{255,255,128}
\definecolor{color_plas}{RGB}{255,128,255}
\definecolor{color_text}{RGB}{128,255,255}
\definecolor{color_petr}{RGB}{255,192,192}
\definecolor{color_made}{RGB}{192,255,192}
\definecolor{color_meta}{RGB}{192,192,255}
\newcommand\diagram[2]{\schema{\schemabox{#1}}{\schemabox{#2}}}

\definecolor{codegreen}{rgb}{0,0.6,0}
\definecolor{codegray}{rgb}{0.5,0.5,0.5}
\definecolor{codepurple}{rgb}{0.58,0,0.82}
\definecolor{backcolour}{rgb}{0.95,0.95,0.92}

\lstdefinestyle{mystyle}{
    backgroundcolor=\color{backcolour},   
    commentstyle=\color{codegreen},
    keywordstyle=\color{magenta},
    numberstyle=\tiny\color{codegray},
    stringstyle=\color{codepurple},
    basicstyle=\ttfamily\footnotesize,
    breakatwhitespace=false,         
    breaklines=true,                 
    captionpos=b,                    
    keepspaces=true,                 
    numbers=left,                    
    numbersep=5pt,                  
    showspaces=false,                
    showstringspaces=false,
    showtabs=false,                  
    tabsize=2
}

\lstset{style=mystyle}
\title{Electricidad I: \\ \emph{Fundamentos de la}\\ \emph{transformada de Laplace}}
\author{
    Juan J. Rojas
}
\institute{Instituto Tecnológico de Costa Rica}
\date{\today}
\background{fig/background.jpg}

\begin{document}
\sisetup{unit-math-rm=\mathrm,math-rm=\mathrm} % change sinitx font
\sisetup{output-decimal-marker = {,}}
\maketitle

\begin{frame}{Operadores y transformadas}
        \begin{itemize}
            \item<1-> Un operador toma una función como entrada y de salida da otra función.
            \item<2> Una transformada toma una función como entrada y de salida da otra función pero con una variable independiente diferente.
        \end{itemize}
       \begin{tikzpicture}
            \draw[white] (0,-2) rectangle (14,2);
            \draw<1->[thick, -latex] (0.5,0) -- (2,0)node[midway,above]{$f(t)$};
            \draw<1->[thick] (2,1) rectangle (5,-1)node[midway]{Operador};
            \draw<1->[thick, -latex] (5,0) -- (6.5,0)node[midway,above]{$g(t)$};
            \draw<2->[thick, -latex] (7.5,0) -- (9,0)node[midway,above]{$f(t)$};
            \draw<2->[thick] (9,1) rectangle (12,-1)node[midway]{Transformada};
            \draw<2->[thick, -latex] (12,0) -- (13.5,0)node[midway,above]{$F(s)$};
        \end{tikzpicture}
\end{frame}

\begin{frame}{Transformada de Laplace}
    \begin{columns}[onlytextwidth]
        \begin{column}{0.5\textwidth}
            \emph{La transformada de Laplace es una transformación integral de una función en el dominio del tiempo $f(t)$ a una función en el dominio de la frequencia compleja $F(s)$}
        \end{column}
        \begin{column}{0.5\textwidth}
            La transformada de Laplace se define como:
            \begin{equation*}
                \mathcal{L}[f(t)] = F(s) = \int_{0^{-}}^\infty f(t)e^{-st}\,dt
            \end{equation*}
            donde $s$ es la frecuencia compleja $s = \sigma + j\omega$\\[8pt]
            $\sigma$ es el coeficiente de amortiguamiento\\[8pt]
            $\omega$ es la frecuencia angular\\[8pt]
        \end{column}
    \end{columns}
\end{frame}

\begin{frame}{Propiedades útiles para este tema}
    \begin{tabularx}{\textwidth}{p{3cm}XX}
        \toprule
        Propiedad & Función & Transformada\\
        \midrule
        Definición & $f(t)$ & $F(s)$\\[12pt]
        Linealidad & $af(t)+bg(t)$ & $aF(s)+bG(s)$\\[12pt]
        Escalamiento & $f(at)$ & $\dfrac{1}{a}F\left(\dfrac{s}{a}\right)$\\[12pt]
        Desplazamiento & $f(t-a)u(t-a)$ & $e^{-as}F(s)$ \\[12pt]
        Diferenciación & $f'(t)$ & $sF(s)-f(0^-)$\\[12pt]
        & $f''(t)$ & $s^2F(s)-sf(0^-)-f'(0^-)$\\[12pt]
        \bottomrule
    \end{tabularx}
\end{frame}

\begin{frame}{Tabla de transformadas útiles para este tema}
    \centering
    \begin{tabularx}{0.6\textwidth}{XX}
        \toprule
        Función & Transformada\\
        \midrule
        $u(t)$ & $\dfrac{1}{s}$\\[12pt]
        $u(t-a)$ & $\dfrac{e^{-as}}{s}$\\[12pt]
        $e^{-at}$ & $\dfrac{1}{s+a}$\\[12pt]
        \bottomrule
    \end{tabularx}
\end{frame}

\begin{frame}{Ventajas de usar la transformada de Laplace}
\begin{itemize}
    \item Se puede obtener la respuesta completa del sistema en un solo paso
    \item No hay necesidad de integrar o derivar
\end{itemize}

\end{frame}

% \begin{frame}{Referencias}

% \bibliographystyle{ieeetr}

% \bibliography{referencias}

% \end{frame}

\end{document}