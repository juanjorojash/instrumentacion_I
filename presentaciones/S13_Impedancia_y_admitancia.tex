\documentclass[aspectratio=169]{beamer}
\usetheme{Bruno}
\usepackage{amsmath}
\usepackage{amssymb}
\usepackage{siunitx}
\usepackage{float}
\usepackage{tikz}
\usepackage{url}
\usepackage[siunitx,american,RPvoltages]{circuitikz}
\ctikzset{capacitors/scale=0.7}
\ctikzset{diodes/scale=0.7}
\usepackage{tabularx}
\newcolumntype{C}{>{\centering\arraybackslash}X}
\renewcommand\tabularxcolumn[1]{m{#1}}% for vertical centering text in X column
\usepackage{tabu}
\usepackage[spanish,es-tabla,activeacute]{babel}
\usepackage{babelbib}
\usepackage{booktabs}
\usepackage{pgfplots}
\usepackage{hyperref}
\hypersetup{colorlinks = true,
            linkcolor = black,
            urlcolor  = blue,
            citecolor = blue,
            anchorcolor = blue}
\usepgfplotslibrary{units, fillbetween} 
\pgfplotsset{compat=1.16}
\usepackage{bm}
\usetikzlibrary{arrows, arrows.meta, shapes, 3d, perspective, positioning}
\renewcommand{\sin}{\sen} %change from sin to sen
\usepackage{bohr}
\setbohr{distribution-method = quantum,insert-missing = true}
\usepackage{elements}
\usepackage{verbatim}
\title{Electricidad I: \\ \emph{Impedancia y admitancia}}
\author{
    Juan J. Rojas
}
\institute{Instituto Tecnológico de Costa Rica}
\date{\today}
\background{fig/background.jpg}

\begin{document}
\input{postamble}
\maketitle

\begin{frame}{Impendacia}
    \begin{columns}[onlytextwidth]
    \begin{column}{0.5\textwidth}
        \begin{center}
        Impedancia
        \begin{gather*}
            \bm{Z} = \frac{\bm{v}}{\bm{i}}\\[2pt]
            \bm{Z} = R+jX = |\bm{Z}\,|\angle\theta
        \end{gather*}
        \begin{align*}
            |\bm{Z}\,|&=\sqrt{R^2+X^2} & \theta &=\arctan\left(\frac{X}{R}\right)\\
            R &= |\bm{Z}\,|\cos\theta & X &= |\bm{Z}\,|\sin\theta
        \end{align*}
        \end{center}
        donde $R$ es la resistencia y $X$ es la reactancia, ambas en ohms ($\Omega$)
    \end{column}
    \begin{column}{0.5\textwidth}
        \begin{center}
        \begin{tikzpicture}[scale=1]
        \draw [white](-1,-0.5) rectangle (3.5,3.5);
        \draw [-latex] (0,0) -- (3,0)node[below]{\small $\mathrm{Re}$};
        \draw [-latex] (0,0) -- (0,2.5)node[left]{\small $\mathrm{Im}$};
        \draw [-latex,thick,black] (0,0) -- (2.2,1.5)node[midway,above]{$Z$};
        \draw [-latex,thick,black] (0,0) -- (2.2,0)node[midway,below]{$R$};
        \draw [-latex,thick,black] (2.2,0) -- (2.2,1.5)node[midway, right]{$X$};
        \draw [-latex] (0.8,0) arc (0:30:0.8)node[midway, right, yshift=0.1cm]{$\theta$};
        \end{tikzpicture}
        \end{center}
    \end{column}
    \end{columns}
\end{frame}

\begin{frame}{Admitancia}
    \begin{columns}[onlytextwidth]
    \begin{column}{0.5\textwidth}
        \begin{center}
        Admitancia
        \begin{gather*}
            \bm{Y} = \frac{\bm{i}}{\bm{v}}\\[2pt]
            \bm{Y} = G+jB = \frac{1}{|\bm{Z}\,|}\angle-\theta
        \end{gather*}
        \end{center}
        donde $G$ es la conductancia y $B$ es la susceptancia, ambas en siemens (S)
    \end{column}
    \begin{column}{0.5\textwidth}
        \begin{center}
        \begin{tikzpicture}[scale=1]
        \draw [white](-1,-2) rectangle (3.5,2);
        \draw [-latex] (0,0) -- (3,0)node[below]{\small $\mathrm{Re}$};
        \draw [-latex] (0,0) -- (0,1)node[left]{\small $\mathrm{Im}$};
        \draw [-latex,thick,black] (0,0) -- (2.2,-1.5)node[midway,below, yshift=-0.1cm]{$Y$};
        \draw [-latex,thick,black] (0,0) -- (2.2,0)node[midway,above]{$G$};
        \draw [-latex,thick,black] (2.2,0) -- (2.2,-1.5)node[midway, right]{$B$};
        \draw [-latex] (0.8,0) arc (0:-30:0.8)node[midway, right, yshift=-0.1cm]{$\theta$};
        \end{tikzpicture}
        \end{center}
    \end{column}
    \end{columns}
\end{frame}

\begin{frame}{Circuito resistivo}
    \vspace{0.2cm}
    \begin{columns}[onlytextwidth]
    \begin{column}{0.2\textwidth}
        \centering
        \begin{circuitikz}[scale=0.8]
        \draw
            (0,3)
                to[short, o-, i=$i$]
            (1.5,3)
                to[R,l=$R$,v=$v$, voltage shift=3]
            (1.5,0) 
                to[short, -o]    
            (0,0)
        ;
        \draw[white](-0.5,-0.5) rectangle (3,3.5);
        \end{circuitikz}
    \end{column}
    \begin{column}{0.4\textwidth}
        \centering
        \begin{gather*}
            i = I \sin{(\omega t+\phi)}\\[2pt]
            v = iR = RI \sin{(\omega t+\phi)}\\[4pt]
            \bm{i} = I \angle \phi\\[2pt]
            \bm{v} = R\bm{i} = RI \angle \phi
        \end{gather*}
    \end{column}
    \begin{column}{0.4\textwidth}
        \centering
        \begin{gather*}
            \bm{Z} = \frac{\bm{v}}{\bm{i}}\\[2pt]
            \bm{Z} = R+j0 = R\angle\SI{0}{\degree}
        \end{gather*}
    \end{column}
    \end{columns}
    \vskip0pt plus 1filll
    \centering
    \begin{tikzpicture}
    \begin{axis}[
        width= 0.7\linewidth,
        height= 0.3\linewidth,
        domain=0:2*pi,
        samples=100,
        ymin=-1.1, ymax=1.1,
        xmin=0, xmax=2*pi+0.2,
        axis x line=center,
        axis y line=left,
        ytick=\empty,
        yticklabels=\empty,
        xtick={pi/2,pi,1.5*pi,2*pi},
        xticklabels={\SI{90}{\degree},\SI{180}{\degree},\SI{270}{\degree},\SI{360}{\degree}},
        ticklabel style={below},
        %xlabel=$t$, % Set the labels
        %ylabel=amplitud,
        x unit=, 
        y unit=,
        % axis lines=center,
        x label style={anchor=west},
        y label style={anchor=south},
        ]
        \addplot[color=blue,mark=none] {sin(deg(x+pi/6))};
        \addplot[color=red,mark=none] {0.7*sin(deg(x+pi/6))};
        \coordinate (vC) at (pi/3,1);
        \coordinate (iC) at (pi/3,0.7);
    \end{axis}
    \draw [blue](vC) node[above]{$v$};
    \draw [red](iC) node[below]{$i$};
    \draw [white](-0.2,-0.2) rectangle (8.5,3);
    \end{tikzpicture}
    \begin{tikzpicture}[scale=1]
        \draw [white](-2,-0.5) rectangle (3,2.7);
        \draw [-latex] (0,0) -- (2.5,0)node[below]{\small $\mathrm{Re}$};
        \draw [-latex] (0,0) -- (0,2.5)node[left]{\small $\mathrm{Im}$};
        \draw [-latex,thick,red] (0,0) -- (30:1.5cm)node[above]{$\bm{i}$};
        \draw [-latex,thick,blue] (0,0) -- (30:2.5cm)node[above]{$\bm{v}$};
        \draw [-latex] (0.8,0) arc (0:30:0.8)node[midway, right, yshift=0.1cm]{$\phi$};
    \end{tikzpicture}
\end{frame}

\begin{frame}{Circuito inductivo}
    \vspace{0.2cm}
    \begin{columns}[onlytextwidth]
    \begin{column}{0.2\textwidth}
        \centering
        \begin{circuitikz}[scale=0.8]
        \draw
            (0,3)
                to[short, o-, i=$i$]
            (1.5,3)
                to[L,l=$L$,v=$v$, voltage shift=3]
            (1.5,0) 
                to[short, -o]    
            (0,0)
        ;
        \draw[white](-0.5,-0.5) rectangle (3,3.5);
        \end{circuitikz}
    \end{column}
    \begin{column}{0.4\textwidth}
        \centering
        \begin{gather*}
            i = I \sin{(\omega t+\phi)}\\
            v = L\frac{di}{dt} = \omega L I \sin{(\omega t+\phi+\SI{90}{\degree})}\\[4pt]
            \bm{i} = I \angle \phi\\[2pt]
            \bm{v} = j\omega L \bm{i} = \omega LI \angle \phi + \SI{90}{\degree}
        \end{gather*}
    \end{column}
    \begin{column}{0.4\textwidth}
        \centering
        \begin{gather*}
            \bm{Z} = \frac{\bm{v}}{\bm{i}}\\[2pt]
            \bm{Z} = 0+j\omega L = \omega L \angle\SI{90}{\degree}\\[2pt]
            X_L = \omega L
        \end{gather*}
    \end{column}
    \end{columns}
    \vskip0pt plus 1filll
    \centering
    \begin{tikzpicture}
    \begin{axis}[
        width= 0.7\linewidth,
        height= 0.3\linewidth,
        domain=0:2*pi,
        samples=100,
        ymin=-1.1, ymax=1.1,
        xmin=0, xmax=2*pi+0.2,
        axis x line=center,
        axis y line=left,
        ytick=\empty,
        yticklabels=\empty,
        xtick={pi/2,pi,1.5*pi,2*pi},
        xticklabels={\SI{90}{\degree},\SI{180}{\degree},\SI{270}{\degree},\SI{360}{\degree}},
        ticklabel style={below},
        %xlabel=$t$, % Set the labels
        %ylabel=amplitud,
        x unit=, 
        y unit=,
        % axis lines=center,
        x label style={anchor=west},
        y label style={anchor=south},
        ]
        \addplot[color=blue,mark=none] {sin(deg(x+pi/6+pi/2))};
        \addplot[color=red,mark=none] {0.7*sin(deg(x+pi/6))};
        \coordinate (vC) at (pi/3,0);
        \coordinate (iC) at (pi/3,0.7);
    \end{axis}
    \draw [blue](vC) node[below]{$v$};
    \draw [red](iC) node[below]{$i$};
    \draw [white](-0.2,-0.2) rectangle (8.5,3);
    \end{tikzpicture}
    \begin{tikzpicture}[scale=1]
        \draw [white](-2,-0.5) rectangle (3,2.7);
        \draw [-latex] (0,0) -- (2.5,0)node[below]{\small $\mathrm{Re}$};
        \draw [-latex] (0,0) -- (0,2.5)node[left]{\small $\mathrm{Im}$};
        \draw [-latex,thick,red] (0,0) -- (30:1.5cm)node[above]{$\bm{i}$};
        \draw [-latex,thick,blue] (0,0) -- (120:2.5cm)node[above]{$\bm{v}$};
        \draw [-latex] (0.8,0) arc (0:30:0.8)node[midway, right, yshift=0.1cm]{$\phi$};
        \draw [-latex] (0,0)+(30:0.8) arc (30:120:0.8)node[midway, above, xshift=0.1cm]{\SI{90}{\degree}};
    \end{tikzpicture}
\end{frame}

\begin{frame}{Circuito capacitivo}
    \vspace{0.2cm}
    \begin{columns}[onlytextwidth]
    \begin{column}{0.2\textwidth}
        \centering
        \begin{circuitikz}[scale=0.8]
        \draw
            (0,3)
                to[short, o-, i=$i$]
            (1.5,3)
                to[C,l=$C$,v=$v$, voltage shift=3]
            (1.5,0) 
                to[short, -o]    
            (0,0)
        ;
        \draw[white](-0.5,-0.5) rectangle (3,3.5);
        \end{circuitikz}
    \end{column}
    \begin{column}{0.4\textwidth}
        \centering
        \begin{gather*}
            v = V \sin{(\omega t+\phi)}\\
            i = C\frac{dv}{dt} = \omega C V\sin{(\omega t+\phi+\SI{90}{\degree})}\\[4pt]
            \bm{v} = V \angle \phi\\[2pt]
            \bm{i} = j\omega C \bm{v} = \omega CV \angle \phi + \SI{90}{\degree}
        \end{gather*}
    \end{column}
    \begin{column}{0.4\textwidth}
        \centering
        \begin{gather*}
            \bm{Z} = \frac{\bm{v}}{\bm{i}}\\[2pt]
            \bm{Z} = 0+\frac{1}{j\omega C} = \frac{1}{\omega C} \angle\SI{-90}{\degree}\\[2pt]
            X_C = \frac{1}{\omega C}
        \end{gather*}
    \end{column}
    \end{columns}
    \vskip0pt plus 1filll
    \centering
    \begin{tikzpicture}
    \begin{axis}[
        width= 0.7\linewidth,
        height= 0.3\linewidth,
        domain=0:2*pi,
        samples=100,
        ymin=-1.1, ymax=1.1,
        xmin=0, xmax=2*pi+0.2,
        axis x line=center,
        axis y line=left,
        ytick=\empty,
        yticklabels=\empty,
        xtick={pi/2,pi,1.5*pi,2*pi},
        xticklabels={\SI{90}{\degree},\SI{180}{\degree},\SI{270}{\degree},\SI{360}{\degree}},
        ticklabel style={below},
        %xlabel=$t$, % Set the labels
        %ylabel=amplitud,
        x unit=, 
        y unit=,
        % axis lines=center,
        x label style={anchor=west},
        y label style={anchor=south},
        ]
        \addplot[color=blue,mark=none] {sin(deg(x+pi/6))};
        \addplot[color=red,mark=none] {0.7*sin(deg(x+pi/6+pi/2))};
        \coordinate (vC) at (pi/3,1);
        \coordinate (iC) at (pi/3,0);
    \end{axis}
    \draw [blue](vC) node[below]{$v$};
    \draw [red](iC) node[above]{$i$};
    \draw [white](-0.2,-0.2) rectangle (8.5,3);
    \end{tikzpicture}
    \begin{tikzpicture}[scale=1]
        \draw [white](-2,-0.5) rectangle (3,2.7);
        \draw [-latex] (0,0) -- (2.5,0)node[below]{\small $\mathrm{Re}$};
        \draw [-latex] (0,0) -- (0,2.5)node[left]{\small $\mathrm{Im}$};
        \draw [-latex,thick,red] (0,0) -- (120:1.5cm)node[above]{$\bm{i}$};
        \draw [-latex,thick,blue] (0,0) -- (30:2.5cm)node[above]{$\bm{v}$};
        \draw [-latex] (0.8,0) arc (0:30:0.8)node[midway, right, yshift=0.1cm]{$\phi$};
        \draw [-latex] (0,0)+(30:0.8) arc (30:120:0.8)node[midway, above, xshift=0.1cm]{\SI{90}{\degree}};
    \end{tikzpicture}
\end{frame}

\begin{frame}{Impedancias en serie}
    \begin{columns}[onlytextwidth]
        \begin{column}{0.5\textwidth}
        Impedancia equivalente
            \begin{equation*}
                Z_{eq}=Z_1+Z_2+Z_3
            \end{equation*}
        Divisor de voltaje
            \begin{gather*}
                \bm{v_n} = \bm{v_T}\cdot \frac{Z_n}{Z_1+Z_2+\cdot\cdot\cdot+Z_n}\\[4pt]
                \bm{v_3} = \bm{v_T}\cdot \frac{Z_3}{Z_1+Z_2+Z_3}
            \end{gather*}
        \end{column}
        \begin{column}{0.5\textwidth}
        \centering
            \begin{circuitikz} [scale=1]\draw
                (3,0)
                    to[generic,l=$Z_3$,v>=$\bm{v_3}$,o-]
                (0,0)	
                    to[generic,l=$Z_2$]
                (0,3)
                    to[generic,l=$Z_1$,i^<=$\bm{i}$,-o]
                (3,3)
                    to[open,v^=$\bm{v_T}$]
                (3,0)
                ;
            \end{circuitikz}
        \end{column}
    \end{columns}
\end{frame}

\begin{frame}{Impedancias en paralelo}
    \begin{columns}[onlytextwidth]
        \begin{column}{0.5\textwidth}
        Impedancia equivalente
            \begin{equation*}
                \frac{1}{Z_{eq}}=\frac{1}{Z_1}+\frac{1}{Z_2}+\frac{1}{Z_3}
            \end{equation*}
        Divisor de corriente
            \begin{gather*}
                \bm{i_n} = \bm{i_T}\cdot \frac{\frac{1}{Z_n}}{\frac{1}{Z_1}+\frac{1}{Z_2}+\cdot\cdot\cdot+\frac{1}{Z_n}}\\[4pt]
                \bm{i_1} = \bm{i_T}\cdot \frac{\frac{1}{Z_1}}{\frac{1}{Z_1}+\frac{1}{Z_2}+\frac{1}{Z_3}}
            \end{gather*}
        \end{column}
        \begin{column}{0.5\textwidth}
            \begin{circuitikz} [scale=1]\draw
                (1,3)
                    to[short,i=$\bm{i_T}$,o-]
                (0,3)	
                    to[generic,l_=$Z_3$]
                (0,0)
                    to[short,-o]
                (1,0)
                (0,3) -- (-1.5,3)
                    to[generic,l_=$Z_2$]
                (-1.5,0) -- (0,0)
                (-1.5,3) -- (-3,3)
                    to[generic,l_=$Z_1$,i>^=$\bm{i_1}$]
                (-3,0) -- (-1.5,0)
                (1,3)
                    to[open,v^=$\bm{v}$]
                (1,0)
                ;
            \end{circuitikz}
        \end{column}
    \end{columns}
\end{frame}

\begin{frame}{Impedancias en paralelo: dos impedancias}
    \begin{columns}[onlytextwidth]
        \begin{column}{0.6\textwidth}
        Impedancia equivalente
            \begin{equation*}
                Z_{eq}=\frac{Z_1 \cdot Z_2}{Z_1+Z_2}
            \end{equation*}
        Divisor de corriente
            \begin{equation*}
            i_1 = i_T\cdot \frac{Z_2}{Z_1+Z_2}
            \end{equation*}
        \end{column}
        \begin{column}{0.4\textwidth}
            \begin{circuitikz} [scale=1]\draw
                (1,3)
                    to[short,i=$\bm{i_T}$,o-]
                (0,3)	
                    to[generic,l_=$Z_2$]
                (0,0)
                    to[short,-o]
                (1,0)
                (0,3) -- (-1.5,3)
                    to[generic,l_=$Z_1$,i>^=$\bm{i_1}$]
                (-1.5,0) -- (0,0)
                (1,3)
                    to[open,v^=$\bm{v}$]
                (1,0)
                ;
            \end{circuitikz}
        \end{column}
    \end{columns}
\end{frame}

% \begin{frame}{Referencias}

% \bibliographystyle{ieeetr}

% \bibliography{referencias}

% \end{frame}

\end{document}