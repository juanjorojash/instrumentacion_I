\documentclass[aspectratio=169]{beamer}
\usetheme{Bruno}
\usepackage{amsmath}
\usepackage{amssymb}
\usepackage{siunitx}
\usepackage{float}
\usepackage{tikz}
\usepackage{url}
\usepackage[siunitx,american,RPvoltages]{circuitikz}
\ctikzset{capacitors/scale=0.7}
\ctikzset{diodes/scale=0.7}
\usepackage{tabularx}
\newcolumntype{C}{>{\centering\arraybackslash}X}
\renewcommand\tabularxcolumn[1]{m{#1}}% for vertical centering text in X column
\usepackage{tabu}
\usepackage[spanish,es-tabla,activeacute]{babel}
\usepackage{babelbib}
\usepackage{booktabs}
\usepackage{pgfplots}
\usepackage{hyperref}
\hypersetup{colorlinks = true,
            linkcolor = black,
            urlcolor  = blue,
            citecolor = blue,
            anchorcolor = blue}
\usepgfplotslibrary{units, fillbetween} 
\pgfplotsset{compat=1.16}
\usepackage{bm}
\usetikzlibrary{arrows, arrows.meta, shapes, 3d, perspective, positioning}
\renewcommand{\sin}{\sen} %change from sin to sen
\usepackage{bohr}
\setbohr{distribution-method = quantum,insert-missing = true}
\usepackage{elements}
\usepackage{verbatim}
\title{Electricidad I: \\ \emph{Potencia y factor}\\ \emph{de potencia}}
\author{
    Juan J. Rojas
}
\institute{Instituto Tecnológico de Costa Rica}
\date{\today}
\background{fig/background.jpg}

\begin{document}
\input{postamble}
\maketitle

\begin{frame}{Valor rms de una onda senoidal}
El valor rms de una onda periodica esta definido por:\\[4pt]
    \begin{equation*}
        X_{rms} = \sqrt{\dfrac{1}{T}\int^T_0 x(t)^2dt}
    \end{equation*}
En el caso de las ondas senoidales periodicas sin componente de DC, su valor rms es igual al valor pico dividido entre la raiz cuadrada de dos\\[4pt]
Es común representar el voltaje y la corriente como fasores cuya magnitud es el valor rms en lugar del valor pico, de ahora en adelante se usarán de esta manera
    \begin{align*}
        \bm{v} &= \dfrac{Vp}{\sqrt{2}}\angle \theta_v & \bm{i} &= \dfrac{Ip}{\sqrt{2}}\angle \theta_i
    \end{align*}
\end{frame}

\begin{frame}{Potencias}
    \begin{columns}[onlytextwidth]
    \begin{column}{0.55\textwidth}
    Potencia compleja
        \begin{equation*}
            \bm{S} = \bm{v}\bm{i^*} = P + jQ = |\bm{v}||\bm{i}|\angle\theta_v - \theta_i\,\,[\mathrm{VA}]
        \end{equation*}
     Potencia aparente
        \begin{equation*}
            S = |\bm{S}| = |\bm{v}||\bm{i}| = \sqrt{P^2+Q^2}\,\,[\mathrm{VA}]
        \end{equation*}
    Potencia real
        \begin{equation*}
            P = \mathrm{Re}(S) = S\cos{(\theta_v - \theta_i)}\,\,[\mathrm{W}]
        \end{equation*}
    Potencia reactiva
        \begin{equation*}
            Q = \mathrm{Im}(S) = S\sin{(\theta_v - \theta_i)}\,\,[\mathrm{VAR}]
        \end{equation*}
    \end{column}
    \begin{column}{0.45\textwidth}
        \begin{center}
        \begin{tikzpicture}[scale=1.3]
        %\draw [white](-1,-0.5) rectangle (3.5,3.5);
        \draw [-latex] (0,0) -- (3,0)node[below]{\small $\mathrm{Re}$};
        \draw [-latex] (0,0) -- (0,2.5)node[left]{\small $\mathrm{Im}$};
        \draw [-latex,thick,black] (0,0) -- (2.2,1.5)node[midway,above]{$\bm{S}$};
        \draw [-latex,thick,black] (0,0) -- (2.2,-1.5)node[midway,below]{$\bm{S}$};
        \draw [-latex,thick,black] (0,0) -- (2.2,0)node[below, xshift=-0.2cm]{$P$};
        \draw [-latex,thick,black] (2.2,0) -- (2.2,1.5)node[midway, right]{$+Q$}node[midway, right, yshift=-0.3cm]{\tiny fp en atraso};;
        \draw [-latex,thick,black] (2.2,0) -- (2.2,-1.5)node[midway, right]{$-Q$}node[midway, right, yshift=-0.3cm]{\tiny fp en adelanto};         ;
        \draw [-latex] (0.8,0) arc (0:30:0.8)node[midway, right, yshift=0.1cm]{$\theta_v-\theta_i$};
        \draw [-latex] (0.8,0) arc (0:-30:0.8)node[midway, right, yshift=-0.1cm]{$\theta_v-\theta_i$};
        \end{tikzpicture}
        \end{center}
        \begin{equation*}
            \mathrm{fp} = \dfrac{P}{S} = \cos{(\theta_v - \theta_i)}
        \end{equation*}
    \end{column}
    \end{columns}
\end{frame}

\begin{frame}{Corrección del factor de potencia}
    \begin{columns}[onlytextwidth]
    \begin{column}{0.55\textwidth}
    Las compañías de distribución y generación de energía eléctrica usualmente cobran una multa cuando el factor de potencia es bajo
        \begin{equation*}
            \bm{S} = \bm{v}\bm{i^*} = P + jQ = |\bm{v}||\bm{i}|\angle\theta_v - \theta_i\,\,[\mathrm{VA}]
        \end{equation*}
    \end{column}
    \begin{column}{0.45\textwidth}
        \begin{center}
        \begin{tikzpicture}[scale=1.3]
        %\draw [white](-1,-0.5) rectangle (3.5,3.5);
        \draw [-latex] (0,0) -- (3,0)node[below]{\small $\mathrm{Re}$};
        \draw [-latex] (0,0) -- (0,2.5)node[left]{\small $\mathrm{Im}$};
        \draw [-latex,thick,black] (0,0) -- (2.2,1.5)node[midway,above]{$\bm{S}$};
        \draw [-latex,thick,black] (0,0) -- (2.2,-1.5)node[midway,below]{$\bm{S}$};
        \draw [-latex,thick,black] (0,0) -- (2.2,0)node[below, xshift=-0.2cm]{$P$};
        \draw [-latex,thick,black] (2.2,0) -- (2.2,1.5)node[midway, right]{$+Q$}node[midway, right, yshift=-0.3cm]{\tiny fp en atraso};;
        \draw [-latex,thick,black] (2.2,0) -- (2.2,-1.5)node[midway, right]{$-Q$}node[midway, right, yshift=-0.3cm]{\tiny fp en adelanto};         ;
        \draw [-latex] (0.8,0) arc (0:30:0.8)node[midway, right, yshift=0.1cm]{$\theta_v-\theta_i$};
        \draw [-latex] (0.8,0) arc (0:-30:0.8)node[midway, right, yshift=-0.1cm]{$\theta_v-\theta_i$};
        \end{tikzpicture}
        \end{center}
        \begin{equation*}
            \mathrm{fp} = \dfrac{P}{S} = \cos{(\theta_v - \theta_i)}
        \end{equation*}
    \end{column}
    \end{columns}
\end{frame}


% \begin{frame}{Referencias}

% \bibliographystyle{ieeetr}

% \bibliography{referencias}

% \end{frame}

\end{document}