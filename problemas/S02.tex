\documentclass[12pt,letterpaper]{article}
\usepackage{amsmath}
\usepackage{amssymb}
\usepackage{siunitx}
\usepackage{float}
\usepackage{tikz}
\usepackage{url}
\usepackage[siunitx,american,RPvoltages]{circuitikz}
\ctikzset{capacitors/scale=0.7}
\ctikzset{diodes/scale=0.7}
\usepackage{tabularx}
\newcolumntype{C}{>{\centering\arraybackslash}X}
\renewcommand\tabularxcolumn[1]{m{#1}}% for vertical centering text in X column
\usepackage{tabu}
\usepackage[spanish,es-tabla,activeacute]{babel}
\usepackage{babelbib}
\usepackage{booktabs}
\usepackage{pgfplots}
\usepackage{hyperref}
\hypersetup{colorlinks = true,
            linkcolor = black,
            urlcolor  = blue,
            citecolor = blue,
            anchorcolor = blue}
\usepgfplotslibrary{units, fillbetween} 
\pgfplotsset{compat=1.16}
\usepackage{bm}
\usetikzlibrary{arrows, arrows.meta, shapes, 3d, perspective, positioning}
\renewcommand{\sin}{\sen} %change from sin to sen
\usepackage{bohr}
\setbohr{distribution-method = quantum,insert-missing = true}
\usepackage{elements}
\usepackage{verbatim}
\usepackage{geometry}  
\geometry{left=18mm,right=18mm,top=21mm,bottom=21mm,headheight=15pt} % Tamaño del área de escritura de la página
\setlength\parindent{0pt} % Removes all indentation from paragraphs
\usepackage{fancyhdr}
\pagestyle{fancy}
\lhead{Problemas de apoyo: Electricidad I}
\rhead{\begin{picture}(0,0) \put(-110,0){\includegraphics[width=40mm]{fig/logo.png}} \end{picture}}
\usepackage{titlesec}

\titleformat{\section}{\normalfont \Large \bfseries}
{Semana\ \thesection}{2.3ex plus .2ex}{} %% Text "Semana" is what you would like the section number to display with.
%\titlespacing{\subsection}{2em}{*1}{*1}

\titleformat{\subsection}[runin]{\normalfont \large \bfseries}
{Problema\ \thesubsection}{0pt}{} %% Text "Problema" is what you would like the section number to display with.

\newcommand{\asection}[2]{
\setcounter{section}{#1}
\addtocounter{section}{-1}
\section{#2}
}


\begin{document}
\input{comunes/postamble}

\asection{3}{Leyes básicas: Problemas complementarios}
1. Para el siguiente circuito, encuentre $v_1$, $i_1$, $v_2$, $i_2$ e $i_3$. 

\begin{center}
    \begin{circuitikz}
        \draw
        (0,0)
            to[V,l=\SI{20}{\volt}]
        (0,4)
            to[R,l=\SI{2}{\ohm},i=$i_1$,v=$v_1$]
        (3,4)
            to[short]
        (9,4)
            to[R,l=\SI{15}{\ohm},i=$i_2$,v=$v_2$]
        (9,0)
            to[short]
        (0,0)
        (3,4)
            to[R,l=\SI{5}{\ohm},i=$i_3$]
        (3,0)
        (5,4)
            to[R,l=\SI{3}{\ohm}]
        (5,2)
            to[R,l=\SI{10}{\ohm}]
        (5,0)
        (7,4)
            to[R,l=\SI{5}{\ohm}]
        (7,0)
        ;
    \end{circuitikz}
\end{center}

R/ $v_1=\SI{10.42}{\volt}$, $i_1=\SI{5.21}{\ampere}$, $v_2=\SI{9.58}{\volt}$, $i_2=\SI{0.64}{\ampere}$, $i_3=\SI{1.92}{\ampere}$\\[8pt]

\begin{tabularx}{\linewidth}{@{}p{0.48\linewidth} @{}p{0.04\linewidth} @{}p{0.48\linewidth}}

2. Para el siguiente circuito, encuentre $v$ e $i$. 

\begin{center}
    \begin{circuitikz}
        \draw
        (1,4)node[right]{$+$\SI{21}{\volt}}   
            to[short,o-]
        (0,4)
            to[R,l=\SI{7}{\ohm}]
        (0,2)
            to[R,l=\SI{4}{\ohm},i=$i$]
        (0,0)node[ground]{}
        (0,2)
            to[short,-o]
        (-1,2)node[left]{$v$}
        ;
    \end{circuitikz}
\end{center}
R/ $v=\SI{7.64}{\volt}$, $i_1=\SI{1.91}{\ampere}$

&
&

3. Para el siguiente circuito, encuentre $i$, $v_0$ e $i_0$. 

\begin{center}
    \begin{circuitikz}
        \draw
        (0,0)
            to[V,l=\SI{10}{\volt}]
        (0,3)
            to[R,l=\SI{7}{\ohm},i=$i$]
        (3,3)
            to[short]
        (5,3)
            to[R,l=\SI{1.5}{\ohm},i=$i_0$,v=$v_0$]
        (5,0)
            to[short]
        (0,0)
        (3,3)
            to[R,l=\SI{2}{\ohm}]
        (3,0)
        ;
    \end{circuitikz}
\end{center}
R/ $i=\SI{1.27}{\ampere}$, $v_0=\SI{1.09}{\volt}$ e $i_0=\SI{0.73}{\ampere}$
\\
\end{tabularx}

4. Para el siguiente circuito, encuentre $R_{eq}$, $i_1$, $i_2$, $i_3$ y $v_0$

\begin{center}
    \begin{circuitikz}
        \draw
        (0,0)
            to[I,l=\SI{3}{\ampere}]
        (0,3)
            to[short]
        (4,3)
            to[R,l=\SI{9}{\ohm},v=$v_0$]
        (7,3)
            to[short]
        (9,3)
           to[R,l=\SI{75}{\ohm},i=$i_3$]
        (9,0)
            to[short]
        (0,0)
        (2,3)
           to[R,l=\SI{10}{\ohm},i=$i_1$]
        (2,0)
        (4,3)
            to[R,l=\SI{7}{\ohm},i=$i_2$]
        (4,0)
        (7,3)
            to[R,l=\SI{15}{\ohm}]
        (7,0)
        ;
        \draw[thick] (4.6,-0.5)node[right]{$R_{eq}$}-- (4.6,0.5);
        \draw[thick,-latex] (4.6,0.5) -- (5.2,0.5);
    \end{circuitikz}
\end{center}

R/ $R_{eq}=\SI{21.5}{\ohm}$ $i_1=\SI{1.04}{\ampere}$, $i_2=\SI{1.48}{\ampere}$, $i_3=\SI{80}{\milli\ampere}$ y $v_0=\SI{4.32}{\volt}$ 



5. Para el siguiente circuito, encuentre $v_0$. 

\begin{center}
    \begin{circuitikz}
        \draw
        (3,4)   
            to[short]
        (0,4)
            to[V,l=\SI{10}{\volt}]
        (0,0)node[ground]{}
        (3,4)
            to[R,l=\SI{1}{\kilo\ohm}]
        (3,2)node[left]{$v_0$}
            to[R,l=\SI{1}{\kilo\ohm}]
        (3,0)node[ground]{}
        (3,2)
            to[short]
        (6,2)
            to[R,l=\SI{10}{\kilo\ohm}]
        (6,0)node[ground]{}

        ;
    \end{circuitikz}
\end{center}

R/$v_0=\SI{5.714}{\volt}$ \\[8pt]  


6. Para el siguiente circuito, encuentre $i_1$, e $i_2$. 

\begin{center}
    \begin{circuitikz}
        \draw
        (0,0)
            to[I,l=\SI{2}{\ampere}]
        (0,3)
            to[short]
        (3,3)
            to[short]
        (5,3)
            to[R,l=\SI{15}{\ohm},i=$i_2$]
        (5,0)
            to[short]
        (0,0)
        (3,3)
            to[R,l=\SI{10}{\ohm},i=$i_1$]
        (3,0)
        ;
    \end{circuitikz}
\end{center}
R/ $i_1=\SI{1.2}{\ampere}$ e $i_2=\SI{0.8}{\ampere}$\\[8pt]

% 8.Encuentre $v_1$ y $v_2$
% \begin{center}
%     \begin{circuitikz}
%         \draw
%         (0,0)
%           to[R,l=\SI{2}{\ohm}]
%         (0,3.5)
%             to[short]
%         (0,6)
%         (-0.1,3.7)node[left]{$v_1$}
%         (0,3.5)
%             to[short]
%         (-3.5,3.5)
%             to[I,l=\SI{2}{\ampere}]
%         (-3.5,0)
%             to[short]
%         (0,0)
%         (0,6)
%           to[R,l=\SI{10}{\ohm}]
%         (4,6)
%             to[short]
%         (4,3.5) 
        
%         (7,3.5)
%           to[I,l=\SI{7}{\ampere}]
%         (7,0)
%             to[short]
%         (0,0)
        
%         (0,3.5)
%             to[V,l=\SI{2}{\volt}]
%         (4,3.5)
%         (4.1,3.7)node[right]{$v_2$}
%         (4,3.5)
%             to[R,l=\SI{4}{\ohm}]
%         (4,0)
%         (4,3.5)
%             to[short] 
%         (7,3.5)
%         ;
%     \end{circuitikz}
% \end{center}
% R/
% $v_1=\SI{-7.333}{\volt}$ \\[8pt]

% 9. Encontrar $v_1$, $v_2$ y $v_3$.
% \begin{center}
%     \begin{circuitikz}
%         \draw
%         (0,0)
%           to[R,l=\SI{2}{\ohm}]
%         (0,2) 
%             to[short]
%         (0,6)
%             to[R,l=\SI{6}{\ohm}]
%         (8,6)
%             to[short]
%         (8,2)
%             to[R,l=\SI{3}{\ohm}]
%         (8,0)
%             to[short]
%         (0,0)
%         (0,3.5)node[left]{$v_1$}
%           to[V,l=\SI{25}{\volt},invert]
%         (4,3.5)
%         (4,3.7)node[above]{$v_2$}
%         (4,3.5) 
%             to[R,l=\SI{4}{\ohm}]
%         (4,0)
%         (4,3.5)
%             to[cV,l=$5i$] 
%         (8,3.5)node[right]{$v_3$}
%         ;
        
%     \end{circuitikz}
% \end{center}
% R/
% $v_1=\SI{7.608}{\volt}$, $v_2=\SI{17.39}{\volt}$ y $v_3=\SI{1.6305}{\volt}$

% \newpage
% 10. Encontrar Vo

% \begin{center}
%     \begin{circuitikz}
%         \draw
%         (0,0)
%             to[R,l=\SI{4}{\ohm},v=$V_o$]
%         (0,5)
%             to[R,l=\SI{6}{\ohm},-*]
%         (3,5)
%             to[R,l=\SI{20}{\ohm}]
%         (6,5)
%             to[cV,l=$5v_o$,invert]
%         (6,0)
%             to[short]
%         (3,0)
%             to[short]
%         (0,0)
%         (3,5)
%             to[V,l=\SI{60}{\volt}, invert]
%         (3,2.5)
%             to[R,l=\SI{20}{\ohm}]
%         (3,0)node[ground]{}
%         ;
%         \end{circuitikz}
% \end{center}

% R/ 
% $v_o=\SI{12}{\volt}$

\end{document}