\documentclass[12pt,letterpaper]{article}
\usepackage{amsmath}
\usepackage{amssymb}
\usepackage{siunitx}
\usepackage{float}
\usepackage{tikz}
\usepackage{url}
\usepackage[siunitx,american,RPvoltages]{circuitikz}
\ctikzset{capacitors/scale=0.7}
\ctikzset{diodes/scale=0.7}
\usepackage{tabularx}
\newcolumntype{C}{>{\centering\arraybackslash}X}
\renewcommand\tabularxcolumn[1]{m{#1}}% for vertical centering text in X column
\usepackage{tabu}
\usepackage[spanish,es-tabla,activeacute]{babel}
\usepackage{babelbib}
\usepackage{booktabs}
\usepackage{pgfplots}
\usepackage{hyperref}
\hypersetup{colorlinks = true,
            linkcolor = black,
            urlcolor  = blue,
            citecolor = blue,
            anchorcolor = blue}
\usepgfplotslibrary{units, fillbetween} 
\pgfplotsset{compat=1.16}
\usepackage{bm}
\usetikzlibrary{arrows, arrows.meta, shapes, 3d, perspective, positioning}
\renewcommand{\sin}{\sen} %change from sin to sen
\usepackage{bohr}
\setbohr{distribution-method = quantum,insert-missing = true}
\usepackage{elements}
\usepackage{verbatim}
\usepackage{geometry}  
\geometry{left=18mm,right=18mm,top=21mm,bottom=21mm,headheight=15pt} % Tamaño del área de escritura de la página
\setlength\parindent{0pt} % Removes all indentation from paragraphs
\usepackage{fancyhdr}
\pagestyle{fancy}
\lhead{Problemas de apoyo: Electricidad I}
\rhead{\begin{picture}(0,0) \put(-110,0){\includegraphics[width=40mm]{fig/logo.png}} \end{picture}}
\usepackage{titlesec}

\titleformat{\section}{\normalfont \Large \bfseries}
{Semana\ \thesection}{2.3ex plus .2ex}{} %% Text "Semana" is what you would like the section number to display with.
%\titlespacing{\subsection}{2em}{*1}{*1}

\titleformat{\subsection}[runin]{\normalfont \large \bfseries}
{Problema\ \thesubsection}{0pt}{} %% Text "Problema" is what you would like the section number to display with.

\newcommand{\asection}[2]{
\setcounter{section}{#1}
\addtocounter{section}{-1}
\section{#2}
}


\begin{document}
\input{comunes/postamble}

\asection{4}{Nodos, Supernodos, Mallas y Supermallas}

1.Encuentre $v_1$ y $v_2$
 \begin{center}
     \begin{circuitikz}
         \draw
         (0,0)
           to[R,l=\SI{2}{\ohm}]
         (0,3.5)
             to[short]
         (0,6)
         (-0.1,3.7)node[left]{$v_1$}
         (0,3.5)
             to[short]
         (-3.5,3.5)
             to[I,l=\SI{2}{\ampere}]
         (-3.5,0)
             to[short]
         (0,0)
         (0,6)
           to[R,l=\SI{10}{\ohm}]
         (4,6)
             to[short]
         (4,3.5) 
        
         (7,3.5)
           to[I,l=\SI{7}{\ampere}]
         (7,0)
             to[short]
         (0,0)
        
         (0,3.5)
             to[V,l=\SI{2}{\volt}]
         (4,3.5)
         (4.1,3.7)node[right]{$v_2$}
         (4,3.5)
             to[R,l=\SI{4}{\ohm}]
         (4,0)
         (4,3.5)
             to[short] 
         (7,3.5)
         ;
     \end{circuitikz}
 \end{center}
 R/
 $v_1=\SI{-7.333}{\volt}$ \\[8pt]

 2. Encontrar $v_1$, $v_2$ y $v_3$.
 \begin{center}
     \begin{circuitikz}
         \draw
         (0,0)
           to[R,l=\SI{2}{\ohm}]
         (0,2) 
             to[short]
         (0,6)
             to[R,l=\SI{6}{\ohm}]
         (8,6)
             to[short]
         (8,2)
             to[R,l=\SI{3}{\ohm}]
         (8,0)
             to[short]
         (0,0)
         (0,3.5)node[left]{$v_1$}
           to[V,l=\SI{25}{\volt},invert]
         (4,3.5)
         (4,3.7)node[above]{$v_2$}
         (4,3.5) 
             to[R,l=\SI{4}{\ohm}]
         (4,0)
         (4,3.5)
             to[cV,l=$5i$] 
         (8,3.5)node[right]{$v_3$}
         ;
        
     \end{circuitikz}
 \end{center}
 R/
 $v_1=\SI{7.608}{\volt}$, $v_2=\SI{17.39}{\volt}$ y $v_3=\SI{1.6305}{\volt}$

 \newpage
 3. Encontrar Vo

 \begin{center}
     \begin{circuitikz}
         \draw
         (0,0)
             to[R,l=\SI{4}{\ohm},v=$V_o$]
         (0,5)
             to[R,l=\SI{6}{\ohm},-*]
         (3,5)
             to[R,l=\SI{20}{\ohm}]
         (6,5)
             to[cV,l=$5v_o$,invert]
         (6,0)
             to[short]
         (3,0)
             to[short]
         (0,0)
         (3,5)
             to[V,l=\SI{60}{\volt}, invert]
         (3,2.5)
             to[R,l=\SI{20}{\ohm}]
         (3,0)node[ground]{}
         ;
         \end{circuitikz}
 \end{center}

R/ 
$v_o=\SI{12}{\volt}$ \\[8pt]

4. Encontrar $v_1$, $v_2$ y $v_3$.
 \begin{center}
     \begin{circuitikz}
         \draw
         (0,0)
           to[R,l=\SI{1}{\ohm}]
         (0,3.5)
            to[short]
         (-3.5,3.5)
            to[I,l=\SI{2}{\ampere},invert]
         (-3.5,0)
            to[short]
         (0,0)
         (0,3)
             to[short]
         (0,6)
             to[R,l=\SI{2}{\ohm}]
         (8,6)
             to[short]
         (8,3.5)
             to[V,l=\SI{13}{\volt}, invert]
         (8,0)
             to[short]
         (0,0)
         (-0.2,3.7)node[left]{$v_1$}
         (0,3.5)
           to[cV,l=$2Vo$]
         (4,3.5)
         (4,3.7)node[above]{$v_2$}
         (4,3.5) 
             to[R,l=\SI{4}{\ohm},v=$V_o$]
         (4,0)
         (4,3.5)
             to[R,l=\SI{8}{\ohm}]
         (8,3.5)node[right]{$v_3$}
         ;
        
     \end{circuitikz}
 \end{center}
R/
 $v_1=\SI{18.86}{\volt}$, $v_2=\SI{6.29}{\volt}$ y $v_3=\SI{13}{\volt}$ 
 \newpage
 
5. Encontrar $V_o$ e $i_o$.
 \begin{center}
     \begin{circuitikz}
         \draw
         (0,0)
           to[short]
         (0,3.5)
            to[R,l=\SI{10}{\ohm},invert]
         (-3.5,3.5)
            to[V,l=\SI{80}{\volt}, invert]
         (-3.5,0)
            to[short]
         (0,0)
         (0,3)
             to[short,i=$i_o$]
         (0,6)
             to[R,l=\SI{40}{\ohm}]
         (4,6)
            to[V,l=\SI{96}{\volt}]
         (8,6)
             to[short]
         (8,3.5)
            to[short]
         (10,3.5)
            to[R,l=\SI{80}{\ohm},v=$V_o$]
         (10,0)
            to[short]
         (8,0)
         (8,3.5)
             to[short]
         (8,0)
             to[short]
         (0,0)
         (0,3.5)
           to[R,l=\SI{29}{\ohm}]
         (3,3.5) 
            to[cV,l=$4Vo$, invert]
         (3,0)
         (5,3.5) 
             to[cI,l=$2io$, invert]
         (5,0)
         (5,3.5)
              to[short]
         (8,3.5)
         ;
        
     \end{circuitikz}
 \end{center}
 R/
 $V_o=\SI{-1070}{\volt}$,$i_o=\SI{-4.48}{\ampere}$\\[8pt]
 
 6. Hallar Vx e ix en el circuito
  \begin{center}
     \begin{circuitikz}
         \draw
         (-2,0)
             to[V,l=\SI{50}{\volt},i=$i_x$]
         (-2,5)
             to[R,l=\SI{10}{\ohm}]
         (0,5)
            to[short]
         (4,5)
             to[short]
         (6,5)
            to[R,l=\SI{5}{\ohm}]
         (6,2)
            to[cV,l=$4i_x$,invert]
         (6,0)
             to[short]
         (3,0)
             to[short]
         (-2,0)
         (0,5)
             to[I,l=\SI{3}{\ampere}, invert]
         (0,3)
            to[short]
         (2,3)
            to[short]
         (4,3)
            to[cI,l=$1/4Vx$]
         (4,5)
         (2,3)
            to[R,l=\SI{2}{\ohm}]
         (2,0)node[ground]{}
         ;
         \end{circuitikz}
 \end{center}
 R/
 $V_o=\SI{-4}{\volt}$,$i_o=\SI{2.105}{\ampere}$\\[8pt]
 
7.Calcule Vo e io en el circuito.
 \begin{center}
     \begin{circuitikz}
         \draw
         (-3,3.5)
            to[V,l=\SI{0.015}{\volt}, invert]
         (-3,0)
            to[short]
         (0,0)
         (8,3.5)
            to[short,-o]
         (10,3.5)
         (8,0) 
            to[short,-o]
         (10,0)
            to[open,v=$V_o$,invert]
         (10,3.5)
         (8,3.5)
             to[R,l=\SI{20}{\kilo\ohm},invert]
         (8,0)
             to[short]
         (0,0)
         (-3,3.5)
           to[R,l=\SI{4}{\kilo\ohm},i<^=$i_o$]
         (0,3.5) 
            to[cV,l=$1/100Vo$, invert]
         (0,0)
         (5,3.5) 
             to[cI,l=$50io$, invert]
         (5,0)
         (5,3.5)
              to[short]
         (8,3.5)
         ;
        
     \end{circuitikz}
 \end{center}
R/
 $V_o=\SI{2.5}{\volt}$,$i_o=\SI{2.5}{\mu\ampere}$\\[8pt]
 
 8. Para el siguiente circuito encuentre $i_1$ e $i_2$, $v_1$ y $v_2$. 


\begin{center}
    \begin{circuitikz}
        \draw
        (0,0)
          to[V,l=\SI{30}{\volt}]
        (0,3.5)
            to[short]
        (0,6)
            to[R,l=\SI{12}{\ohm},i=$i_1$, v=$v_1$ ]
        (4,6)
            to[short]
        (4,3.5) 
        
        (7,3.5)
            to[R,l=\SI{40}{\ohm},i=$i_2$,v=$v_2$]
        (7,0)
            to[short]
        (0,0)
        
        (0,3.5)
            to[R,l=\SI{6}{\ohm}]
        (4,3.5)
            to[R,l=\SI{10}{\ohm}]
        (4,0)
        (4,3.5)
            to[short] 
        (7,3.5)
        ;
    \end{circuitikz}
\end{center}

R/$i_1=\SI{833.3}{\milli\ampere}$, $i_2=\SI{500}{\milli\ampere}$, $v_1=\SI{10}{\volt}$, $v_2=\SI{20}{\volt}$
\\[8pt]
 
\end{document}