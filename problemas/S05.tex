\documentclass[12pt,letterpaper]{article}
\usepackage{amsmath}
\usepackage{amssymb}
\usepackage{siunitx}
\usepackage{float}
\usepackage{tikz}
\usepackage{url}
\usepackage[siunitx,american,RPvoltages]{circuitikz}
\ctikzset{capacitors/scale=0.7}
\ctikzset{diodes/scale=0.7}
\usepackage{tabularx}
\newcolumntype{C}{>{\centering\arraybackslash}X}
\renewcommand\tabularxcolumn[1]{m{#1}}% for vertical centering text in X column
\usepackage{tabu}
\usepackage[spanish,es-tabla,activeacute]{babel}
\usepackage{babelbib}
\usepackage{booktabs}
\usepackage{pgfplots}
\usepackage{hyperref}
\hypersetup{colorlinks = true,
            linkcolor = black,
            urlcolor  = blue,
            citecolor = blue,
            anchorcolor = blue}
\usepgfplotslibrary{units, fillbetween} 
\pgfplotsset{compat=1.16}
\usepackage{bm}
\usetikzlibrary{arrows, arrows.meta, shapes, 3d, perspective, positioning}
\renewcommand{\sin}{\sen} %change from sin to sen
\usepackage{bohr}
\setbohr{distribution-method = quantum,insert-missing = true}
\usepackage{elements}
\usepackage{verbatim}
\usepackage{geometry}  
\geometry{left=18mm,right=18mm,top=21mm,bottom=21mm,headheight=15pt} % Tamaño del área de escritura de la página
\setlength\parindent{0pt} % Removes all indentation from paragraphs
\usepackage{fancyhdr}
\pagestyle{fancy}
\lhead{Problemas de apoyo: Electricidad I}
\rhead{\begin{picture}(0,0) \put(-110,0){\includegraphics[width=40mm]{fig/logo.png}} \end{picture}}
\usepackage{titlesec}

\titleformat{\section}{\normalfont \Large \bfseries}
{Semana\ \thesection}{2.3ex plus .2ex}{} %% Text "Semana" is what you would like the section number to display with.
%\titlespacing{\subsection}{2em}{*1}{*1}

\titleformat{\subsection}[runin]{\normalfont \large \bfseries}
{Problema\ \thesubsection}{0pt}{} %% Text "Problema" is what you would like the section number to display with.

\newcommand{\asection}[2]{
\setcounter{section}{#1}
\addtocounter{section}{-1}
\section{#2}
}


\begin{document}
\input{comunes/postamble}

\asection{5}{Transformación de fuentes y Superposición}

1. Para el siguiente circuito, encuentre $i_1$, $i_2$ e $i_3$.

\begin{center}
    \begin{circuitikz}
        \draw
        (0,0)
          to[R,l=\SI{5}{\ohm}, i=$i_2$]
        (0,2)
            to[short]
        (0,6)
            to[R,l=\SI{7}{\ohm},i=$i_1$ ]
        (8,6)
            to[short]
        (8,2)
            to[R,l=\SI{3}{\ohm}, i=$i_3$]
        (8,0)
            to[short]
        (0,0)
        
        (0,3.5)
            to[R,l=\SI{2}{\ohm}]
        (4,3.5)
            to[R,l=\SI{1}{\ohm}]
        (4,2)
            to[V,l=\SI{8}{\volt},invert]
        (4,0)
        (4,3.5)
            to[V,l=\SI{5}{\volt},invert] 
        (8,3.5)
      ;
    \end{circuitikz}
\end{center}

R/  $i_1=\SI{0.369}{\ampere}$,
$i_2=\SI{-0.840}{\ampere}$
$i_3=\SI{0.540}{\ampere}$\\[8pt]

2.Encuentre $i_1$, $i_2$, $i_x$, $i_y$ e $i_z$.(malla $i_1$ izquierda y malla $i_2$ derecha) (dirección del análisis de mallas sentido horario) 


\begin{center}
    \begin{circuitikz}
        \draw
        (0,0)
            to[V,l=\SI{12}{\volt},i=$i_x$]
        (0,3)
            to[R,l=\SI{4}{\ohm}]
        (3,3)
            to[V,l=\SI{10}{\volt}, invert]
        (6,3)
            to[R,l=\SI{2}{\ohm},i=$i_z$]
        (6,0)
            to[short]
        (0,0)
        (3,3)
            to[R,l=\SI{6}{\ohm},i=$i_y$]
        (3,0)
        ;
    \end{circuitikz}
\end{center}

R/  $i_1=\SI{818.2}{\milli\ampere}$
$i_2=\SI{-636.4}{\milli\ampere}$, $i_x=\SI{-818.2}{\milli\ampere}$, $i_y=\SI{1454.6}{\milli\ampere}$, $i_z=\SI{-636.4}{\milli\ampere}$\\[8pt]

3.Encuentre $v_0$
 \begin{center}
     \begin{circuitikz}
         \draw
         (0,0)
           to[R,l=\SI{4}{\ohm}]
         (0,3.5)
             to[short]
         (0,6)
         (-0.1,3.7)node[left]{$v_1$}
         (0,3.5)
             to[short]
         (-3.5,3.5)
             to[I,l=\SI{3}{\ampere}]
         (-3.5,0)
             to[short]
         (0,0)
         (0,6)
           to[I,l=\SI{2}{\ampere}]
         (4,6)
             to[short]
         (4,3.5) 
        
         (7,3.5)
           to[I,l=\SI{6}{\ampere}]
         (7,0)
            to[short]
         (4,0)
            to[V,l=\SI{30}{\volt}]
         (1.5,0)
         (1.7,0)
            to[R,l=\SI{2}{\ohm},v=$V_o$,invert]
         (0,0)
    
         (0,3.5)
             to[R,l=\SI{9}{\ohm}]
         (4,3.5)
         (4.1,3.7)node[right]{$v_2$}
         (4,3.5)
             to[R,l=\SI{5}{\ohm}]
         (4,0)
         (4,3.5)
             to[short] 
         (7,3.5)
         ;
     \end{circuitikz}
 \end{center}
 R/
 $v_0=\SI{-6.6}{\volt}$ \\[8pt]

4.Encuentre $v_m$ 
 \begin{center}
     \begin{circuitikz}
         \draw
         (0,0)
           to[I,l=\SI{6}{\ampere}]
         (0,3.5)
             to[short]
         (-3.5,3.5)
             to[R,l=\SI{2}{\ohm}]
         (-3.5,0)
             to[short]
         (0,0)
         
         (4,3.5) 
        
         (7,3.5)
           to[R,l=\SI{8}{\ohm},v=$V_m$]
         (7,0)
            to[short]
         (4,0)
             to[cV,l=$4_im$,invert]
         (0,0)
            
         (0,3.5)
             to[short]
         (4,3.5)
             to[I,l=\SI{4}{\ampere},invert]
         (4,0)
         (4,3.5)
             to[short] 
         (7,3.5)
         ;
     \end{circuitikz}
 \end{center}
 R/
 $v_m=\SI{-26,67}{\volt}$ \\[8pt]

5.Encuentre $v_x$ 
 \begin{center}
     \begin{circuitikz}
         \draw
         (0,0)
           to[R,l=\SI{40}{\ohm}]
         (0,3.5)
             to[R,l=\SI{10}{\ohm}]
         (-3.5,3.5)
             to[V,l=\SI{50}{\volt}, invert]
         (-3.5,0)
             to[short]
         (0,0)
         
         (4,3.5) 
            to[R,l=\SI{20}{\ohm}]
         (7,3.5)
           to[V,l=\SI{40}{\volt}, invert]
         (7,0)
            to[short]
         (4,0)
             to[short]
         (0,0)
         
         (0,3.5)
             to[R,l=\SI{12}{\ohm},v=$V_x$]
         (4,3.5)
             to[I,l=\SI{8}{\ampere},invert]
         (4,0)
         ;
     \end{circuitikz}
 \end{center}
 R/
 $v_x=\SI{-48}{\volt}$ \\[8pt]

6.Encuentre $i_0$
 \begin{center}
     \begin{circuitikz}
         \draw
         (0,0)
           to[R,l=\SI{2}{\ohm}]
         (0,3.5)
             to[short]
         (0,6)
         
         (0,3.5)
             to[short]
         (-3.5,3.5)
             to[I,l=\SI{6}{\ampere},invert]
         (-3.5,0)
             to[short]
         (0,0)
         (0,6)
           to[I,l=\SI{3}{\ampere}]
         (4,6)
             to[short]
         (4,3.5) 
        
         (7,3.5)
            to[V,l=\SI{20}{\volt}]
         (7,0)
            to[short]
         (0,0)
    
         (0,3.5)
             to[R,l=\SI{9}{\ohm}]
         (4,3.5)
             to[R,l=\SI{4}{\ohm},i>^=$i_x$]
         (7,3.5)
         ;
     \end{circuitikz}
 \end{center}
 R/
 $i_0=\SI{636,4}{\milli\ampere}$ \\[8pt]


7.Encuentre $v_o$ y $i_o$ 
 \begin{center}
     \begin{circuitikz}
         \draw
         (0,0)
           to[R,l=\SI{40}{\ohm}]
         (0,3.5)
             to[short]
         (-3.5,3.5)
             to[I,l=\SI{6}{\ampere},invert]
         (-3.5,0)
             to[short]
         (0,0)
         
         (4,3.5) 
            to[R,l=\SI{20}{\ohm}]
         (7,3.5)
           to[V,l=\SI{30}{\volt}]
         (7,0)
            to[short]
         (4,0)
             to[short]
         (0,0)
         
         (0,3.5)
             to[R,l=\SI{10}{\ohm},v=$V_o$, i>^=$i_o$]
         (4,3.5)
             to[cI,l=$4_io$,invert]
         (4,0)
         ;
     \end{circuitikz}
 \end{center}


\end{document}