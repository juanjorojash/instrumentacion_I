\documentclass[12pt,letterpaper]{article}
\usepackage{amsmath}
\usepackage{amssymb}
\usepackage{siunitx}
\usepackage{float}
\usepackage{tikz}
\usepackage{url}
\usepackage[siunitx,american,RPvoltages]{circuitikz}
\ctikzset{capacitors/scale=0.7}
\ctikzset{diodes/scale=0.7}
\usepackage{tabularx}
\newcolumntype{C}{>{\centering\arraybackslash}X}
\renewcommand\tabularxcolumn[1]{m{#1}}% for vertical centering text in X column
\usepackage{tabu}
\usepackage[spanish,es-tabla,activeacute]{babel}
\usepackage{babelbib}
\usepackage{booktabs}
\usepackage{pgfplots}
\usepackage{hyperref}
\hypersetup{colorlinks = true,
            linkcolor = black,
            urlcolor  = blue,
            citecolor = blue,
            anchorcolor = blue}
\usepgfplotslibrary{units, fillbetween} 
\pgfplotsset{compat=1.16}
\usepackage{bm}
\usetikzlibrary{arrows, arrows.meta, shapes, 3d, perspective, positioning}
\renewcommand{\sin}{\sen} %change from sin to sen
\usepackage{bohr}
\setbohr{distribution-method = quantum,insert-missing = true}
\usepackage{elements}
\usepackage{verbatim}
\usepackage{geometry}  
\geometry{left=18mm,right=18mm,top=21mm,bottom=21mm,headheight=15pt} % Tamaño del área de escritura de la página
\setlength\parindent{0pt} % Removes all indentation from paragraphs
\usepackage{fancyhdr}
\pagestyle{fancy}
\lhead{Problemas de apoyo: Electricidad I}
\rhead{\begin{picture}(0,0) \put(-110,0){\includegraphics[width=40mm]{fig/logo.png}} \end{picture}}
\usepackage{titlesec}

\titleformat{\section}{\normalfont \Large \bfseries}
{Semana\ \thesection}{2.3ex plus .2ex}{} %% Text "Semana" is what you would like the section number to display with.
%\titlespacing{\subsection}{2em}{*1}{*1}

\titleformat{\subsection}[runin]{\normalfont \large \bfseries}
{Problema\ \thesubsection}{0pt}{} %% Text "Problema" is what you would like the section number to display with.

\newcommand{\asection}[2]{
\setcounter{section}{#1}
\addtocounter{section}{-1}
\section{#2}
}


\begin{document}
\input{comunes/postamble}

\asection{6}{Thevenin, Norton y Máxima transferencia de potencia}

1. Determine i mediante el Teorema de Thevenin.


\begin{center}
    \begin{circuitikz}
        \draw
        (0,0)
            to[V,l=\SI{50}{\volt}]
        (0,1.5)
            to[R,l=\SI{10}{\ohm}]
        (0,3)
            to[short]
        (3,3)
            to[short]
        (6,3)
            to[R,l=\SI{40}{\ohm}]
        (6,0)
            to[short]
        (0,0)
        (3,3)
        to[short,i=$i$]
        (3,2.7)
        (3,2.7)
            to[R,l=\SI{12}{\ohm}]
        (3,1.5)
            to[V,l=\SI{30}{\volt},invert]
        (3,0)
        ;
    \end{circuitikz}
\end{center}
R/ $i=\SI{500}{\milli\ampere}$ \\[16pt]

2. Determine el equivalente de Thevenin en las terminales a-b.
 
\begin{center}
    \begin{circuitikz}
        \draw
        (0,0)
            to[V,l=\SI{70}{\volt}]
        (0,3)
            to[R,l=\SI{10}{\kilo\ohm},v=$V_o$]
        (3,3)
            to[R,l=\SI{20}{\kilo\ohm}]
        (6,3)
            to[cV,l=$4V_o$, invert]
        (6,0)
            to[short]
        (0,0)
        (3,3)
            to[short, -o]
        (3,2)node[left]{$a$}    
        (3,1)node[left]{$b$}
            to[short, o-]
        (3,0)
        ;
        
    \end{circuitikz}
\end{center}

R/ $R_Th=\SI{2,857}{\kilo\ohm}$ \\[16pt]

3.Determine los equivalentes de Thevenin y Norton en las terminales a-b.

 \begin{center}
     \begin{circuitikz}
         \draw
         (0,0)
           to[R,l=\SI{6}{\ohm}]
         (0,3.5)
             to[short]
         (-3.5,3.5)
            to[I,l=\SI{1}{\ampere},invert]
         (-3.5,0)
             to[short]
         (0,0)
         
         (4,3.5) 
            to[R,l=\SI{20}{\ohm}]
         (7,3.5)
           to[R,l=\SI{5}{\ohm}]
         (7,0)
            to[short,-o]
         (7.5,0)node[above]{$b$}
         (7,3.5)
            to[short,-o]
         (7.5,3.5)node[above]{$a$}
         (7,0)
            to[short]
         (4,0)
             to[short]
         (0,0)
         
         (0,3.5)
            to[V,l=\SI{14}{\volt}] 
         (2,3.5)
            to[R,l=\SI{14}{\ohm}]
         (4,3.5)
            to[I,l=\SI{1}{\ampere}]
         (4,0)
         ;
     \end{circuitikz}
 \end{center}
R/ $R_Th=\SI{4}{\ohm}$,
y $R_N=\SI{4}{\ohm}$, $V_Th=\SI{-8}{\volt}$ e $I_N=\SI{-2}{\ampere}$
\\[16pt]
 4.Determine el equivalente de
Norton visto desde las terminales:
a) a-b, b) c-d
 \begin{center}
     \begin{circuitikz}
         \draw
         (0,0)
           to[R,l=\SI{3}{\ohm}]
         (0,3.5)
             to[R,l=\SI{6}{\ohm}]
         (-3.5,3.5)
             to[V,l=\SI{120}{\volt},invert]
         (-3.5,0)
             to[short]
         (0,0)
         
         (4,3.5) 
            to[short]
         (7,3.5)
           to[R,l=\SI{2}{\ohm}]
         (7,0)
            to[short,-o]
         (7.5,0)node[above]{$d$}
         (7,3.5)
            to[short,-o]
         (7.5,3.5)node[above]{$c$}
         (7,0)
            to[short]
         (4,0)
             to[short]
         (0,0)
         (4,3.5)
             to[short,-o]
         (4,4)node[above]{$b$}
         (0,3.5)
            to[short,-o]
         (0,4)node[above]{$a$}
         
         (0,3.5)
            to[R,l=\SI{4}{\ohm}]
         (4,3.5)
            to[I,l=\SI{6}{\ampere},invert]
         (4,0)
         ;
     \end{circuitikz}
 \end{center}
 
 R/
 a) 
  $R_N=\SI{2}{\ohm}$, $V_Th=\SI{14}{\volt}$ e $I_N=\SI{7}{\ampere}$\\[8pt]
b) 
  $R_N=\SI{1,5}{\ohm}$, $V_Th=\SI{19}{\volt}$ e $I_N=\SI{12,667}{\ampere}$\\[16pt]
5. Encuentre el equivalente de Norton en las terminales a-b.
 \begin{center}
     \begin{circuitikz}
         \draw
         (0,0)
           to[R,l=\SI{3}{\ohm},v=$V_o$]
         (0,3.5)
             to[short]
         (0,6)
         (0,3.5)
             to[R,l=\SI{6}{\ohm}]
         (-3.5,3.5)
             to[V,l=\SI{18}{\volt},invert]
         (-3.5,0)
             to[short]
         (0,0)
         (0,6)
           to[cI,l=$0.25V_o$, invert]
         (4,6)
             to[short]
         (4,3.5) 
        
         (4,0)
            to[short, -o]
         (4.5,0)node[above]{$b$}
            to[short]
         (0,0)
    
         (0,3.5)
             to[R,l=\SI{2}{\ohm}]
         (4,3.5)
             to[short, -o]
         (4.5,3.5)node[above]{$a$}
         ;
     \end{circuitikz}
 \end{center}
R/
  
  $R_N=\SI{3}{\ohm}$ e $I_N=\SI{1}{\ampere}$\\[16pt]



6. Determine los equivalentes de Thevenin y Norton en las terminales a-b.

\begin{center}
    \begin{circuitikz}
        \draw
        (3,4)   
            to[short]
        (0,4)
            to[I,l=\SI{8}{\ampere}, invert]
        (0,0)
            to[short]
        (6,0)
        (3,4)
            to[short]
        (6,4)
            to[R,l=\SI{20}{\ohm}]
        (6,2)
        (3,4)
            to[R,l=\SI{10}{\ohm}]
        (3,2)
            to[R,l=\SI{50}{\ohm}]
        (3,0)
            
        (3,2)
            to[short,-o]
        (4,2)node[above]{$a$}
            
        (5,2)node[above]{$b$}
            to[short,-o]
        (6,2)
            to[R,l=\SI{40}{\ohm}]
        (6,0)

        ;
    \end{circuitikz}
\end{center}
R/
$R_Th=\SI{22,5}{\ohm}$, $V_Th=\SI{40}{\volt}$ e $I_N=\SI{1,7778}{\ampere}$\\[16pt]

7.Determine la máxima potencia suministrada al resistor variable R.

 \begin{center}
     \begin{circuitikz}
         \draw
         (0,0)
           to[R,l=\SI{15}{\ohm}]
         (0,3.5)
             to[short]
         (0,6)
         
         (0,3.5)
             to[R,l=\SI{5}{\ohm}]
         (-3.5,3.5)
             to[V,l=\SI{4}{\volt},invert]
         (-3.5,0)
             to[short]
         (0,0)
         (0,6)
           to[cI,l=$3V_x$]
         (4,6)
             to[short]
         (4,3.5) 
        
         (4,0)
            to[R,l=\SI{6}{\ohm},v=$V_x$]
         (0,0)
    
         (0,3.5)
             to[R,l=\SI{5}{\ohm}]
         (4,3.5)
         
         (4,3.5)
            to[R,l=$R$]
         (4,0)
         
         ;
     \end{circuitikz}
 \end{center}
 R/
 $Pmax=\SI{21.48}mW$ \\[8pt]





\end{document}