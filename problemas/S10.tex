\documentclass[12pt,letterpaper]{article}
\usepackage{amsmath}
\usepackage{amssymb}
\usepackage{siunitx}
\usepackage{float}
\usepackage{tikz}
\usepackage{url}
\usepackage[siunitx,american,RPvoltages]{circuitikz}
\ctikzset{capacitors/scale=0.7}
\ctikzset{diodes/scale=0.7}
\usepackage{tabularx}
\newcolumntype{C}{>{\centering\arraybackslash}X}
\renewcommand\tabularxcolumn[1]{m{#1}}% for vertical centering text in X column
\usepackage{tabu}
\usepackage[spanish,es-tabla,activeacute]{babel}
\usepackage{babelbib}
\usepackage{booktabs}
\usepackage{pgfplots}
\usepackage{hyperref}
\hypersetup{colorlinks = true,
            linkcolor = black,
            urlcolor  = blue,
            citecolor = blue,
            anchorcolor = blue}
\usepgfplotslibrary{units, fillbetween} 
\pgfplotsset{compat=1.16}
\usepackage{bm}
\usetikzlibrary{arrows, arrows.meta, shapes, 3d, perspective, positioning}
\renewcommand{\sin}{\sen} %change from sin to sen
\usepackage{bohr}
\setbohr{distribution-method = quantum,insert-missing = true}
\usepackage{elements}
\usepackage{verbatim}
\usepackage{geometry}  
\geometry{left=18mm,right=18mm,top=21mm,bottom=21mm,headheight=15pt} % Tamaño del área de escritura de la página
\setlength\parindent{0pt} % Removes all indentation from paragraphs
\usepackage{fancyhdr}
\pagestyle{fancy}
\lhead{Problemas de apoyo: Electricidad I}
\rhead{\begin{picture}(0,0) \put(-110,0){\includegraphics[width=40mm]{fig/logo.png}} \end{picture}}
\usepackage{titlesec}

\titleformat{\section}{\normalfont \Large \bfseries}
{Semana\ \thesection}{2.3ex plus .2ex}{} %% Text "Semana" is what you would like the section number to display with.
%\titlespacing{\subsection}{2em}{*1}{*1}

\titleformat{\subsection}[runin]{\normalfont \large \bfseries}
{Problema\ \thesubsection}{0pt}{} %% Text "Problema" is what you would like the section number to display with.

\newcommand{\asection}[2]{
\setcounter{section}{#1}
\addtocounter{section}{-1}
\section{#2}
}


\begin{document}
\input{comunes/postamble}

\asection{10}{Funciones de singularidad, respuesta completa en  RL y RC}

\subsection{} El capacitor de la figura se encuentra cargado con $v_c(0)=\SI{8}{\volt}$. Escriba las expresiones matematicas para $v_c$ e $i_x$ para todo tiempo $t>0$.\\

\begin{center}
    \begin{circuitikz}
    \draw
        (0,0) node[left]{$+\SI{10}{\volt}$} 
            to[R,l=\SI{2}{\kilo\ohm},o-]
        (3,0) node[ground]{}
            to[R,l=\SI{6.8}{\kilo\ohm},i<^=$i_x$]
        (6,0)
            to[spst, l=$t\text{$=$}0.4$] 
        (7,0)
            to[C,l=\SI{30}{\micro\farad},v=$v_c$,voltage shift=3,-o]
        (10,0) node[right]{$+\SI{15}{\volt}$} 
    ;
    \end{circuitikz}   
\end{center}

\subsection{} El capacitor de la figura se encuentra cargado con $v_c(0)=\SI{10}{\volt}$. Escriba las expresiones matematicas para $v_c$ e $i_x$ para todo tiempo $t>0$.\\

\begin{center}
    \begin{circuitikz}
    \draw
        (0,0) node[left]{$+\SI{10}{\volt}$} 
            to[R,l=\SI{2}{\kilo\ohm},o-]
        (3,0) 
            to[R,l=\SI{6}{\kilo\ohm}]
        (6,0)
            to[spst, l=$t\text{$=$}0.2$] 
        (7,0)
            to[C,l=\SI{20}{\micro\farad},v=$v_c$,voltage shift=3]
        (10,0) 
            to[R,l=$\SI{2}{\kilo\ohm}$,-o]
        (13,0)node[right]{$+\SI{15}{\volt}$} 
        (3,0)
            to[short,i=$i_x$]
        (3,-1) node[ground]{}
    ;
    \end{circuitikz}   
\end{center}

\end{document}