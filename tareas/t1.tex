\documentclass[12pt]{article}
\usepackage[margin=1in]{geometry} 
\usepackage{amsmath}
\usepackage{amssymb}
\usepackage{siunitx}
\usepackage{float}
\usepackage{tikz}
\def\checkmark{\tikz\fill[scale=0.4](0,.35) -- (.25,0) -- (1,.7) -- (.25,.15) -- cycle;} 
\usepackage{url}
\usepackage[siunitx,american,RPvoltages]{circuitikz}
\ctikzset{capacitors/scale=0.7}
\ctikzset{diodes/scale=0.7}
\usepackage{tabularx}
\newcolumntype{C}{>{\centering\arraybackslash}X}
\renewcommand\tabularxcolumn[1]{m{#1}}% for vertical centering text in X column
\usepackage{tabu}
\usepackage[spanish,es-tabla,activeacute]{babel}
\usepackage{babelbib}
\usepackage{booktabs}
\usepackage{pgfplots}
\usepackage{hyperref}
\hypersetup{colorlinks = true,
            linkcolor = black,
            urlcolor  = blue,
            citecolor = blue,
            anchorcolor = blue}
\usepgfplotslibrary{units, fillbetween} 
\pgfplotsset{compat=1.16}
\usepackage{bm}
\usetikzlibrary{arrows, arrows.meta, shapes, 3d, perspective, positioning,mindmap,trees,backgrounds}
\renewcommand{\sin}{\sen} %change from sin to sen
\usepackage{bohr}
\setbohr{distribution-method = quantum,insert-missing = true}
\usepackage{elements}
\usepackage{verbatim}
\usepackage[edges]{forest}
\usepackage{etoolbox}
\usepackage{schemata}
\usepackage{appendix}
\usepackage{listings}

\definecolor{color_mate}{RGB}{255,255,128}
\definecolor{color_plas}{RGB}{255,128,255}
\definecolor{color_text}{RGB}{128,255,255}
\definecolor{color_petr}{RGB}{255,192,192}
\definecolor{color_made}{RGB}{192,255,192}
\definecolor{color_meta}{RGB}{192,192,255}
\newcommand\diagram[2]{\schema{\schemabox{#1}}{\schemabox{#2}}}

\definecolor{codegreen}{rgb}{0,0.6,0}
\definecolor{codegray}{rgb}{0.5,0.5,0.5}
\definecolor{codepurple}{rgb}{0.58,0,0.82}
\definecolor{backcolour}{rgb}{0.95,0.95,0.92}

\lstdefinestyle{mystyle}{
    backgroundcolor=\color{backcolour},   
    commentstyle=\color{codegreen},
    keywordstyle=\color{magenta},
    numberstyle=\tiny\color{codegray},
    stringstyle=\color{codepurple},
    basicstyle=\ttfamily\footnotesize,
    breakatwhitespace=false,         
    breaklines=true,                 
    captionpos=b,                    
    keepspaces=true,                 
    numbers=left,                    
    numbersep=5pt,                  
    showspaces=false,                
    showstringspaces=false,
    showtabs=false,                  
    tabsize=2
}

\lstset{style=mystyle}
\usepackage{lastpage}
\usepackage{fancyhdr}
\usepackage{enumitem}
\usepackage{csvsimple,booktabs}
\pagestyle{fancy}
\setlength{\headheight}{42pt}

 
\begin{document}
\lhead{Ingeniería Física \\ Escuela de Física \\ Tecnológico de Costa Rica} 
\rhead{Instrumentación I \\ Tarea \#1  \\ Entrega: Semana 4} 
\cfoot{\thepage\ de \pageref{LastPage}}
\setlength{\parindent}{0em}

\begin{enumerate}[label=(\Alph*)]

\item Investigue el principio de funcionamiento de la Balance de Kibble, y comente su importancia en la redefinición del kilogramo.

\item El Sistema Internacional de Unidades constituido por siete unidades básicas que definen a las correspondientes magnitudes físicas fundamentales: metro, kilogramo, segundo, kelvin, amperio, mol y candela. Investigue cómo ha evolucionado la definición de cada una de las unidade básicas desde su incorporación al Sistema hasta el día de hoy. 

\item Asuma que cuenta con los siguientes equipos
\begin{itemize}
    \item Una fuente de corriente directa configurable para entregar voltaje constante o corriente constante según se requiera. 
    \item Un equipo registrador de mediciones con conexiones para varios tipos de sensores
    \item Un sensor de corriente con resistencia interna $R_{sh}$ conectado al equipo registrador
    \item Un sensor de voltaje conectado al equipo registrador
\end{itemize}
El \href{https://www.ee.iitb.ac.in/~spilab/Tips/ansii_graphic_symbols_for_electrical_and_electronics_daigrams_1993.pdf}{IEEE Std 315-1975} define la forma en que se representan los diagramas eléctricos, en base a ese estándar se diagrama el circuito tal como se muestra en la Figura \ref{fig:circuito}. Este diagrama contiene varias inexactitudes, entre ellas:
\begin{itemize}
    \item No existe una representación de un registrador de mediciones
    \item El sensor de corriente se representa como un amperímetro y no hay nada que indique su conexión con el registrador
    \item El sensor de voltaje se representa como un voltímetro y no hay nada que indique su conexión con el registrador
\end{itemize}

\begin{figure}[H]
    \centering
    \begin{circuitikz} 
        \draw
        (0,0) 	
            to[V, l=$V_f$] 
        (0,4)
        	to[short] 
        (3,4)
        	to[rmeter, t=A]
        (3,2) 
            to[R, l=$R_{sh}$]
        (3,0)-- (0,0)
        (3,4) -- (5,4)
            to[rmeter, t=V] 
        (5,0) -- (3,0)
        ;
    \end{circuitikz}
    \caption{Medición de corriente y voltaje en un circuito.}
    \label{fig:circuito}
\end{figure}

Tome en cuenta lo siguiente:
\begin{itemize}
    \item La exactitud se calculará como el mayor error relativo obtenido en cualquier punto de de datos en todas las corridas de medición, en este caso solo hay una corrida.
    \item La precisión se calculará como la mayor desviación estándar entre todas las corridas de medición, en este caso solo hay una corrida.
    \item La repetibilidad se calculará de la siguiente manera:
    \begin{equation*}
        \mathrm{repetibilidad} = \sqrt{\dfrac{\sum_{i=1}^n(\mathrm{error\,absoluto})^2}{n}}
    \end{equation*}
\end{itemize}
\newpage
\noindent\textbf{Determine:}
\begin{enumerate}
    \item Usando el estándar \href{http://integrated.cc/cse/Instrumentation_Symbols_and_Identification.pdf}{ANSI/ISA-5.1-2009} realice un diagrama que describa de mejor manera el proceso de medición descrito. El diagrama básico de conexión debe ser como el mostrado en la Figura \ref{fig:simple}. El resto del diagrama debe realizarlo en base a ANSI/ISA-5.1-2009
    \begin{figure}[H]
        \centering
        \begin{circuitikz} 
            \draw
            (0,0) 	
                to[V, l=$V_f$] 
            (0,3)
            	to
            (3,3)
                to[R, l=$R_{sh}$]
            (3,0)-- (0,0)
            ;
        \end{circuitikz}
        \caption{Medición de corriente y voltaje en un circuito.}
        \label{fig:simple}
    \end{figure}
    
    \begin{table}[H]
        \centering
        \caption{Mediciones tomadas a una frecuencia de \SI{1}{\hertz}}
        \vspace{0.5cm}
        \begin{tabular}{ccc}%
        \toprule
        \bfseries tiempo & \bfseries corriente & \bfseries voltaje \\
        {[\si{\second}]} & [\si{\ampere}] & [\si{\volt}]\\
        %\[\si{\second}\] & \[\si{\ampere}\] & \[\si{\volt}\] \\
        \midrule
        \csvreader[
            late after line=\\,
            late after last line=,
            before reading={\catcode`\#=12},
            after reading={\catcode`\#=6}]%
            {data.csv}{1=\time1,2=\current1,3=\voltage1}{\time1 &\current1 & \voltage1}\\
            \bottomrule
        \end{tabular}
        \label{tab:datos}
    \end{table}

    \item Se realizó una medición por segundo durante 20 segundos usando el proceso de medición descrito y se obtuvieron los datos mostrados en la Tabla \ref{tab:datos}. Calcule el valor de $R_{sh}$ para cada uno de los $n$ puntos de datos y realice una tabla que incluya las tres columnas de la Tabla \ref{tab:datos} y una tercera columna para los valores de $R_{sh}$.
    \item Para el conjunto de valores de $R_{sh}$, calcule lo siguiente e incluya en una tabla:
    \begin{enumerate}
        \item El valor promedio, $\overline{R_{sh}}$ 
        \item La desviación estándar, $\sigma$
        \item El valor promedio de error relativo tomando \SI{1}{\ohm} como valor real
        \item El valor de la incertidumbre estándar, $\sigma_x = \sigma / \sqrt{n}$
        \item La exactitud
        \item La precisión
        \item La repetibilidad
    \end{enumerate}
\end{enumerate}
\end{enumerate}




% \bibliographystyle{IEEEtran}
% \bibliography{ref_tareas}

\end{document}