\documentclass[12pt]{article}
\usepackage[margin=1in]{geometry} 
\usepackage{amsmath}
\usepackage{amssymb}
\usepackage{siunitx}
\usepackage{float}
\usepackage{tikz}
\def\checkmark{\tikz\fill[scale=0.4](0,.35) -- (.25,0) -- (1,.7) -- (.25,.15) -- cycle;} 
\usepackage{url}
\usepackage[siunitx,american,RPvoltages]{circuitikz}
\ctikzset{capacitors/scale=0.7}
\ctikzset{diodes/scale=0.7}
\usepackage{tabularx}
\newcolumntype{C}{>{\centering\arraybackslash}X}
\renewcommand\tabularxcolumn[1]{m{#1}}% for vertical centering text in X column
\usepackage{tabu}
\usepackage[spanish,es-tabla,activeacute]{babel}
\usepackage{babelbib}
\usepackage{booktabs}
\usepackage{pgfplots}
\usepackage{hyperref}
\hypersetup{colorlinks = true,
            linkcolor = black,
            urlcolor  = blue,
            citecolor = blue,
            anchorcolor = blue}
\usepgfplotslibrary{units, fillbetween} 
\pgfplotsset{compat=1.16}
\usepackage{bm}
\usetikzlibrary{arrows, arrows.meta, shapes, 3d, perspective, positioning,mindmap,trees,backgrounds}
\renewcommand{\sin}{\sen} %change from sin to sen
\usepackage{bohr}
\setbohr{distribution-method = quantum,insert-missing = true}
\usepackage{elements}
\usepackage{verbatim}
\usepackage[edges]{forest}
\usepackage{etoolbox}
\usepackage{schemata}
\usepackage{appendix}
\usepackage{listings}

\definecolor{color_mate}{RGB}{255,255,128}
\definecolor{color_plas}{RGB}{255,128,255}
\definecolor{color_text}{RGB}{128,255,255}
\definecolor{color_petr}{RGB}{255,192,192}
\definecolor{color_made}{RGB}{192,255,192}
\definecolor{color_meta}{RGB}{192,192,255}
\newcommand\diagram[2]{\schema{\schemabox{#1}}{\schemabox{#2}}}

\definecolor{codegreen}{rgb}{0,0.6,0}
\definecolor{codegray}{rgb}{0.5,0.5,0.5}
\definecolor{codepurple}{rgb}{0.58,0,0.82}
\definecolor{backcolour}{rgb}{0.95,0.95,0.92}

\lstdefinestyle{mystyle}{
    backgroundcolor=\color{backcolour},   
    commentstyle=\color{codegreen},
    keywordstyle=\color{magenta},
    numberstyle=\tiny\color{codegray},
    stringstyle=\color{codepurple},
    basicstyle=\ttfamily\footnotesize,
    breakatwhitespace=false,         
    breaklines=true,                 
    captionpos=b,                    
    keepspaces=true,                 
    numbers=left,                    
    numbersep=5pt,                  
    showspaces=false,                
    showstringspaces=false,
    showtabs=false,                  
    tabsize=2
}

\lstset{style=mystyle}
\usepackage{lastpage}
\usepackage{fancyhdr}
\usepackage{csvsimple,booktabs}
\pagestyle{fancy}
\setlength{\headheight}{42pt}

 
\begin{document}
\lhead{Ingeniería Física \\ Escuela de Física \\ Tecnológico de Costa Rica} 
\rhead{Instrumentación I \\ Tarea \#2  \\ Entrega: Semana 6} 
\cfoot{\thepage\ de \pageref{LastPage}}
\setlength{\parindent}{0em}

\noindent\textbf{Problema a resolver}
\newline
\newline
Ustedes se encuentran del diseño de un sistema de monitoreo para una planta de \href{https://www.youtube.com/watch?v=oPHlnLcOxyM&ab_channel=JoseViFerrer}{fundición de silicio para microchips.} El proceso requiere la instalación de múltiples sensores de temperatura, que presentan las siguientes características: 
\begin{itemize}
    \item La temperatura máxima de operación es de $1650 ^{\circ} C$. Presentada en la fundición de silicio. 
    \item Se requiere mantener elementos de dopaje y minerales de tierras raras en almacenamiento a una temperatura de $-5 ^{\circ} C$.
    \item La mayor parte de la operación ocurre bajo vacío y en condiciones oxidantes para el material. 
    \item Dado lo delicado del proceso, se requiere tener un control de calidad de temperatura. Para lo que requiere una precisión máxima permitida en la medición de $\pm 0.5 ^{\circ} C $. Para poder cumplir con lo estipulado en la normativa \href{https://www.iso.org/obp/ui/#iso:std:iso-iec:17025:ed-3:v2:es}{ISO-IEC 17025:2017} 
    \item Se requiere un monitoreo constante de temperatura, por lo que el dispositivo deberá contar con todas las interfaces necesarias para poder ser leído en un sistema de adquisición de datos \textit{DAQ}.
\end{itemize}

\noindent\textbf{Instrucciones}
\newline

Su trabajo consiste en la realización de una propuesta técnica para que la gerencia apruebe la propuesta de adquisición de los sensores. Por lo que deberá: 
 \begin{itemize}
     \item Buscar al menos un sensor comercial de la tecnología \textit{Termopar}, \textit{RTD},  \textit{Termistor} y algún sobre alguna otra tecnología \textit{infrarroja}. (Para los sensores seleccionados deberá adjuntar la hoja de datos correspondiente). 
     \item Para tomar una decisión, deberá realizar una comparativa utilizando una \href{https://asana.com/es/resources/decision-matrix-examples}{matriz de decisión}. Se sugiere la utilización del \href{https://www.isixsigma.com/dictionary/pugh-matrix/}{método de Pough}.
     \item Para cada uno de los criterios utilizados, deberán explicar porqué los utilizan. 
     \item En la consideración de los costos, deberán contemplar la utilización de interfaces de acondicionamiento de señal, en caso de ser necesario. 
     \item Un aspecto importante por la compañía, es el mantenimiento y la calibración que se le debe dar a cada uno de los sensores. Para cada uno de los sensores seleccionados, deberán establecer los rangos de mantenimiento y recalibración que necesitan a futuro, para que esto pueda ser seleccionado en el plan general de mantenimiento de la compañía. 
 \end{itemize}
\end{document}