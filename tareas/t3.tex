\documentclass[12pt]{article}
\usepackage[margin=1in]{geometry} 
\usepackage{amsmath}
\usepackage{amssymb}
\usepackage{siunitx}
\usepackage{float}
\usepackage{tikz}
\def\checkmark{\tikz\fill[scale=0.4](0,.35) -- (.25,0) -- (1,.7) -- (.25,.15) -- cycle;} 
\usepackage{url}
\usepackage[siunitx,american,RPvoltages]{circuitikz}
\ctikzset{capacitors/scale=0.7}
\ctikzset{diodes/scale=0.7}
\usepackage{tabularx}
\newcolumntype{C}{>{\centering\arraybackslash}X}
\renewcommand\tabularxcolumn[1]{m{#1}}% for vertical centering text in X column
\usepackage{tabu}
\usepackage[spanish,es-tabla,activeacute]{babel}
\usepackage{babelbib}
\usepackage{booktabs}
\usepackage{pgfplots}
\usepackage{hyperref}
\hypersetup{colorlinks = true,
            linkcolor = black,
            urlcolor  = blue,
            citecolor = blue,
            anchorcolor = blue}
\usepgfplotslibrary{units, fillbetween} 
\pgfplotsset{compat=1.16}
\usepackage{bm}
\usetikzlibrary{arrows, arrows.meta, shapes, 3d, perspective, positioning,mindmap,trees,backgrounds}
\renewcommand{\sin}{\sen} %change from sin to sen
\usepackage{bohr}
\setbohr{distribution-method = quantum,insert-missing = true}
\usepackage{elements}
\usepackage{verbatim}
\usepackage[edges]{forest}
\usepackage{etoolbox}
\usepackage{schemata}
\usepackage{appendix}
\usepackage{listings}

\definecolor{color_mate}{RGB}{255,255,128}
\definecolor{color_plas}{RGB}{255,128,255}
\definecolor{color_text}{RGB}{128,255,255}
\definecolor{color_petr}{RGB}{255,192,192}
\definecolor{color_made}{RGB}{192,255,192}
\definecolor{color_meta}{RGB}{192,192,255}
\newcommand\diagram[2]{\schema{\schemabox{#1}}{\schemabox{#2}}}

\definecolor{codegreen}{rgb}{0,0.6,0}
\definecolor{codegray}{rgb}{0.5,0.5,0.5}
\definecolor{codepurple}{rgb}{0.58,0,0.82}
\definecolor{backcolour}{rgb}{0.95,0.95,0.92}

\lstdefinestyle{mystyle}{
    backgroundcolor=\color{backcolour},   
    commentstyle=\color{codegreen},
    keywordstyle=\color{magenta},
    numberstyle=\tiny\color{codegray},
    stringstyle=\color{codepurple},
    basicstyle=\ttfamily\footnotesize,
    breakatwhitespace=false,         
    breaklines=true,                 
    captionpos=b,                    
    keepspaces=true,                 
    numbers=left,                    
    numbersep=5pt,                  
    showspaces=false,                
    showstringspaces=false,
    showtabs=false,                  
    tabsize=2
}

\lstset{style=mystyle}
\usepackage{lastpage}
\usepackage{fancyhdr}
\usepackage{csvsimple,booktabs}
\pagestyle{fancy}
\setlength{\headheight}{42pt}

 
\begin{document}
\lhead{Ingeniería Física \\ Escuela de Física \\ Tecnológico de Costa Rica} 
\rhead{Instrumentación I \\ Tarea \#3  \\ Entrega: Semana 8} 
\cfoot{\thepage\ de \pageref{LastPage}}
\setlength{\parindent}{0em}

\noindent\textbf{Problema a resolver}
\newline
\newline
Se desea crear un altímetro a partir de sensor de presión absoluta modelo:\href{https://www.te.com/commerce/DocumentDelivery/DDEController?Action=srchrtrv&DocNm=MS5803-02BA&DocType=Data+Sheet&DocLang=English}{MS5803-02BA} de TE Connectivity, un microcontrolador y una pantalla. Este sensor entrega un dato digital entre 1000 y 120000 que corresponde a 10mbar y 1200mbar respectivamente, es totalmente lineal. 

\vspace{0.5cm}
\noindent\textbf{Requerimientos:}
\begin{itemize}
    \item El sensor debe ser capaz de enviar un dato al controlador cada \SI{5}{\milli\second}
    \item El controlador debe promediar las mediciones de presión recibidas y enviar un dato de altitud cada 5 segundos al usuario
    \item La resolución debe ser menor a 0.04 \si{\milli bar}
    \item El sensor solo será utilizado en altitudes menores a los 2000 msnm
\end{itemize}

\noindent\textbf{Respoda lo siguiente:}

\begin{enumerate}
    \item Según lo visto en clase y la información disponible en la hoja de datos. ¿Que tipo de sensor es este?
    \item Seleccione la relación de sobremuestreo y el rango de operación del sensor en base a los requerimientos e indíquelo en una tabla.
    \item Realice una tabla resumen con las siguientes características del sensor para la relación de sobremuestreo y rango seleccionados
    \begin{itemize}
        \item Rango
        \item Tiempo de respuesta
        \item Resolución
        \item Exactitud
    \end{itemize}
    \item Se tienen un grupo de mediciones del sensor de presión promediadas cada 5 segundos la cuales corresponden a un viaje en automóvil que se realizó desde Cartago centro a Tobosi de Cartago, ida y vuelta. Los datos se pueden descargar \href{https://estudianteccr-my.sharepoint.com/:u:/g/personal/prof_juan_rojas_estudiantec_cr/ES5u-pJPlFNBnMxKUXgPvi0B6mYtfNTDfYd4Du76JagNqA?e=5g6mPB}{aquí}. Investigue como calcular la altitud a partir de estos datos, calcúlela y genere una gráfica que relacione el tiempo en minutos con la altitud. 
\end{enumerate}
\end{document}