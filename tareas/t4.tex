\documentclass[12pt]{article}
\makeatletter
\def\input@path{{../common/}{../guide/}{../data/}{../code/}}
\makeatother
\usepackage{amsmath}
\usepackage{amssymb}
\usepackage{siunitx}
\usepackage{float}
\usepackage{tikz}
\usepackage{url}
\usepackage[siunitx,american,RPvoltages]{circuitikz}
\ctikzset{capacitors/scale=0.7}
\ctikzset{diodes/scale=0.7}
\usepackage{tabularx}
\newcolumntype{C}{>{\centering\arraybackslash}X}
\renewcommand\tabularxcolumn[1]{m{#1}}% for vertical centering text in X column
\usepackage{tabu}
\usepackage[spanish,es-tabla,activeacute]{babel}
\usepackage{babelbib}
\usepackage{booktabs}
\usepackage{pgfplots}
\usepackage{hyperref}
\hypersetup{colorlinks = true,
            linkcolor = black,
            urlcolor  = blue,
            citecolor = blue,
            anchorcolor = blue}
\usepgfplotslibrary{units, fillbetween} 
\pgfplotsset{compat=1.16}
\usepackage{bm}
\usetikzlibrary{arrows, arrows.meta, shapes, 3d, perspective, positioning}
\renewcommand{\sin}{\sen} %change from sin to sen
\usepackage{bohr}
\setbohr{distribution-method = quantum,insert-missing = true}
\usepackage{elements}
\usepackage{verbatim}
\pagestyle{fancy}
\setlength{\headheight}{42pt}


\begin{document}
\lhead{Ingeniería Física \\ Escuela de Física \\ Tecnológico de Costa Rica} 
\rhead{Instrumentación I \\ Tarea \#4  \\ Entrega: Semana 14} 
\cfoot{\thepage\ de \pageref{LastPage}}
\setlength{\parindent}{0em}

La \href{https://www.osha.gov/}{Occupational Safety and Health Administration (OSHA)}, de los EEUU, es uno de los principales entes rectores en temas de salud ocupacional.
Un aspecto a destacar es la iluminación requerida en los puestos de trabajo de acuerdo a la actividad, que está regulado en el estándar \href{https://www.osha.gov/laws-regs/regulations/standardnumber/1926/1926.56}{1926.56.}

Para oficinas de trabajo, la iluminación mínima requerida es 323 \si{\lux}. 
Considerando la popularización del teletrabajo se proponer verificar la iluminancia de el lugar de trabajo del estudiante.

Para ello:

\begin{itemize}
    \item Instale la aplicación \textit{Arduino Science Journal} en su teléfono móvil
    \item Configure una toma de datos con el sensor de luz.
    \item Antes de arrancar la medición, asegúrese su teléfono está completamente acomodado en la mesa, ya que las orientaciones del teléfono pueden dar datos erróneos.
    \item Realice cinco juegos de mediciones con al menos 30s de duración cada una: 
    \begin{itemize}
        \item De mañana con luz natural
        \item De mañana con luz natural y artificial
        \item En la tarde con luz natural
        \item En la tarde con luz natural y artificial
        \item En la noche con luz artificial
    \end{itemize}
    \item Exporte los valores registrados en un archivo \emph{.csv}
    \item Usando la librería \emph{pandas} en Python lea los datos del archivo .csv y limpie los datos de forma que pueda realizar el siguiente paso.   
    \item Determine el promedio y la desviación estandar de cada set de iluminancia.
    \item Usando como valor base lo determinado por \textit{OSHA}, calcule la diferencia entre lo medido y lo recomendado como un error relativo.
    \item Discuta los resultados obtenidos y proponga recomendaciones para mejorar la iluminación en su lugar de trabajo en caso de ser necesario.
\end{itemize}

\noindent\textbf{Entregable}

Un solo archivo .zip que contenga lo siguiente
\begin{itemize}
    \item Un archivo \emph{.py} que incluya comentarios aclaratorios de los pasos que se realizaron y cinco gráficas. Una por cada set de medición.
    \item Un archivo \emph{.pdf} donde se analicen los datos obtenidos.
    \item Cinco archivos \emph{.csv} con los datos de las mediciones
\end{itemize}

\end{document}